\documentclass[10pt,a4paper,oneside]{book}
\usepackage{listings}
\usepackage{fontspec}
\usepackage{xeCJK}
\usepackage{xcolor}
\usepackage{graphicx}
\usepackage{fancyhdr}
\usepackage{bm}
\usepackage{geometry}
\geometry{left=2.0cm,right=2.0cm,top=2.0cm,bottom=2.0cm}
\setmonofont{Consolas}
\setsansfont{Consolas}
\pagestyle{fancy}
\setcounter{tocdepth}{4}
\setcounter{secnumdepth}{3}
\lstset{
    language    = c++,
    breaklines  = true,
    captionpos  = b,
    tabsize     = 4,
    numbers     = left,
    columns     = fullflexible,
    keepspaces  = true,
    commentstyle = \color[RGB]{0,128,0},
    keywordstyle = \color[RGB]{0,0,255},
    basicstyle   = \small\ttfamily,
    rulesepcolor = \color{red!20!green!20!blue!20},
    showstringspaces = false,
}
\title{
    \begin{center}
        \includegraphics[width=4in]{logo.png}
        \\ 
        \textbf{ICPC Template Manual}
    \end{center}
}
\author{
    \includegraphics[width=3cm]{author.png}
    \\
    \textbf{作者:贺梦杰}
}
\date{\today}

\begin{document}
    \maketitle
    \tableofcontents

    \chapter{基础}
    \include{tex/基础/基础}
    %基础部分

    % \chapter{搜索}
    % %搜索部分

    \chapter{动态规划}
    \section{背包DP}

\subsection{0-1背包 洛谷P1048 采药}
    \subsubsection{题目描述}
        有$n (1\le n\le 100)$株药草,每一株有一个采摘时长$a_i$和价值$b_i$ $(1\le a_i,b_i\le 100)$,给你$m (1\le m\le 1000)$时间,问最大价值是多少?
    \subsubsection{代码}
\begin{lstlisting}
#include <bits/stdc++.h>

using namespace std;
const int N = 1e5 + 10;

int n, m;            // 物品数量,最大容量
int f[N];            // f[i]表示用i容量能得到的最大价值
int c[110], v[110];  // 每一个物品的消耗和价值

int main() {
    ios::sync_with_stdio(0);
    cin.tie(0);

    int i, j;
    cin >> m >> n;
    for (i = 1; i <= n; i++)
        cin >> c[i] >> v[i];

    // 对于每个物品i,恰好使用j容量时,不取或取a[i]
    // 注意j要倒着更新
    for (i = 1; i <= n; i++)
        for (j = m; j >= c[i]; j--)
            f[j] = max(f[j], f[j - c[i]] + v[i]);

    int ans = 0;
    for (i = 1; i <= m; i++)
        ans = max(ans, f[i]);

    cout << ans << endl;

    return 0;
}
\end{lstlisting}

\subsection{完全背包 洛谷P1616 疯狂的采药}
    \subsubsection{题目描述}
        与上一题描述差不多,但此题和原题的不同点:\\
        1.每种草药可以无限制地疯狂采摘。\\
        2.药的种类眼花缭乱,采药时间好长好长啊!\\
        另外,物品数$n (1\le n\le 10000)$,最大容量$m (1\le m\le 100000)$,物品消耗和价值$(1\le a_i,b_i\le 10000)$,
        并且$n*m\le 10^7$
    \subsubsection{代码1,直接枚举每种取的数量}
        大概率TLE,但还是要放在这,因为有时候可能数据范围不大,但却有别的限制
\begin{lstlisting}
#include <bits/stdc++.h>

using namespace std;
const int N = 1e4 + 10, M = 1e5 + 10;

int n, m;        // 物品数量,最大容量
int f[M];        // f[i]表示用i容量能得到的最大价值
int c[N], v[N];  // 每一个物品的消耗和价值

int main() {
    ios::sync_with_stdio(0);
    cin.tie(0);

    int i, j, k;
    cin >> m >> n;
    for (i = 1; i <= n; i++)
        cin >> c[i] >> v[i];

    // 对于每个物品i,恰好使用j容量时,取k个a[i]
    // 注意j要倒着更新
    for (i = 1; i <= n; i++)
        for (j = m; j >= c[i]; j--)
            for (k = 0; k <= j / c[i]; k++)
                f[j] = max(f[j], f[j - k * c[i]] + k * v[i]);

    int ans = 0;
    for (i = 1; i <= m; i++)
        ans = max(ans, f[i]);

    cout << ans << endl;

    return 0;
}
\end{lstlisting}
    \subsubsection{代码2,巧妙地用无限取这个条件}
        由于可以无限取,所以其实不用关心先后更新
\begin{lstlisting}
// 对于每个物品i,恰好使用j容量时,取k个a[i]
// 注意j的更新顺序,非常妙
for (i = 1; i <= n; i++)
    for (j = c[i]; j <= m; j++)
        f[j] = max(f[j], f[j - c[i]] + v[i]);
}
\end{lstlisting}

\subsection{分组背包}
    \subsubsection{问题描述}
        有N件物品和一个容量为V的背包。第i件物品的费用是c[i],价值是w[i]。这些物品被划分为若干组,每组中的物品互相冲突,最多选一件。求解将哪些物品装入背包可使这些物品的费用总和不超过背包容量,且价值总和最大。
    \subsubsection{解决方案}
        状态空间:f[k][v]表示前k组物品花费费用v能取得的最大权值\\
        状态转移方程:\\
        f[k][v]=max{f[k-1][v],f[k-1][v-c[i]]+w[i]|物品i属于第k组}\\
        使用一维数组的伪代码如下:
\begin{lstlisting}
for 所有的组k
    for v=V..0
        for 所有的i属于组k
            f[v]=max{f[v],f[v-c[i]]+w[i]}
\end{lstlisting}

\subsection{多重背包}
    \subsubsection{问题描述}
        有 N 种物品和一个容量是 V 的背包。\\
        第 i 种物品最多有 si 件,每件体积是 vi,价值是 wi。\\
        求解将哪些物品装入背包,可使物品体积总和不超过背包容量,且价值总和最大。\\
        输出最大价值。
    \subsubsection{解法1:直接枚举}
        与完全背包的暴力解法相似,只要加上k<=s[i]这个条件
    \subsubsection{解法2:二进制优化}
\begin{lstlisting}
#include <bits/stdc++.h>

using namespace std;
const int N = 16 * 1e3 + 10, M = 2e3 + 10;

int n, m;              // 物品数量,最大容量
int f[M];              // f[i]表示用i容量能得到的最大价值
int s[N], v[N], w[N];  // 每一个物品的个数、消耗和价值

int main() {
    ios::sync_with_stdio(0);
    cin.tie(0);

    int i, j, cnt;
    cin >> n >> m;
    for (i = 1; i <= n; i++)
        cin >> v[i] >> w[i] >> s[i];

    // 二进制优化
    cnt = n;
    for (i = 1; i <= n; i++) {
        int t = s[i], d = 1;
        while (t >= d)
            v[++cnt] = v[i] * d, w[cnt] = w[i] * d, t -= d, d <<= 1;
        if (t)
            v[++cnt] = v[i] * t, w[cnt] = w[i] * t;
    }

    // 0-1背包DP
    // 注意i要从n+1开始,因为前n个是未拆分的,如果算上那么每个物品的数量就被错误的乘2了
    for (i = n + 1; i <= cnt; i++)
        for (j = m; j >= v[i]; j--)
            f[j] = max(f[j], f[j - v[i]] + w[i]);

    int ans = 0;
    for (i = 1; i <= m; i++)
        ans = max(ans, f[i]);

    cout << ans << endl;

    return 0;
}
\end{lstlisting}
    \subsubsection{解法3:单调队列优化}
\begin{lstlisting}
#include <bits/stdc++.h>

using namespace std;
const int N = 1e3 + 10, M = 2e4 + 10;

int n, m;              // 物品数量,最大容量
int f[M];              // f[i]表示用i容量能得到的最大价值
int s[N], v[N], w[N];  // 每一个物品的个数、消耗和价值

struct CycQue {
    const static int MAXSZ = M;
    typedef pair<int, int> Ele;
    Ele q[MAXSZ];
    int head, tail, sz;
    CycQue()
        : head(1), tail(0), sz(0) {}
    void clear() {
        head = 1, tail = 0, sz = 0;
    }
    bool empty() {
        return !sz;
    }
    void push_back(Ele x) {
        // 队列满了
        if (sz >= MAXSZ - 1)
            sz = sz / (sz - sz);  // 除0,抛出异常
        tail = (tail + 1) % MAXSZ, q[tail] = x, ++sz;
    }
    void push_front(Ele x) {
        if (sz >= MAXSZ - 1)
            sz = sz / (sz - sz);
        head = (head - 1 + MAXSZ) % MAXSZ, q[head] = x, ++sz;
    }
    void pop_front() {
        if (!sz)
            sz = sz / (sz - sz);
        head = (head + 1) % MAXSZ, --sz;
    }
    void pop_back() {
        if (!sz)
            sz = sz / (sz - sz);
        tail = (tail - 1 + MAXSZ) % MAXSZ, --sz;
    }
    Ele front() {
        if (!sz)
            sz = sz / (sz - sz);
        return q[head];
    }
    Ele back() {
        if (!sz)
            sz = sz / (sz - sz);
        return q[tail];
    }
} q;

int main() {
    ios::sync_with_stdio(0);
    cin.tie(0);

    int i, j, b;
    scanf("%d%d", &n, &m);
    for (i = 1; i <= n; i++)
        scanf("%d%d%d", &v[i], &w[i], &s[i]);

    // 枚举物品
    for (i = 1; i <= n; i++) {
        // 枚举余数
        for (b = 0; b < v[i]; b++) {
            q.clear();
            // 枚举当前取的个数
            for (j = 0; j <= (m - b) / v[i]; j++) {
                int t = f[b + j * v[i]] - j * w[i];
                // 入队
                while (!q.empty() && q.back().second <= t)
                    q.pop_back();
                q.push_back({j, t});
                while (j - q.front().first > s[i])
                    q.pop_front();
                f[b + j * v[i]] = q.front().second + j * w[i];
            }
        }
    }

    int ans = 0;
    for (i = 1; i <= m; i++)
        ans = max(ans, f[i]);

    printf("%d\n", ans);

    return 0;
}
\end{lstlisting}

\subsection{杭电多校2019第一场C HDU6580 Milk}
    \subsubsection{题目描述}
        有一个n*m的矩阵,左右可以随便走,但只能在每一行的中点往下走,每走一格花费时间1.
        现在这个矩阵里放了k瓶牛奶,第i个牛奶喝下去需要ti时间
        起点是(1,1)
        对于每个i∈[1,k],问喝掉k瓶牛奶花费的最小时间
    \subsubsection{解决方案}
        首先对瓶子的横坐标离散化处理,一行一行地更新答案。
        每到新的一行,先预处理出向左(右)走喝了\(i\)瓶水后,回到原点以及不回到原点所需要的最短时间,分别记为\(l,r,l\_back,r\_back\)。
        然后将左右两边合并,求出喝了\(i\)瓶水后,回到或不回到原点所需时间,记为\(g,g\_back\)。
        设\(f[i]\)为从\((1,1)\)出发,喝了\(i\)瓶水后回到当前行中点所需的最短时间,
        则这时就可以根据\(g\)来和之前的\(f[i]\)来更新答案,并用\(g\_back\)来更新\(f\)的值。时间复杂度为\(O(k^2)\)

\subsection{ICPC上海2019网络预选赛 J-Stone game}
    \subsubsection{题目描述}
        有n个石头,每个石头都有一个重量为$a_i$。现要取出一部分,我们设它们为集合A,则剩余部分为集合B,要求重量$A\ge B$,
        并且当移去A中的任意一个石头$t$时,要求$(A-t)\le B$。问取法有多少种,最后输出模$1e9+7$\\
        其中$n\le 300, a_il\le 500$,样例数$T\le 100$
    \subsubsection{解决方案}
        容易发现,当划分出A之后,只要移除A中\textbf{最轻的石头t}是满足题意的这种方案就是可行的。\\
        于是我们可以先将\textbf{数组排序,然后枚举t},对每一个t分别考虑。最暴力的想法是考虑t后面的每一个$a_i$取或不取,
        但是这样复杂度上天。\\
        再仔细想想,就会发现所有$a_i$加起来也就$1.5e5$,也就是可以用\textbf{背包dp},$f(i,j)$表示$t=i$时总重量为j的方案数。
        复杂度是?不考虑T,枚举t为$O(N)$,对于每个t,考虑其后的每个数取或不取为$O(N*1.5e5)$,总复杂度为$O(N^2*1.5*10^5)$,又上天了。\\
        但是当我们\textbf{倒过来想},\textbf{将数组从大到小排序},然后从前向后枚举t,就会发现复杂度降到了$O(N*1.5*10^5)$。
        每次$++t$,都相当于整体加上$a_t$($a_t$必取),然后再考虑不取$a_{t-1}$的情况($a_{t-1}$在上一个t中是必选的)。
    \section{位置DP}
这类dp一般都会以数字出现位置作为一个维度。

\subsection{杭电多校2019第一场A HDU6578 Blank}
    \subsubsection{题目描述}
        有 n (≤100) 个格子,向其中填入 0、1、2、3 这4个数,但是有 m (≤100) 个限制\\
        限制 l r x :表示 l ~ r 的格子内不同的数的个数为x\\        
        问一共有多少种填入方案?
    \subsubsection{解决方案}
        \textbf{初步分析:}
        构建dp[i][j][k][t][cur],表示到cur位置为止,0,1,2,3四个数最后出现在位置i,j,k,t的情况数。\\
        dp[cur][j][k][t][cur]=dp[cur][j][k][t][cur]+dp[i][j][k][t][cur−1]\\
        dp[i][cur][k][t][cur]=dp[i][cur][k][t][cur]+dp[i][j][k][t][cur−1]\\
        dp[i][j][cur][t][cur]=dp[i][j][cur][t][cur]+dp[i][j][k][t][cur−1]\\
        dp[i][j][k][cur][cur]=dp[i][j][k][cur][cur]+dp[i][j][k][t][cur−1]\\
        \textbf{(注意以上求法是对每个待求dp分别求,等号后面第二项需要迭代)}\\
        这样当cur==r,只要依据四个数出现位置是否大于等于l就可以判断有几个不同的数,若不满足限制则令dp等于0。\\\\
        \textbf{优化1:}
        我们发现上面状态中一定会有一个cur,即有一维度是不会变的,但是这一维度的位置一直在变。\\
        那么就可以构建 dp[i][j][k][cur] ,其中 i < j < k < cur (若为0可等于): 
        只知道 i, j, k, cur 是不同的数最后出现的位置,且i < j < k < cur (并不需要知道 i,j,k,cur对应的填入数是哪个,因为这并不影响对限制的判断)\\
        dp[j][k][cur−1][cur]=dp[j][k][cur−1][cur]+dp[i][j][k][cur−1]\\
        dp[i][k][cur−1][cur]=dp[i][k][cur−1][cur]+dp[i][j][k][cur−1]\\
        dp[i][j][cur−1][cur]=dp[i][j][cur−1][cur]+dp[i][j][k][cur−1]\\
        dp[i][j][k][cur]=dp[i][j][k][cur]+dp[i][j][k][cur−1]\\
        \textbf{(注意以上求法是对每一个已知dp去累加至待求dp,因此不需要迭代}\\
        分别表示 将最后一次出现位置为 i, j, k ,cur-1的数填入第cur个格子中\\\\
        \textbf{优化2:}
        我们发现上面状态中最后一个维度要么为cur要么为cur-1,即当前状态仅与上一个状态有关,所以可以用滚动数组优化空间。\\
        构建 dp[i][j][k][2],得到状态转移方程:\\
        dp[j][k][cur−1][now]=dp[j][k][cur−1][now]+dp[i][j][k][pre]\\
        dp[i][k][cur−1][now]=dp[i][k][cur−1][now]+dp[i][j][k][pre]\\
        dp[i][j][cur−1][now]=dp[i][j][cur−1][now]+dp[i][j][k][pre]\\
        dp[i][j][k][now]=dp[i][j][k][now]+dp[i][j][k][pre]\\
        注意: 因为这里是累加,所以用滚动数组时,dp[i][j][k][pre]用完以后要清空!

\subsection{杭电多校2019第一场J HDU6587 Kingdom}
    \subsubsection{题目描述}
        给出n(<=100)个点的树的前序遍历和中序遍历,这两个序列中某些编号未知(用0来表示)。
        问有多少种合法的树。
    \subsubsection{解决方案}
        位置DP,二叉树前序和中序遍历的关系\\\\
        最关键的不是dp的状态和方程,而是限制,\textbf{要求左子树和右子树的节点必须在对应的区间内(即不交叉,不出界)}\\
        \textbf{记录a中节点在b中出现的位置,b中节点在a中出现的位置。}\\
        \textbf{在每进入一个新的区间时,都检查一遍,是否a的区间中所有节点都在b的对应区间中,同理反过来再做一次}\\
    \include{tex/动态规划/树形DP}
    \section{最长上升子序列}

    \subsection{基本实现}
\begin{lstlisting}
#include <bits/stdc++.h>

using namespace std;
const int N = 1e3 + 10;

int n, a[N];
int b[N], c[N];  // b[i]是以a[i]为右端点的最长上升子序列的长度,c[i]是长度为i的最长上升子序列的右端点的最小值

inline void solve() {
    int i, p, len = 0;
    for (i = 1; i <= n; i++) {
        p = lower_bound(c + 1, c + 1 + len, a[i]) - c;
        b[i] = p, c[p] = a[i], len = max(len, p);
    }
}

int main() {
    ios::sync_with_stdio(0);
    cin.tie(0);

    int i;
    cin >> n;
    for (i = 1; i <= n; i++)
        cin >> a[i];
    solve();
    int ans = 0;
    for (i = 1; i <= n; i++)
        ans = max(ans, b[i]);
    cout << ans << endl;

    return 0;
}
\end{lstlisting}

    \subsection{另一种解法:权值线段树}
        对于a[i],我们可以通过查找以比a[i]小的数为结尾的最长长度,然后+1即可得到以a[i]为结尾的最长长度。\\
        权值线段树维护以某权值为结尾的最大长度。注意需要离散化。\\
        由于是查前缀,所以权值线段树可以换用树状数组
\begin{lstlisting}
#include <bits/stdc++.h>

using namespace std;
const int N = 1e5 + 10;

int n, a[N];
int b[N];    // b[i]是以a[i]为右端点的最长上升子序列的长度
int i2x[N];  // 离散化
int m;       // 离散化的长度
int t[N];    // 树状数组

inline int x2i(int x) {
    return lower_bound(i2x + 1, i2x + 1 + m, x) - i2x;
}

inline int lowbit(int x) {
    return -x & x;
}

inline void upd(int p, int v) {
    for (; p <= m; p += lowbit(p))
        t[p] = max(t[p], v);
}

inline int qry(int p) {
    int ret = 0;
    for (; p > 0; p -= lowbit(p))
        ret = max(ret, t[p]);
    return ret;
}

inline void solve() {
    int i;
    for (i = 1; i <= n; i++) {
        b[i] = qry(x2i(a[i]) - 1) + 1;
        upd(x2i(a[i]), b[i]);
    }
}

int main() {
    ios::sync_with_stdio(0);
    cin.tie(0);

    int i;
    cin >> n;
    for (i = 1; i <= n; i++)
        cin >> a[i], i2x[i] = a[i];
    sort(i2x + 1, i2x + 1 + n);
    m = unique(i2x + 1, i2x + 1 + n) - i2x - 1;

    solve();

    int ans = 0;
    for (i = 1; i <= n; i++)
        ans = max(ans, b[i]);
    cout << ans << endl;

    return 0;
}
\end{lstlisting}

    \subsection{输出最小字典序的下标}
        原问题是记录了以a[i]为结尾的最长上升子序列的长度,
        而此问题是记录以a[i]为\textbf{开头}的最长上升子序列的长度\\
        PS:若要求输出最大字典序的下标,只要还改成记录以a[i]为\textbf{结尾}的最长上升子序列的长度,然后倒过来筛就可以了
\begin{lstlisting}
#include <bits/stdc++.h>

using namespace std;
const int N = 1e5 + 10;

int n, a[N];
int d[N], c[N];  // d[i]是以a[i]为左端点的最长上升子序列的长度,c[i]用于记录从右向左长度为i的最小左端点

// 此处定义的是小于号,注意,不可以有等于。lower_bound配上<=就是upper_bound。
// 所以无论是lower还是upper都不带等号
bool cmp(const int& lhs, const int& rhs) {
    return lhs > rhs;
}

inline void solve() {
    int i, len, mx = 0;
    for (i = n; i >= 1; i--) {
        len = lower_bound(c + 1, c + 1 + mx, a[i], cmp) - c;
        d[i] = len, c[len] = a[i], mx = max(mx, len);
    }
}

int main() {
    ios::sync_with_stdio(0);
    cin.tie(0);

    int i;
    cin >> n;
    for (i = 1; i <= n; i++)
        cin >> a[i];

    solve();

    int mx = 0;
    for (i = 1; i <= n; i++)
        mx = max(mx, d[i]);

    cout << mx << endl;

    int cur = mx, last = 0;
    for (i = 1; i <= n; i++) {
        if (d[i] == cur && a[i] > last) {
            if (cur != mx)
                cout << " ";
            cout << i;  // 输出下标
            --cur, last = a[i];
        }
    }
    cout << endl;

    return 0;
}
\end{lstlisting}

    \subsection{输出值的最小字典序}
        与上面问题不同的是,这里要求的是值的最小字典序。\\
        例如:6 7 8 9 1 2 3 4,我们要输出的是1 2 3 4。\\
        对于这个问题,我们只要记录每个数的前驱即可。
\begin{lstlisting}
#include <bits/stdc++.h>

using namespace std;
const int N = 1e5 + 10;

int n, a[N];
int b[N], c[N];      // b[i]是以a[i]为右端点的最长上升子序列的长度,c[i]用于记录长度为i的最小端点
int pos[N], pre[N];  // pos[i]是与c[i]对应的下标,pre[i]是a[i]的前驱的下标
int ans[N];
// a[] b[] pre[] 下标一致;c[] pos[] 下标一致

inline void solve() {
    int i, len, mx = 0;
    for (i = 1; i <= n; i++) {
        len = lower_bound(c + 1, c + 1 + mx, a[i]) - c;
        b[i] = len, pre[i] = pos[len - 1];
        c[len] = a[i], pos[len] = i;
        mx = max(mx, len);
    }
}

int main() {
    ios::sync_with_stdio(0);
    cin.tie(0);

    int i, T;
    cin >> T;
    while (T--) {
        cin >> n;
        for (i = 1; i <= n; i++)
            cin >> a[i];

        solve();

        int mx = 0;
        for (i = 1; i <= n; i++)
            mx = max(mx, b[i]);

        cout << mx << endl;

        int p = pos[mx], len = mx;
        while (p) {
            ans[len--] = a[p];
            p = pre[p];
        }

        for (i = 1; i <= mx; i++) {
            if (i > 1)
                cout << " ";
            cout << ans[i];
        }
        cout << endl;
    }

    return 0;
}
\end{lstlisting}


    \subsection{最长不降子序列}
\begin{lstlisting}
p = upper_bound(c + 1, c + 1 + len, a[i]) - c;
\end{lstlisting}
    \include{tex/动态规划/最长公共子序列}
    \section{约瑟夫问题}
    $n$个人标号$0,1,\dots,n-1$。逆时针站成一圈,从0号开始,每一次从当前的人逆时针数m个,然后让这个人出局。问最后剩下的人是谁。\\
    参考:https://fancypei.github.io/JosephusProblem/
    \subsection{递推法}
        时间$O(n)$,空间$O(1)$\\
        令$f(n)$表示n个人最后的幸存者编号,则递推方程为:
        $$f(n)=(f(n-1)+m)\bmod n$$
        初始条件$f(1)=0$
\begin{lstlisting}
// 共n个人,编号从0到n-1,从编号为0的人开始报数,每报到m的人被抬走(从1开始报数)。返回最后存活的人的编号
inline int solve1(int n, int m) {
    int i, ret = 0;
    for (i = 2; i <= n; i++)
        ret = (ret + m) % i;
    return ret;
}
\end{lstlisting}
    \subsection{另类递归}
        时间$O(logN)$,空间$O(logN)$\\
        一次去掉$\left \lfloor \frac{n}{m} \right \rfloor$个数,其中n是当前剩余个数,m不变
\begin{lstlisting}
// 时空O(logN)
inline int solve0(int n, int m) {
    if (n == 1)
        return 0;
    if (n < m)
        return (solve0(n - 1, m) + m) % n;  // 类似于O(N)递推的f(n)=(f(n-1)+m)%n
    int s = solve0(n - n / m, m) - n % m;
    return s < 0 ? s + n : s + s / (m - 1);
}
\end{lstlisting}
    \subsection{终极方法}
        时间$O(logN)$,空间$O(1)$
        这个方法同时还解决了扩展约瑟夫问题,即第k个被抬走的人的编号是什么?
\begin{lstlisting}
// 时间O(logN),空间O(1)
// 共n人,数m个数,第k个被抬走的(k取[0,n-1]),返回0~n-1中的一个数
inline int solve(int n, int m, int k) {
    int i;
    for (i = (k + 1) * m - 1; i >= n; i = (i - n) + (i - n) / (m - 1))
        ;
    return i;
}
\end{lstlisting}

\section{反向约瑟夫问题}
    提问:共n人,数m个数,编号为p的人第几个被抬走?
    \subsection{解法}
        最简单的还是模拟,用双端队列即可。\\
        还有一种方法,时间$O(MlogN)$。\\
        这个方法隐含在上面“终极方法”的推导过程中,我们固定p点,看每次p报数之前几个人被抬走以及p在本次报数中是否被抬走。
\begin{lstlisting}
// n个人,编号为1~n,报m个数(从1到m),编号为p的人是第几个被抬走的,返回1~n中的一个数
inline int solve(int n, int m, int p) {
    // a是每次p报数之前被抬走的人数
    int a = p / m;
    // 如果整除则代表p在本次报数就是那个被抬走的人
    while (p % m) {
        // 若本次p没被抬走,那么下一次p是第p+n-a个报数的人
        p = p + n - a;
        // 下一次p报数之前被抬走的人数
        a = p / m;
    }
    // 返回被抬走的人数
    return a;
}
\end{lstlisting}
    %动态规划

    \chapter{字符串L}
    \section{Hash}
    Hash 的核心思想在于,暴力算法中,单次比较的时间太长了,应当如何才能缩短一些呢?\\
    如果要求每次只能比较$O(1)$个字符,应该怎样操作呢?\\
    我们定义一个\textbf{把string映射成int}的函数$f$,这个$f$称为是 Hash 函数。\\
    我们需要关注的是时间复杂度和 Hash 的准确率。\\
    通常我们采用的是多项式 Hash 的方法,即
    $$f(s)=\sum{(s[i]*b^i)}(\bmod M)$$
    其中$b$与$M$互质,且M越大错误率越小。(单次匹配错误率$\frac{1}{M}$,n次匹配的错误率为$\frac{n}{M}$)

\subsection{基础Hash匹配}
\begin{lstlisting}
#include <bits/stdc++.h>

using namespace std;
typedef long long ll;
// b和M互质;M可以尽量取大、随机化。对于本问题,无需建立关于M的数组,所以M最大可以达到1<<31大小。
const ll N = 1e3 + 10, b = 131, M = 1 << 20;

char s[2][N];  // s[0]是待匹配串,s[1]是模式串。下标从1开始
ll len[2];     // 两串的长度
ll Exp[N];     // Exp[i]=b^i

// 返回s[l..r]的哈希值 s[l]*Exp[1]+s[l+1]*Exp[2]+..
inline ll Hash(char s[], ll l, ll r) {
    ll i, ret = 0;
    for (i = 1; l + i - 1 <= r; i++)
        ret = (ret + Exp[i] * s[l + i - 1]) % M;
    return ret;
}

vector<ll> ans;  // 匹配子串的开始下标
inline void match() {
    ans.clear();
    ll i;
    ll h0 = Hash(s[0], 1, len[1]);  // 初始时待匹配串对应模式串那部分子串的哈希值
    ll h1 = Hash(s[1], 1, len[1]);  // 初始时模式串的哈希值
    for (i = 1; i <= len[0] - len[1] + 1; i++) {
        // 若两哈希值一致,则认为匹配
        if ((h0 - h1 * Exp[i - 1]) % M == 0)
            ans.push_back(i);
        h0 = (h0 - Exp[i] * s[0][i] + Exp[i + len[1]] * s[0][i + len[1]]) % M;  // 模式串向后移动,对应h0也要改变
    }
}

int main() {
    ios::sync_with_stdio(0);
    cin.tie(0);

    ll i, j;

    // 初始化
    Exp[0] = 1;
    for (i = 1; i < N; i++)
        Exp[i] = Exp[i - 1] * b % M;

    cin >> (s[0] + 1) >> (s[1] + 1);
    len[0] = strlen(s[0] + 1), len[1] = strlen(s[1] + 1);

    match();

    return 0;
}
\end{lstlisting}
    \section{KMP}
    \subsection{前缀函数}
        给定一个长度为n的字符串s(假定下标从1开始),其\textbf{前缀函数}被定义为一个长度为n的数组$\pi$,其中$\pi[i]$为
        既是子串$s[1\dots i]$的前缀同时也是该子串的后缀的最长真前缀(proper prefix)长度。一个字符串的真前缀是其前缀但
        不等于该字符串自身。根据定义,$\pi[1]=0$。\\\\
        前缀函数的定义可用数学语言描述如下:
        $$\pi[i]=\max _{k=0 \ldots i-1}\{k : s[1 \ldots k]=s[i-k+1 \ldots i]\}$$
        举例来说,字符串abcabcd的前缀函数为$[0,0,0,1,2,3,0]$,字符串aabaaab的前缀函数为$[0,1,0,1,2,2,3]$。
        \subsubsection{朴素算法}
            直接按定义计算前缀函数:
            \begin{lstlisting}
// 朴素法求前缀函数O(n^3),下标从1开始
void prefix_func0(char t[], int n) {
    int i, k;
    for (i = 1; i <= n; i++)   // 对每一个子串
        for (k = 0; k < i; k++)  // 枚举前缀后缀长度,并判断是否相等
            if (!strncmp(t + 1, t + i - k + 1, k))
                pi[i] = k;
}
            \end{lstlisting}
        \subsubsection{第一个优化}
            第一个重要的事实是相邻的前缀函数值至多增加1。(如不然,会产生矛盾)\\
            所以当移动到下一个位置时,前缀函数要么增加1,要么不变或减少。
            实际上,该事实已经允许我们将复杂度降至$O(n^2)$。
            因为每一步中前缀函数至多增加1,因此在总的运行过程中,前缀函数至多增加n,同时也至多减小n。
            这意味着我们仅需进行$O(n)$次字符串比较,所以总复杂度为$O(n^2)$。
            \begin{lstlisting}
void prefix_func1(char t[], int n) {
    int i, j;
    i = 2, j = 1;
    while (i <= n) {
        if (t[i] == t[j])  // 加1
            pi[i] = pi[i - 1] + 1, ++j;
        else {  // 开始减
            pi[i] = pi[i - 1];
            while (strncmp(t + i - pi[i] + 1, t + 1, pi[i]))
                --pi[i];
            j = pi[i] + 1;
        }
        ++i;
    }
}
            \end{lstlisting}
        \subsubsection{第二个优化}
            考虑计算位置i+1的前缀函数$\pi$的值,如果$s[i+1]=s[\pi[i]+1]$,显然$\pi[i+1]=\pi[i]+1$。
            $$ \underbrace{\overbrace{s_1 ~ s_2 ~ s_3}^{\pi[i]} ~ \overbrace{s_4}^{s_4 = s_{i+1}}}_{\pi[i+1] = \pi[i] + 1} ~ \dots ~ \underbrace{\overbrace{s_{i-2} ~ s_{i-1} ~ s_{i}}^{\pi[i]} ~ \overbrace{s_{i+1}}^{s_4 = s_{i+1}}}_{\pi[i+1] = \pi[i] + 1} $$
            如果不是上述情况,即$s[i+1] \neq s[\pi[i]+1]$,我们需要尝试更短的字符串。为了加速,我们希望直接移动到
            最长的长度$j<\pi[i]$,使得在位置i的前缀性质仍得以保持,也即$s[1 \dots j] = s[i-j+1 \dots i]$:
            $$\overbrace{\underbrace{s_1 ~ s_2}_j ~ s_3 ~ s_4}^{\pi[i]} ~ \dots ~ \overbrace{s_{i-3} ~ s_{i-2} ~ \underbrace{s_{i-1} ~ s_{i}}_j}^{\pi[i]} ~ s_{i+1}$$
            实际上,如果我们找到了这样的j,我们仅需要再次比较$s[i+1]$和$s[j+1]$。如果它们相等,则$\pi[i+1]=j+1$,
            否则,我们就需要找小于j的最大的新的j使得前缀性质仍然保持,如此反复,直到$s[i+1]=s[j+1]$或者确实完全找不到(令$j=-1$)。
            最后$\pi[i+1]=j+1$。\\\\
            所以我们已经有了一个大致框架,现在仅剩的问题是对于满足$s[1 \dots j] = s[i-j+1 \dots i]$的j,
            如何快速找到小于j的最大的新的j,我们令新的j为k,使得$s[1 \dots k] = s[i-k+1 \dots i]$仍然满足。
            $$\overbrace{\underbrace{s_1 ~ s_2}_k ~ s_3 ~ s_4}^j ~ \dots ~ \overbrace{s_{i-3} ~ s_{i-2} ~ \underbrace{s_{i-1} ~ s_{i}}_k}^j ~s_{i+1}$$
            由上图,我们要求的是比j小的最大的k,而两边长度为j的前后缀本身是相等的,那么新的长为k的前后缀则可以只放到最左边
            长为j的前缀中去考虑:
            $$\overbrace{\underbrace{s_1 ~ s_2}_k ~ \underbrace{s_3 ~ s_4}_k}^j$$
            即$k=\pi[j]$,而$\pi[j]$之前已经求过了。
            \begin{lstlisting}
void prefix_func2(char t[], int n) {
    int i, j;
    pi[0] = -1, pi[1] = 0;  // 确实没有找到任何相等的
    for (i = 1; i < n; ++i) {
        j = pi[i];
        while (j >= 0 && t[j + 1] != t[i + 1])  // 若不相等,找更小的新的j
            j = pi[j];
        pi[i + 1] = j + 1;  //最后得出pi[i+1]
    }
}
            \end{lstlisting}
    
    \subsection{统计每个前缀出现次数}
        

    \chapter{字符串}
    %TODO 字符串Hash
    \section{字符串Hash}
\begin{lstlisting}
#define ull unsigned long long
#define P 131
ull f[N], p[N];
void Init()
{
    p[0] = 1, f[0] = 0;
    for (int i = 1; i <= n; i++)
    {
        p[i] = p[i - 1] * P;
        f[i] = f[i - 1] * P + str[i];
    }
}
ull Hash(int left, int right)
{
    return f[right] - f[left - 1] * p[right - left + 1];
}
\end{lstlisting}


\subsection{应用:后缀数组}

\begin{lstlisting}
#define ull unsigned long long
#define P 131
const int N = 3e5 + 50;
ull f[N], p[N];
char str[N];
int SA[N], n, Height[N];
void Init()
{
    p[0] = 1, f[0] = 0;
    for (int i = 1; i <= n; i++)
    {
        p[i] = p[i - 1] * P;
        f[i] = f[i - 1] * P + str[i];
    }
}
ull Hash(int left, int right)
{
    return f[right] - f[left - 1] * p[right - left + 1];
}
// k:[0,n) 表示后缀S(k,n-1)
// 最长公共前缀
int LCP(int a, int b)
{
    int left = 0;
    int right = N;
    int mid;
    while (left < right)
    {
        mid = (left + right + 1) >> 1;
        if (a + mid - 1 <= n && b + mid - 1 <= n && Hash(a, a + mid - 1) == Hash(b, b + mid - 1))
            left = mid;
        else
            right = mid - 1;
    }
    return left;
}
bool cmp(int a, int b)
{
    int len = LCP(a, b);
    return str[a + len] < str[b + len];
}
void calc_height()
{
    Height[1] = 0;
    for (int i = 2; i <= n; i++)
        Height[i] = LCP(SA[i], SA[i - 1]);
}
int main()
{
    scanf("%s", str + 1);
    n = strlen(str + 1);
    for (int i = 1; i <= n; i++)
        SA[i] = i;
    Init();
    sort(SA + 1, SA + n + 1, cmp);
    calc_height();
}
\end{lstlisting}

\subsection{应用:二维Hash}
给定一个M行N列的01矩阵(只包含数字0或1的矩阵),再执行Q次询问,每次询问给出一个A行B列的01矩阵,求该矩阵是否在原矩阵中出现过。

做法:选取两个不同的P值分别对行列进行Hash处理,应用二维前缀和求取矩阵Hash值。
\begin{lstlisting}
#define P 131
#define Q 13331
#define ull unsigned long long
void Init()
{
    char ch;
    for (int i = 1; i <= m; i++)
        for (int j = 1; j <= n; j++)
            cin >> ch, Hash[i][j] = Hash[i][j - 1] * P + ch;
    for (int i = 1; i <= m; i++)
        for (int j = 1; j <= n; j++)
            Hash[i][j] += Hash[i - 1][j] * Q;
}
ull temp = Hash[i][j] - Hash[i - a][j] * q[a] - Hash[i][j - b] * p[b] + Hash[i - a][j - b] * q[a] * p[b];
\end{lstlisting}

\subsection{应用:一类同构判定的问题}
参考:杨弋《Hash在信息学竞赛中的一类应用》
    %TODO 后缀自动机初步+Manacher+回文树
    \section{后缀自动机}
\begin{lstlisting}
#define ll long long
const int MAXLEN = 1e6 + 50;
struct SAM
{
    int len[MAXLEN << 1], link[MAXLEN << 1], next[MAXLEN << 1][26];
    ll sze[MAXLEN << 1]; ////每个结点所代表的字符串的出现次数
    int sz, last, rt;
    int NewNode(int x = 0)
    {
        len[sz] = x;
        link[sz] = -1;
        memset(next[sz], -1, sizeof(next[sz]));
        return sz++;
    }
    void Init()
    {
        //重置
        sz = last = 0, rt = NewNode();
    }
    void Extend(int c)
    {
        int cur = NewNode(len[last] + 1);
        sze[cur] = 1;
        int p = last;
        while (~p && next[p][c] == -1)
            next[p][c] = cur, p = link[p];
        if (p == -1)
            link[cur] = rt;
        else
        {
            int q = next[p][c];
            if (len[q] == len[p] + 1)
                link[cur] = q;
            else
            {
                int clone = NewNode(len[p] + 1);
                memcpy(next[clone], next[q], sizeof(next[q]));
                link[clone] = link[q], link[q] = link[cur] = clone;
                while (~p && next[p][c] == q)
                    next[p][c] = clone, p = link[p];
            }
        }
        last = cur;
    }
    int id[MAXLEN << 1], c[MAXLEN];
    void Topo()
    {
        //计数排序
        memset(c, 0, sizeof(c));
        for (int i = 0; i < sz; i++)
            c[len[i]]++;
        for (int i = 1; i < MAXLEN; i++)
            c[i] += c[i - 1];
        for (int i = 0; i < sz; i++)
            id[--c[len[i]]] = i;
        for (int i = sz - 1; ~i; i--)
        {
            int u = id[i];
            if (~link[u])
                sze[link[u]] += sze[u];
        }
    }
};
\end{lstlisting}
\section{Manacher}
\begin{lstlisting}
const int MAXLEN = 1e5 + 50;
char ori[MAXLEN], str[MAXLEN * 2];
int d1[MAXLEN * 2], n, m;    
void Manacher()
{
    for (int i = 0, l = 0, r = -1; i < m; i++)
    {
        int k = (i > r) ? 1 : min(d1[l + r - i], r - i);
        while (i - k >= 0 && i + k < m && str[i - k] == str[i + k])
            k++;
        d1[i] = k--;
        if (i + k > r)
            l = i - k, r = i + k;
    }
}
int main()
{
    scanf("%s", ori);
    n = strlen(ori);
    str[0] = '#';
    for (int i = 0; i < n; i++)
    {
        str[(i + 1) * 2] = '#';
        str[(i + 1) * 2 - 1] = ori[i];
    }
    m = n * 2 + 1;
    Manacher();
}
\end{lstlisting}

\section{回文树/回文自动机}
\begin{lstlisting}
const int MAXLEN=5e5+50;
struct Palindromic_Tree
{
    int nxt[MAXLEN][26],fail[MAXLEN],len[MAXLEN],s[MAXLEN];
    int cnt[MAXLEN];// 结点表示的本质不同的回文串的个数(调用Count()后) 
    int num[MAXLEN];// 结点表示的最长回文串的最右端点为回文串结尾的回文串个数 
    int last,sz,n;
    int NewNode(int x)
    {
        memset(nxt[sz],0,sizeof(nxt[sz]));
        cnt[sz]=num[sz]=0,len[sz]=x;
        return sz++;
    }
    void Init()
    {
        sz=0;
        NewNode(0),NewNode(-1);
        last=n=0,s[0]=-1,fail[0]=1;
    }
    int GetFail(int u)
    {
        while(s[n-len[u]-1]!=s[n]) u=fail[u];
        return u;
    }
    void Add(int c)
    {
        //c-='a'
        s[++n]=c;
        int u=GetFail(last);
        if(!nxt[u][c])
        {
            int np=NewNode(len[u]+2);
            fail[np]=nxt[GetFail(fail[u])][c];
            num[np]=num[fail[np]]+1;
            nxt[u][c]=np;
        }
        last=nxt[u][c];
        cnt[last]++;
    }
    void Count()
    {
        for(int i=sz-1;~i;i--)
            cnt[fail[i]]+=cnt[i];
    }
}
\end{lstlisting}
    %TODO KMP
    \section{KMP}
    \subsection{前缀函数}
        给定一个长度为n的字符串s(假定下标从1开始),其\textbf{前缀函数}被定义为一个长度为n的数组$\pi$,其中$\pi[i]$为
        既是子串$s[1\dots i]$的前缀同时也是该子串的后缀的最长真前缀(proper prefix)长度。一个字符串的真前缀是其前缀但
        不等于该字符串自身。根据定义,$\pi[1]=0$。\\\\
        前缀函数的定义可用数学语言描述如下:
        $$\pi[i]=\max _{k=0 \ldots i-1}\{k : s[1 \ldots k]=s[i-k+1 \ldots i]\}$$
        举例来说,字符串abcabcd的前缀函数为$[0,0,0,1,2,3,0]$,字符串aabaaab的前缀函数为$[0,1,0,1,2,2,3]$。
        \subsubsection{朴素算法}
            直接按定义计算前缀函数:
            \begin{lstlisting}
// 朴素法求前缀函数O(n^3),下标从1开始
void prefix_func0(char t[], int n) {
    int i, k;
    for (i = 1; i <= n; i++)   // 对每一个子串
        for (k = 0; k < i; k++)  // 枚举前缀后缀长度,并判断是否相等
            if (!strncmp(t + 1, t + i - k + 1, k))
                pi[i] = k;
}
            \end{lstlisting}
        \subsubsection{第一个优化}
            第一个重要的事实是相邻的前缀函数值至多增加1。(如不然,会产生矛盾)\\
            所以当移动到下一个位置时,前缀函数要么增加1,要么不变或减少。
            实际上,该事实已经允许我们将复杂度降至$O(n^2)$。
            因为每一步中前缀函数至多增加1,因此在总的运行过程中,前缀函数至多增加n,同时也至多减小n。
            这意味着我们仅需进行$O(n)$次字符串比较,所以总复杂度为$O(n^2)$。
            \begin{lstlisting}
void prefix_func1(char t[], int n) {
    int i, j;
    i = 2, j = 1;
    while (i <= n) {
        if (t[i] == t[j])  // 加1
            pi[i] = pi[i - 1] + 1, ++j;
        else {  // 开始减
            pi[i] = pi[i - 1];
            while (strncmp(t + i - pi[i] + 1, t + 1, pi[i]))
                --pi[i];
            j = pi[i] + 1;
        }
        ++i;
    }
}
            \end{lstlisting}
        \subsubsection{第二个优化}
            考虑计算位置i+1的前缀函数$\pi$的值,如果$s[i+1]=s[\pi[i]+1]$,显然$\pi[i+1]=\pi[i]+1$。
            $$ \underbrace{\overbrace{s_1 ~ s_2 ~ s_3}^{\pi[i]} ~ \overbrace{s_4}^{s_4 = s_{i+1}}}_{\pi[i+1] = \pi[i] + 1} ~ \dots ~ \underbrace{\overbrace{s_{i-2} ~ s_{i-1} ~ s_{i}}^{\pi[i]} ~ \overbrace{s_{i+1}}^{s_4 = s_{i+1}}}_{\pi[i+1] = \pi[i] + 1} $$
            如果不是上述情况,即$s[i+1] \neq s[\pi[i]+1]$,我们需要尝试更短的字符串。为了加速,我们希望直接移动到
            最长的长度$j<\pi[i]$,使得在位置i的前缀性质仍得以保持,也即$s[1 \dots j] = s[i-j+1 \dots i]$:
            $$\overbrace{\underbrace{s_1 ~ s_2}_j ~ s_3 ~ s_4}^{\pi[i]} ~ \dots ~ \overbrace{s_{i-3} ~ s_{i-2} ~ \underbrace{s_{i-1} ~ s_{i}}_j}^{\pi[i]} ~ s_{i+1}$$
            实际上,如果我们找到了这样的j,我们仅需要再次比较$s[i+1]$和$s[j+1]$。如果它们相等,则$\pi[i+1]=j+1$,
            否则,我们就需要找小于j的最大的新的j使得前缀性质仍然保持,如此反复,直到$s[i+1]=s[j+1]$或者确实完全找不到(令$j=-1$)。
            最后$\pi[i+1]=j+1$。\\\\
            所以我们已经有了一个大致框架,现在仅剩的问题是对于满足$s[1 \dots j] = s[i-j+1 \dots i]$的j,
            如何快速找到小于j的最大的新的j,我们令新的j为k,使得$s[1 \dots k] = s[i-k+1 \dots i]$仍然满足。
            $$\overbrace{\underbrace{s_1 ~ s_2}_k ~ s_3 ~ s_4}^j ~ \dots ~ \overbrace{s_{i-3} ~ s_{i-2} ~ \underbrace{s_{i-1} ~ s_{i}}_k}^j ~s_{i+1}$$
            由上图,我们要求的是比j小的最大的k,而两边长度为j的前后缀本身是相等的,那么新的长为k的前后缀则可以只放到最左边
            长为j的前缀中去考虑:
            $$\overbrace{\underbrace{s_1 ~ s_2}_k ~ \underbrace{s_3 ~ s_4}_k}^j$$
            即$k=\pi[j]$,而$\pi[j]$之前已经求过了。
            \begin{lstlisting}
void prefix_func2(char t[], int n) {
    int i, j;
    pi[0] = -1, pi[1] = 0;  // 确实没有找到任何相等的
    for (i = 1; i < n; ++i) {
        j = pi[i];
        while (j >= 0 && t[j + 1] != t[i + 1])  // 若不相等,找更小的新的j
            j = pi[j];
        pi[i + 1] = j + 1;  //最后得出pi[i+1]
    }
}
            \end{lstlisting}
    
    \subsection{统计每个前缀出现次数}
        


    \chapter{数据结构}
    \include{tex/数据结构/并查集}
    \include{tex/数据结构/树状数组}
    \include{tex/数据结构/线段树}
    \include{tex/数据结构/主席树}
    \section{带Lazy标记的线段树}
    以下是区间修改+区间最大值查询。\\
    若是区间修改+\textbf{区间和查询},则在pushdown时需要将lazy标记\textbf{乘上区间长度}加到结点上。
\begin{lstlisting}
#include <bits/stdc++.h>

using namespace std;
const int N = 1e5 + 10;

int n;  // 长度

struct Node {
    int l, r, m;
    int mx;   // [l,r)中的最大值
    int tag;  // lazy标记
} t[4 * N];   // 数组大小不要忘记 * 4

// 将lazy标记下推
inline void pushdown(int x) {
    Node& cur = t[x];
    if (cur.r - cur.l == 1)
        return;
    Node &lch = t[x * 2], &rch = t[x * 2 + 1];
    lch.mx += cur.tag, rch.mx += cur.tag;
    lch.tag += cur.tag, rch.tag += cur.tag;
    cur.tag = 0;
}

// 由x的儿子更新x结点,此时应确保x的儿子为最新
inline void pushup(int x) {
    Node& cur = t[x];
    if (cur.r - cur.l == 1)
        return;
    Node &lch = t[x * 2], &rch = t[x * 2 + 1];
    cur.mx = max(lch.mx, rch.mx);
}

// 建树。注意初始时叶子是否为0,若不是,需要pushup
void build(int l, int r, int x) {
    Node& cur = t[x];
    cur.l = l, cur.r = r, cur.m = (l + r) / 2;
    cur.mx = 0, cur.tag = 0;  // 初始化最大值、lazy标记
    if (r - l == 1)
        return;
    build(l, cur.m, x * 2);
    build(cur.m, r, x * 2 + 1);
    pushup(x);  // 若初始值非0,则这句一定要加
}

// [l,r)每个元素加v。注意打标记时应更新被打标记结点的mx,并且应立即pushdown,pushup,过程中也需要pushdown,pushup
void update(int l, int r, int x, int v) {
    Node& cur = t[x];
    if (cur.l == l && cur.r == r) {
        cur.tag += v, cur.mx += v;  // 打标记的时候一定是同时更新max的
        pushdown(x), pushup(x);
        return;
    }
    pushdown(x);
    if (r <= cur.m)
        update(l, r, x * 2, v);
    else if (l >= cur.m)
        update(l, r, x * 2 + 1, v);
    else
        update(l, cur.m, x * 2, v), update(cur.m, r, x * 2 + 1, v);
    pushup(x);
}

// 查询[l,r)的最大值。过程中注意pushdown
int query(int l, int r, int x) {
    Node& cur = t[x];
    pushdown(x);
    if (cur.l == l && cur.r == r)
        return cur.mx;
    int mx = 0;
    if (r <= cur.m)
        mx = query(l, r, x * 2);
    else if (l >= cur.m)
        mx = query(l, r, x * 2 + 1);
    else
        mx = max(query(l, cur.m, x * 2), query(cur.m, r, x * 2 + 1));
    return mx;
}

int main() {
    ios::sync_with_stdio(0);
    cin.tie(0);

    int m, i, l, r, v;
    cin >> n >> m;
    build(1, n + 1, 1);  // 不要忘记build初始化,[l,r+1)
    for (i = 1; i <= m; i++) {
        cin >> v >> l >> r;
        if (!v)
            cout << query(l, r + 1, 1) << endl;
        else
            update(l, r + 1, 1, 1);
    }

    return 0;
}
\end{lstlisting}
    \section{归并树}
\subsection{简介}
    归并排序,自底向上越来越有序。而归并树则是将归并排序的中间结果保留了下来。\\
    为什么要这么做呢?因为这样当询问区间时,总是可以将询问区间二分成某一次或几次归并排序的中间结果。
\subsection{区间第k小值}
    \subsubsection{问题简述}
        给定n个数,m次查询,每次查询[l,r]内从小到大第k个数,输出这个数。
    \subsubsection{解决方案}
        对这n个数先排序,然后二分尝试,将二分得到的中间值代入归并树中,看是否排名为k,\textbf{对所有排名小于等于k的取最大值}。\\
        建树时间复杂度O(NlogN),查询时间复杂度O(logNlogN)。
\begin{lstlisting}
#include <bits/stdc++.h>

using namespace std;
typedef long long ll;
const int N = 1e5 + 10, INF = 1e9 + 8;

int a[N], n, m;

struct Node {
    int l, r, m;
    vector<int> a;
} t[N * 3];

// 由a[]数组初始化归并树
void build(int l, int r, int x) {
    Node& cur = t[x];
    cur.l = l, cur.r = r, cur.m = (l + r) / 2;
    cur.a.clear();
    if (r - l == 1) {
        cur.a.push_back(a[cur.l]);
        return;
    }
    build(l, cur.m, x * 2);
    build(cur.m, r, x * 2 + 1);
    Node &lch = t[x * 2], &rch = t[x * 2 + 1];
    vector<int>::iterator it1, it2;
    it1 = lch.a.begin(), it2 = rch.a.begin();
    // 合并左右子树
    while (it1 != lch.a.end() && it2 != rch.a.end()) {
        if (*it1 <= *it2)
            cur.a.push_back(*it1), it1++;
        else
            cur.a.push_back(*it2), it2++;
    }
    while (it1 != lch.a.end())
        cur.a.push_back(*it1), it1++;
    while (it2 != rch.a.end())
        cur.a.push_back(*it2), it2++;
}

// 查询在区间[l,r)内比v小的数的个数,稍加修改便可查询大于等于k的最小值
int query(int l, int r, int v, int x) {
    Node& cur = t[x];
    if (cur.l == l && cur.r == r)
        return lower_bound(cur.a.begin(), cur.a.end(), v) - cur.a.begin();
    if (r <= cur.m)
        return query(l, r, v, x * 2);
    else if (l >= cur.m)
        return query(l, r, v, x * 2 + 1);
    else
        return query(l, cur.m, v, x * 2) + query(cur.m, r, v, x * 2 + 1);
}

// 查询区间[l,r)区间内的从小到大第k个值
int QR(int ql, int qr, int k) {
    int l = 1, r = n + 1, m, ans = -INF;
    while (l < r) {
        m = (l + r) / 2;
        int rank = query(ql, qr, a[m], 1) + 1;
        if (rank <= k)  // 可能有相等的值,所以需要<=
            ans = max(ans, a[m]);
        if (rank <= k)
            l = m + 1;
        else
            r = m;
    }
    return ans;
}

int main() {
    int i, l, r, k;
    while (scanf("%d%d", &n, &m) != EOF) {
        for (i = 1; i <= n; i++)
            scanf("%d", &a[i]);
        build(1, n + 1, 1);
        sort(a + 1, a + 1 + n);
        for (i = 1; i <= m; i++) {
            scanf("%d%d%d", &l, &r, &k);
            printf("%d\n", QR(l, r + 1, k));
        }
    }

    return 0;
}
\end{lstlisting}
    \subsubsection{思考}
        如何取区间内不重复的第k大?\\
        结点中用两个vector,其中一个是原来的不变,另一个用于存放不重复的数,查询的时候在后者中查。

\subsection{区间内大于等于v的最小值}
    \subsubsection{问题简述}
        给定n个数,m次查询,每次查询[l,r]内比v大的最小值,输出这个最小值。
    \subsubsection{解决方案}
        归并树天生适合的问题。对于对应区间,直接返回lower\_bound找到的第一个数注意,如果没找到返回一个无限大,
        对于其余祖先节点,返回左右子树中找到的最小值。
\begin{lstlisting}
int query(int l, int r, int v, int x) {
    Node& cur = t[x];
    if (cur.l == l && cur.r == r) {
        vector<int>::iterator it;
        it = lower_bound(cur.a.begin(), cur.a.end(), v);
        if (it == cur.a.end())
            return n + 1;
        return *it;
    }
    if (r <= cur.m)
        return query(l, r, v, x * 2);
    else if (l >= cur.m)
        return query(l, r, v, x * 2 + 1);
    else
        return min(query(l, cur.m, v, x * 2), query(cur.m, r, v, x * 2 + 1));
}
\end{lstlisting}
    \include{tex/数据结构/划分树}
    \include{tex/数据结构/左偏树}
    \include{tex/数据结构/线段树练习}
    \section{树状数组套权值线段树}
树状数组套权值线段树通常用来解决一种二维查询,第一维是区间,第二维是值。\\
最典型的例子就是带修改的区间k小值查询,思路是,把二分答案的操作和查询小于一个值的数的数量两种操作结合起来。\\
在修改操作进行时,先在线段树上从上往下跳到被修改的点,删除所经过的点所指向的动态开点权值线段树上的原来的值,然后插入新的值。\\
在查询答案时,先取出该区间覆盖在线段树上的所有点,然后用类似于静态区间k小值的方法,将这些点一起向左儿子或向右儿子跳。如果所有这些点左儿子存储的值大于等于k,则往左跳,否则往右跳。
\begin{lstlisting}
/*
例题:洛谷 P2617 Dynamic Rankings
线段树套动态开点权值线段树
*/

#include <bits/stdc++.h>

using namespace std;
const int N = 2e5 + 10;  // FIXME:

struct Qr {
    char op[3];   // 哪种操作
    int l, r, k;  // 查询
    int p, v;     // 修改
} qr[N];

int n, m, a[N];    // 元素个数、操作个数、原数组
int q1[N], q2[N];  // 待查区间左右端点的前缀和,数组元素表示组成前缀和的权值线段树的根结点编号
int cnt1, cnt2;    // 两个前缀和被划分成的节点个数

// 离散化
int i2x[N], len;
inline int x2i(int x) {
    return lower_bound(i2x + 1, i2x + 1 + len, x) - i2x;
}

struct SegT {
    int cnt;                       // 已经使用的结点个数
    int rt[N * 2];                 // rt[i]:树状数组中下标为i的节点对应的权值线段树的根结点编号
    int lc[N * 300], rc[N * 300];  // 左右儿子编号
    int s[N * 300];                // 结点中值的个数

    // o是当前结点的编号,[l,r]表示当前结点所表示的区间,v是要添加或删除的权值,d表示添加或删除
    void update(int& o, int l, int r, int v, int d) {
        if (!o)
            o = ++cnt;
        s[o] += d;
        if (l == r)
            return;
        int m = (l + r) / 2;
        if (v <= m)
            update(lc[o], l, m, v, d);
        else
            update(rc[o], m + 1, r, v, d);
    }
} st;

inline int lowbit(int x) {
    return -x & x;
}

// p是要修改的位置,v是要修改的权值,d=-1或1表示删除或添加
void upd(int p, int v, int d) {
    for (; p <= n; p += lowbit(p))
        st.update(st.rt[p], 1, len, v, d);
}

// 获取构成区间[1,p]的所有权值线段树的根结点,放到a[]中,共cnt个
void gtv(int p, int a[], int& cnt) {
    cnt = 0;
    for (; p; p -= lowbit(p))
        a[++cnt] = st.rt[p];
}

// [l,r]表示当前两组结点表示的值域,查询第k小值,返回的是离散化之后的值
int qry(int l, int r, int k) {
    if (l == r)
        return l;

    int m = (l + r) / 2, ltot = 0, i;

    // 统计左子树中权值个数,前缀和之差
    for (i = 1; i <= cnt2; i++)
        ltot += st.s[st.lc[q2[i]]];
    for (i = 1; i <= cnt1; i++)
        ltot -= st.s[st.lc[q1[i]]];

    if (ltot >= k) {  // 向左搜
        for (i = 1; i <= cnt2; i++)
            q2[i] = st.lc[q2[i]];
        for (i = 1; i <= cnt1; i++)
            q1[i] = st.lc[q1[i]];
        return qry(l, m, k);
    } else {  // 向右搜
        for (i = 1; i <= cnt2; i++)
            q2[i] = st.rc[q2[i]];
        for (i = 1; i <= cnt1; i++)
            q1[i] = st.rc[q1[i]];
        return qry(m + 1, r, k - ltot);
    }
}

int main() {
    int i;

    scanf("%d%d", &n, &m);
    for (i = 1; i <= n; i++)
        scanf("%d", &a[i]), i2x[++len] = a[i];

    for (i = 1; i <= m; i++) {
        scanf("%s", qr[i].op);
        if (qr[i].op[0] == 'Q')
            scanf("%d%d%d", &qr[i].l, &qr[i].r, &qr[i].k);
        else
            scanf("%d%d", &qr[i].p, &qr[i].v), i2x[++len] = qr[i].v;
    }

    // 离散化
    sort(i2x + 1, i2x + 1 + len);
    len = unique(i2x + 1, i2x + 1 + len) - i2x - 1;

    // 初始化树状数组
    for (i = 1; i <= n; i++)
        upd(i, x2i(a[i]), 1);

    // 操作
    for (i = 1; i <= m; i++) {
        if (qr[i].op[0] == 'C') {               // 修改
            upd(qr[i].p, x2i(a[qr[i].p]), -1);  // 删去旧值
            upd(qr[i].p, x2i(qr[i].v), 1);      // 添加新值
            a[qr[i].p] = qr[i].v;
        } else {  // 查询
            gtv(qr[i].l - 1, q1, cnt1);
            gtv(qr[i].r, q2, cnt2);
            printf("%d\n", i2x[qry(1, len, qr[i].k)]);
        }
    }

    return 0;
}
\end{lstlisting}
    %数据结构

    \chapter{图论}
    \include{tex/图论/ShortPath}
    %最短路
    \include{tex/图论/MST}
    %最小生成树
    \section{树的直径}
\subsection{树形DP求树的直径}
仅能求出直径长度,无法得知路径信息,可处理负权边。
\begin{lstlisting}
int dp[N];
//dp[rt] 以rt为根的子树 从rt出发最远可达距离
/*
    对于每个结点x f[x]:经过节点x的最长链长度
*/
void DP(int rt)
{
    dp[rt]=0;//单点
    vis[rt]=1;
    for(int i=head[rt];i;i=nxt[i])
    {
        int s=ver[i];
        if(!vis[s])
        {
            DP(s);
            diameter=max(diameter,dp[rt]+dp[s]+edge[i]);
            dp[rt]=max(dp[rt],dp[s]+edge[i]);
        }
    }
}
\end{lstlisting}
\subsection{两次BFS/DFS求树的直径}
无法处理负权边,容易记录路径
\begin{lstlisting}
void DFS(int start,bool record_path)
{
    vis[start]=1;
    for(int i=head[start];i;i=nxt[i])
    {
        int s=ver[i];
        if(!vis[s])
        {
            dis[s]=dis[start]+edge[i];
            if(record_path) path[s]=i;
            DFS(s,record_path);
        }
    }
    vis[start]=0;//清理
}
\end{lstlisting}
例题分析

P3629 [APIO2010]巡逻(两种求树直径方法的综合应用)

P1099 树网的核(枚举)

\section{最近公共祖先(LCA)}
\subsection{树上倍增}
\begin{lstlisting}
void BFS()
{
    queue<int> q;
    q.push(1);
    d[1] = 1;
    while (!q.empty())
    {
        int x = q.front();
        q.pop();
        for (int i = head[x]; i; i = nxt[i])
        {
            int y = ver[i];
            if (!d[y])
            {
                d[y] = d[x] + 1;
                fa[y][0] = x;
                for (int j = 1; j <= k; j++)
                {
                    fa[y][j] = fa[fa[y][j - 1]][j - 1];
                }
                q.push(y);
            }
        }
    }
}
int LCA(int x, int y)
{
    if (d[x] < d[y])
        swap(x, y);
    for (int i = k; i >= 0; i--)
        if (d[fa[x][i]] >= d[y])
            x = fa[x][i];
    if (x == y)
        return y;
    for (int i = k; i >= 0; i--)
        if (fa[x][i] != fa[y][i])
            x = fa[x][i], y = fa[y][i];
    return fa[x][0];
}
\end{lstlisting}
\subsection{Tarjan}
\begin{lstlisting}
int Find(int x)
{
    if (x == fa[x])
        return x;
    return fa[x] = Find(fa[x]);
}
void Tarjan(int x)
{
    vis[x] = 1;
    for (int i = head[x]; i; i = nxt[i])
    {
        int y = ver[i];
        if (!vis[y])
        {
            Tarjan(y);
            fa[y] = x;
        }
    }
    for (int i = 0; i < q[x].size(); i++)
    {
        int y = q[x][i].first, id = q[x][i].second;
        if (vis[y] == 2)
            lca[id] = Find(y);
    }
    vis[x] = 2;
}
\end{lstlisting}

\section{树上差分}
例题分析

POJ3417 Network(LCA,树上差分,边覆盖)

6302 雨天的尾巴(LCA,树上差分,点覆盖,权值线段树,线段树合并)

P1600 天天爱跑步(LCA,树上差分)

\section{LCA的综合应用}
例题分析

CH56C 异象石(dfn时间戳,LCA)

P4180 【模板】严格次小生成树[BJWC2010](树上倍增)
\begin{lstlisting}
#include <algorithm>
#include <iostream>
#include <math.h>
#include <queue>
#include <stdio.h>
using namespace std;
const int N = 1e5 + 50;
const int M = 6e5 + 50;
#define ll long long
#define pii pair<int, int>
#define inf 0x3f3f3f3f
int n, m, k, F[N][20], d[N], fa[N];
int head[N], ver[M], nxt[M], edge[M], tot;
ll G[N][20][2];
void add(int x, int y, int z)
{
    ver[++tot] = y, nxt[tot] = head[x], head[x] = tot, edge[tot] = z;
}
struct edge
{
    int x, y, z;
    bool used;
    bool operator<(const edge &e) const
    {
        return z < e.z;
    }
} e[M];
int Find(int x)
{
    if (fa[x] == x)
        return x;
    return fa[x] = Find(fa[x]);
}
ll Kruskal()
{
    for (int i = 1; i <= n; i++)
        fa[i] = i;
    sort(e + 1, e + 1 + m);
    ll ans = 0;
    int cnt = 0;
    for (int i = 1; i <= m; i++)
    {
        int fx = Find(e[i].x), fy = Find(e[i].y);
        if (fx != fy)
        {
            fa[fx] = fy;
            ans += e[i].z;
            e[i].used = true;
            cnt++;
            if (cnt >= n - 1)
                break;
        }
    }
    return ans;
}
void BFS()
{
    k = log2(n) + 1;
    queue<int> q;
    q.push(1), d[1] = 1;
    for (int i = 0; i <= k; i++)
        G[1][i][0] = G[1][i][1] = -inf;
    while (q.size())
    {
        int x = q.front();
        q.pop();
        for (int i = head[x]; i; i = nxt[i])
        {
            int y = ver[i];
            if (!d[y])
            {
                d[y] = d[x] + 1;
                F[y][0] = x;
                G[y][0][0] = edge[i];
                G[y][0][1] = -inf;
                for (int j = 1; j <= k; j++)
                {
                    F[y][j] = F[F[y][j - 1]][j - 1];
                    G[y][j][0] = max(G[y][j - 1][0], G[F[y][j - 1]][j - 1][0]);
                    if (G[y][j - 1][0] == G[F[y][j - 1]][j - 1][0])
                        G[y][j][1] = max(G[y][j - 1][1], G[F[y][j - 1]][j - 1][1]);
                    else if (G[y][j - 1][0] > G[F[y][j - 1]][j - 1][0])
                        G[y][j][1] = max(G[F[y][j - 1]][j - 1][0], G[y][j - 1][1]);
                    else
                        G[y][j][1] = max(G[y][j - 1][0], G[F[y][j - 1]][j - 1][1]);
                }
                q.push(y);
            }
        }
    }
}
pii LCA(int x, int y)
{
    ll val1 = -inf, val2 = -inf;
    if (d[y] > d[x])
        swap(x, y);
    for (int i = k; i >= 0; i--)
    {
        if (d[F[x][i]] >= d[y])
        {
            if (G[x][i][0] > val1)
                val1 = G[x][i][0], val2 = max(val2, G[x][i][1]);
            else if (G[x][i][0] < val1)
                val2 = max(val2, G[x][i][0]);
            x = F[x][i];
        }
    }
    if (x == y)
        return make_pair(val1, val2);
    for (int i = k; i >= 0; i--)
    {
        if (F[x][i] != F[y][i])
        {
            val1 = max(val1, max(G[x][i][0], G[y][i][0]));
            val2 = max(val2, (val1 == G[x][i][0]) ? G[x][i][1] : G[x][i][0]);
            val2 = max(val2, (val1 == G[y][i][0]) ? G[y][i][1] : G[y][i][0]);
            x = F[x][i], y = F[y][i];
        }
    }
    val1 = max(val1, max(G[x][0][0], G[y][0][0]));
    val2 = max(val2, (val1 == G[x][0][0]) ? G[x][0][1] : G[x][0][0]);
    val2 = max(val2, (val1 == G[y][0][0]) ? G[y][0][1] : G[y][0][0]);
    return make_pair(val1, val2);
}
int main()
{
    scanf("%d%d", &n, &m);
    for (int i = 1; i <= m; i++)
        scanf("%d%d%d", &e[i].x, &e[i].y, &e[i].z), e[i].used = false;
    ll sum = Kruskal(), ans = 0x3f3f3f3f3f3f3f3f;
    for (int i = 1; i <= m; i++)
        if (e[i].used)
            add(e[i].x, e[i].y, e[i].z), add(e[i].y, e[i].x, e[i].z);
    BFS();
    for (int i = 1; i <= m; i++)
    {
        if (!e[i].used)
        {
            pii temp = LCA(e[i].x, e[i].y);
            if (e[i].z > temp.first)
                ans = min(ans, sum - temp.first + e[i].z);
            else if (e[i].z == temp.first)
                ans = min(ans, sum - temp.second + e[i].z);
        }
    }
    cout << ans << endl;
    //system("pause");
    return 0;
}
\end{lstlisting}

P1084 疫情控制(二分,树上倍增,贪心)
    %树的直径与最近公共祖先
    \section{基环树}
    %TODO 基环树 
    \include{tex/图论/SystemOfDifferenceConstraints}
    %负环与差分约束
    \include{tex/图论/TarjanDCC}
    %Tarjan算法与无向图连通性
    \section{Tarjan算法与有向图连通性}
\subsection{强连通分量(SCC)判定法则}
\begin{lstlisting}
void Tarjan(int x)
{
    dfn[x]=low[x]=++num;
    stack[++top]=x,in_stack[x]=true;
    for(int i=head[x];i;i=nxt[i])
    {
        int y=ver[i];
        if(!dfn[y])
        {
            Tarjan(y);
            low[x]=min(low[x],low[y]);
        }
        else if(in_stack[y])
            low[x]=min(low[x],dfn[y]);
    }
    if(dfn[x]==low[x])
    {
        cnt++;
        int y;
        do
        {
            y=stack[top--],in_stack[y]=false;
            color[y]=cnt, scc[cnt].push_back(y);
        } while (x!=y);
    }
}
\end{lstlisting}
\subsection{SCC -> DAG}
\begin{lstlisting}
void SCC()
{
    for (int i = 0; i <= n; i++)
        if (!dfn[i])
            Tarjan(i);
    //缩点
    for (int x = 1; x <= n; x++)
    {
        for (int i = head[x]; i; i = nxt[i])
        {
            int y = ver1[i];
            if (color[x] != color[y])
                add_c(color[x], color[y]);
        }
    }
}
\end{lstlisting}
例题分析

POJ1236 Network of Schools(SCC->DAG,入度出度)

P3275 [SCOI2011]糖果(SPFA TLE,SCC->DAG,Topo,DP)

\subsection{有向图的必经点与必经边}
对于有向无环图(DAG):

    在原图中按照拓扑序进行动态规划,求出起点S到图中每个点 x 的路径条数 fs[x]。

    在反图上再次按照拓扑序进行动态规划,求出每个点 x 到终点T的路径条数 ft[x]。

    显然,fs[T] 表示从S到T的路径总条数。根据乘法原理:

    1:对于一条有向边 (x,y),若 fs[x]*ft[y]=fs[T],则 (x,y) 是有向无环图从S到T的必经边。
    
    2:对于一个点 x,若 fs[x]*ft[x]=fs[T],则 x 是有向无环图从S到T的必经点。

    路径条数规模较大,可对大质数取模后保存,但有概率误判。


例题分析

6703 PKU ACM Team's Excursion (DAG必经边,枚举,DP)

\subsection{2-SAT问题}

原命题与逆否命题互为等价命题,在建立有向边关系时要注意对称性。

例题分析

POJ3678 Katu Puzzle(2-SAT,Tarjan,,SCC)

POJ3683 Priest John's Busiest Day(2-SAT方案构造)

\begin{lstlisting}
for (int i = 1; i <= 2 * n; i++)
{
    val[i] = color[i] > color[opp[i]];
    //若c[i] 大于 c[opp[i]]   opp[i]赋0 
}
\end{lstlisting}
    
    %Tarjan算法与有向图连通性
    \section{二分图的匹配}
\subsection{二分图判定}
    一张无向图是二分图,当且仅当图中不存在奇环(长度为奇数的环)。
\begin{lstlisting}
//染色法判定奇环
bool DFS(int x,int color)
{
    vis[x]=color;
    for(int i=head[x];i;i=nxt[i])
    {
        int y=ver[i];
        if(!vis[y])
        {
            if(!DFS(y,3-color)) return false;
        }
        else if(vis[y]==color) return false;
    }
    return true;
}
\end{lstlisting}

例题分析

P1525 关押罪犯(判定二分图,二分)

\subsection{二分图最大匹配}
二分图匹配的模型要素

1:节点能分成独立的两个集合,每个集合内部有 0 条边。 “0要素”

2:每个节点只能与 1 条匹配边相连。 “1要素”
\begin{lstlisting}
\\匈牙利算法
\\在主函数中对每个左部节点调用寻找增广路时,需要对 vis 重置。
bool DFS(int x)
{
    for (int i = head[x]; i; i = nxt[i])
    {
        int y = ver[i];
        if (!vis[y])
        {
            vis[y] = 1;
            if (!match[y] || DFS(match[y]))
            {
                match[y] = x;
                return true;
            }
        }
    }
    return false;
}
\end{lstlisting}

例题分析

6801 棋盘覆盖(奇偶染色)

6802 車的放置(行列)

6803 导弹防御塔(二分,拆点多重匹配)

\subsection{二分图带权匹配}
二分图带权最大匹配的前提是匹配数最大,然后再最大化匹配边的权值总和。
\begin{lstlisting}
/*KM 稠密图上效率高于费用流,但是有较大局限性,只能在满足“带权最大匹配一定是完备匹配”的图中正确求解。
w[][]:边权
la[],lb[]: 左,右部点顶标
visa[],visb[]: 访问标记,是否在交错树中
ans: Σw[match[i]][i]
*/
bool DFS(int x)
{
    visa[x] = true;
    for (int y = 1; y <= n; y++)
    {
        if (!visb[y])
        {
            double temp = fabs(la[x] + lb[y] - w[x][y]);//对于浮点数,相等子图的判定
            if (temp < eps)
            {
                visb[y] = true;
                if (!match[y] || DFS(match[y]))
                {
                    match[y] = x;
                    return true;
                }
            }
            else
                upd[y] = min(upd[y], la[x] + lb[y] - w[x][y]);
        }
    }
    return false;
}
void KM()
{
    for (int i = 1; i <= n; i++)
    {
        la[i] = -inf;
        lb[i] = 0;
        for (int j = 1; j <= n; j++)
            la[i] = max(la[i], w[i][j]);
    }
    for (int i = 1; i <= n; i++)
    {
        while (true)
        {
            memset(visa, 0, sizeof(visa));
            memset(visb, 0, sizeof(visb));
            for (int j = 1; j <= n; j++)
                upd[j] = inf;
            if (DFS(i))
                break;
            else
            {
                delta = inf;
                for (int j = 1; j <= n; j++)
                    if (!visb[j])
                        delta = min(delta, upd[j]);
                for (int j = 1; j <= n; j++)
                {
                    if (visa[j])
                        la[j] -= delta;
                    if (visb[j])
                        lb[j] += delta;
                }
            }
        }
    }
}
\end{lstlisting}

例题分析

POJ3565 Ants(三角形不等式,二分图带权最小匹配)
    %二分图的匹配
    \section{二分图的覆盖与独立集}
\subsection{二分图最小点覆盖}
二分图最小覆盖模型特点:

每条边有 2 个端点,二者至少选择一个。  “2要素”
\subsubsection{König's theorem}
二分图最小点覆盖包含的点数等于二分图最大匹配包含的边数。

例题分析

POJ1325 Machine Schedule(二分图最小覆盖)

POJ2226 Muddy Fields(行列连续块,二分图最小覆盖)

\subsection{二分图最大独立集}

无向图 G 的最大团等于其补图 G' 的最大独立集。(补图转化)

设G是有 n 个节点的二分图,G 的最大独立集的大小等于 n 减去最大匹配数。

例题分析

6901 骑士放置 (奇偶染色)

\subsection{有向无环图的最小路径点覆盖}
给定一张有向无环图,要求用尽量少的不相交的简单路径,覆盖有向无环图的所有顶点(也就是每个顶点恰好被覆盖一次)。
这个问题被称为有向无环图的最小路径点覆盖,简称“最小路径覆盖”。

有向无环图 G 的最小路径点覆盖包含的路径条数,等于n(有向无环图的点数)减去拆点二分图 G2 的最大匹配数。

若简单路径可相交,即一个节点可被覆盖多次,这个问题称为有向无环图的最小路径可重复点覆盖。

对于这个问题,可先对G求传递闭包,得到有向无环图 G',再在 G' 上求一般的(路径不可相交的)最小路径点覆盖。

例题分析

6902 Vani和Cl2捉迷藏(最小路径可重复点覆盖,构造方案)
\begin{lstlisting}
// 构造方案,先把所有路径终点(左部非匹配点)作为藏身点
for (int i = 1; i <= n; i++) succ[match[i]] = true;
for (int i = 1, k = 0; i <= n; i++)
    if (!succ[i]) hide[++k] = i;
memset(vis, 0, sizeof(vis));
bool modify = true;
while (modify) {
    modify = false;
    // 求出 next(hide)
    for (int i = 1; i <= ans; i++) 
        for (int j = 1; j <= n; j++)
            if (cl[hide[i]][j]) vis[j] = true;
    for (int i = 1; i <= ans; i++)
        if (vis[hide[i]]) {
            modify = true;
            // 不断向上移动
            while (vis[hide[i]]) hide[i] = match[hide[i]];
        }
}
for (int i = 1; i <= ans; i++) printf("%d ", hide[i]);
cout << endl;
\end{lstlisting}
    %二分图的覆盖与独立集
    \section{网络流初步}
\subsection{最大流}
\subsubsection{Edmonds Karp增广路}
\begin{lstlisting}
bool BFS() {
    memset(vis, 0, sizeof(vis));
    queue<int> q;
    q.push(S); vis[S] = 1;
    incf[S] = inf; // 增广路上各边的最小剩余容量
    while (q.size()) {
        int x = q.front(); q.pop();
        for (int i = head[x]; i; i = Next[i])
            if (edge[i]) {
                int y = ver[i];
                if (vis[y]) continue;
                incf[y] = min(incf[x], edge[i]);
                pre[y] = i; // 记录前驱,便于找到最长路的实际方案
                q.push(y), vis[y] = 1;
                if (y == t) return 1;
            }
    }
    return 0;
}
void Update() { // 更新增广路及其反向边的剩余容量
    int x = t;
    while (x != s) {
        int i = pre[x];
        edge[i] -= incf[t];
        edge[i ^ 1] += incf[t]; // 利用“成对存储”的xor 1技巧
        x = ver[i ^ 1];
    }
    maxflow += incf[t];
}
\end{lstlisting}

\subsubsection{Dinic}
\begin{lstlisting}
//可加入当前弧优化(&):在增广时复制head[]到cur[],在增广时同步修改cur[],目的是递归时跳过已增广的边。
bool BFS()
{
    memset(d, 0, sizeof(d));
    queue<int> q;
    q.push(S);
    d[S] = 1; //不为1  陷入死循环
    while (q.size())
    {
        int x = q.front();
        q.pop();
        for (int i = head[x]; i; i = nxt[i])
        {
            int y = ver[i];
            if (edge[i] && !d[y])
            {
                d[y] = d[x] + 1;
                q.push(y);
                if (y == T)
                    return true;
            }
        }
    }
    return false;
}
int Dinic(int x, int flow)
{
    if (x == T)
        return flow;
    int rest = flow, k;
    for (int i = head[x]; i && rest; i = nxt[i])
    {
        int y = ver[i];
        if (edge[i] && d[y] == d[x] + 1)
        {
            k = Dinic(y, min(edge[i], rest));
            if (!k)
                d[y] = 0;
            edge[i] -= k;
            edge[i ^ 1] += k;
            rest -= k;
        }
    }
    return flow - rest;
}
\end{lstlisting}

\subsubsection{二分图最大匹配的必须边与可行边}
在一般的二分图中,可以用最大流计算任一组最大匹配。

此时:
    必须边的判定条件为:(x,y) 流量为 1 ,并且在残量网络上属于不同的SCC。

    可行边的判定条件为:(x,y) 流量为 1 ,或者在残量网络上属于同一个SCC。

例题分析

CH17C 舞动的夜晚(Dinic,Tarjan,二分图可行边)

\subsection{最小割}
\subsubsection{最大流最小割定理}
任何一个网络的最大流量等于最小割中边的容量之和。

例题分析

POJ1966 Cable TV Network (枚举,点边转化)

\subsection{费用流}
\subsubsection{Edmonds Karp 增广路}
BFS寻找增广路 -> SPFA寻找单位费用之和最小的增广路(将费用作为边权,在残量网络上求最短路)。

注意:反向边的费用为相反数。
\begin{lstlisting}
bool SPFA()
{
    memset(dis, 0xcf, sizeof(dis));//-inf
    memset(vis, 0, sizeof(vis));
    queue<int> q;
    dis[S] = 0, vis[S] = 1, incf[S] = 1 << 30;
    q.push(S);
    while (q.size())
    {
        int x = q.front();
        q.pop();
        vis[x] = 0;
        for (int i = head[x]; i; i = nxt[i])
        {
            if (edge[i])
            {
                int y = ver[i];
                if (dis[y] < dis[x] + cost[i])
                {
                    dis[y] = dis[x] + cost[i];
                    incf[y] = min(incf[x], edge[i]);
                    pre[y] = i;
                    if (!vis[y])
                        q.push(y), vis[y] = 1;
                }
            }
        }
    }
    if (dis[T] == 0xcfcfcfcf)
        return false;
    return true;
}
int max_flow, ans;
void Update()
{
    int x = T;
    while (x != S)
    {
        int i = pre[x];
        edge[i] -= incf[T];
        edge[i ^ 1] += incf[T];
        x = ver[i ^ 1];
    }
    max_flow += incf[T];
    ans += incf[T] * dis[T];
}
\end{lstlisting}
例题分析

POJ3422 Kaka's Matrix Travels(点边转化,费用流)
    %网络流初步
\end{document}