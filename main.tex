\documentclass[10pt,a4paper,oneside]{book}
\usepackage{listings}
\usepackage{fontspec}
\usepackage{xeCJK}
\usepackage{xcolor}
\usepackage{graphicx}
\usepackage{fancyhdr}
\usepackage{bm}
\usepackage{geometry}
\geometry{left=2.0cm,right=2.0cm,top=2.0cm,bottom=2.0cm}
\setmonofont{Consolas}
\setsansfont{Consolas}
\pagestyle{fancy}
\setcounter{tocdepth}{4}
\setcounter{secnumdepth}{3}
\lstset{
    language    = c++,
    breaklines  = true,
    captionpos  = b,
    tabsize     = 4,
    numbers     = left,
    columns     = fullflexible,
    keepspaces  = true,
    commentstyle = \color[RGB]{0,128,0},
    keywordstyle = \color[RGB]{0,0,255},
    basicstyle   = \small\ttfamily,
    rulesepcolor = \color{red!20!green!20!blue!20},
    showstringspaces = false,
}
\title{
    \begin{center}
        \includegraphics[width=4in]{logo.png}
        \\ 
        \textbf{ICPC Template Manual}
    \end{center}
}
\author{
    \includegraphics[width=3cm]{author.png}
    \\
    \textbf{作者:贺梦杰}
}
\date{\today}

\begin{document}
    \maketitle
    \tableofcontents

    \chapter{基础}
    % \include{tex/Basic/Basic}
    %基础部分

    \chapter{搜索}
    %搜索部分

    \chapter{动态规划}
    % \include{tex/DynamicProgramming/BagDP}
    % \include{tex/DynamicProgramming/PosDP}
    % \section{树形动态规划}

\subsection{加分二叉树}
    \subsubsection{问题描述}
        设一个n个节点的二叉树tree的中序遍历为(l,2,3,…,n),其中数字1,2,3,…,n为节点编号。每个节点都有一个分数(均为正整数),记第i个节点的分数为di,tree及它的每个子树都有一个加分,任一棵子树subtree(也包含tree本身)的加分计算方法如下:\\
        subtree的左子树的加分× subtree的右子树的加分+subtree的根的分数\\
        若某个子树为空,规定其加分为1,叶子的加分就是叶节点本身的分数。不考虑它的空子树。\\
        试求一棵符合中序遍历为(1,2,3,…,n)且加分最高的二叉树tree。要求输出;\\
        (1)tree的最高加分\\
        (2)tree的前序遍历
    \subsubsection{输入格式}
        第1行:一个整数n(n<30),为节点个数。\\
        第2行:n个用空格隔开的整数,为每个节点的分数(分数<100)。
    \subsubsection{输出格式}
        第1行:一个整数,为最高加分(结果不会超过4,000,000,000)。\\
        第2行:n个用空格隔开的整数,为该树的前序遍历。
    \subsubsection{思路}
        这道题看上去是树形DP,但仔细思考,我们发现很难依据中序遍历建出树。
        但中序遍历有其独特之处,即一旦根被确定,则左右子树也被确定。\\
        所以我们应该用区间DP来解决这道题。\\
        $f(i,j)$:选i到j的节点作为一颗子树最大的得分
        $$f(i,j)=\max_{i \leq k \leq j}{\{f(i,k-1)*f(k+1,j)+f(k,k)\}}$$
        由题意,当$i>j$,$f(i,j)=1$为空节点。\\
        前序遍历就是重新用dfs走一遍:每次找到使$f(i,j)$最大的$k$,然后以$k$为分界向左右两边递归。

\subsection{洛谷P2015 二叉苹果树}
    \subsubsection{问题描述}
        有一棵苹果树,如果树枝有分叉,一定是分2叉(就是说没有只有1个儿子的结点)这棵树共有N个结点(叶子点或者树枝分叉点),编号为1-N,树根编号一定是1。我们用一根树枝两端连接的结点的编号来描述一根树枝的位置。现在这颗树枝条太多了,需要剪枝。但是一些树枝上长有苹果。\\
        给定需要保留的树枝数量,求出最多能留住多少苹果。
    \subsubsection{输入格式}
        第1行2个数,N和Q(1<=Q<= N,1<N<=100)。\\
        N表示树的结点数,Q表示要保留的树枝数量。接下来N-1行描述树枝的信息。\\
        每行3个整数,前两个是它连接的结点的编号。第3个数是这根树枝上苹果的数量。\\
        每根树枝上的苹果不超过30000个。
    \subsubsection{输出格式}
        剩余苹果的最大数量。
    \subsubsection{思路}
        $f(i,j)$表示以i为节点的根保留k条边的最大值
        接下来对每一个节点分类讨论,共三种情况:\\
        1.全选左子树\\
        2.全选右子树\\
        3.左右子树都有一部分\\
        为了方便,我们设左儿子为ls,右儿子为rs,连接左儿子的边为le,连接右儿子的边为re\\
        $$f(i,j)=\max{\{f(ls,j-1)+le, f(rs,j-1)+re, \max_{0 \leq k \leq j-2}{\{f(ls,k)+f(rs,j-2-k)\}}+le+re\}}$$

\subsection{最大利润}
    \subsubsection{问题描述}
        政府邀请了你在火车站开饭店,但不允许同时在两个相连接的火车站开。任意两个火车站有且只有一条路径,每个火车站最多有50个和它相连接的火车站。\\
        告诉你每个火车站的利润,问你可以获得的最大利润为多少。\\
        最佳投资方案是在1,2,5,6这4个火车站开饭店可以获得利润为90
    \subsubsection{输入格式}
        第一行输入整数N(<=100000),表示有N个火车站,分别用1,2。。。,N来编号。接下来N行,每行一个整数表示每个站点的利润,接下来N-1行描述火车站网络,每行两个整数,表示相连接的两个站点。
    \subsubsection{输出格式}
        输出一个整数表示可以获得的最大利润。
    \subsubsection{思路}
        这道题虽然是多叉树,但状态仍然是比较简单的。\\
        对于某个结点,如果选择该节点,则该节点的所有儿子都不能选,如果不选该节点,则它的儿子可选可不选。\\
        所以,我们令$f(i)$表示以i节点为根的子树中选i的最大利润,$h(i)$表示以i节点为根的子树中不选i的最大利润。a[i]是i本身的利润,j是i的儿子
        $$f(i)=a[i]+\sum_{j}{h(j)}$$
        $$h(i)=\sum_{j}{\max{\{f(j),h(j)\}}}$$

\subsection{洛谷P2014 选课}
    \subsubsection{问题描述}
        学校实行学分制。每门的必修课都有固定的学分,同时还必须获得相应的选修课程学分。学校开设了N(N<300)门的选修课程,每个学生可选课程的数量M是给定的。学生选修了这M门课并考核通过就能获得相应的学分。\\
        在选修课程中,有些课程可以直接选修,有些课程需要一定的基础知识,必须在选了其它的一些课程的基础上才能选修。例如《Frontpage》必须在选修了《Windows操作基础》之后才能选修。我们称《Windows操作基础》是《Frontpage》的先修课。每门课的直接先修课最多只有一门。两门课也可能存在相同的先修课。每门课都有一个课号,依次为1,2,3,…。\\
        你的任务是为自己确定一个选课方案,使得你能得到的学分最多,并且必须满足先修课优先的原则。假定课程之间不存在时间上的冲突。
    \subsubsection{输入格式}
        输入文件的第一行包括两个整数N、M(中间用一个空格隔开),其中$1\leq N\leq 300,1\leq M\leq N$。
        以下N行每行代表一门课。课号依次为1,2,…,N。每行有两个数(用一个空格隔开),第一个数为这门课先修课的课号(若不存在先修课则该项为0),第二个数为这门课的学分。学分是不超过10的正整数。
    \subsubsection{输出格式}
        只有一个数:实际所选课程的学分总数。
    \subsubsection{思路一}
        分析一下,似乎也不是很难,对于某个节点,只有选择它,才能选择它的儿子们。\\
        令$f(x,i)$表示在以x结点为根的子树中选择i个点的最大学分。$f(x,1)=s[x]$,s[x]是课程x本身的学分。\\
        假设结点x有k个儿子,标号为$y_1,y_2,..,y_k$,让他们分别选$i_1,i_2,..,i_k$门课程,那么状态转移方程似乎是..
        $$f(x,i)=s[x]+\max_{i_1+i_2+..+i_k=i-1}{\sum_{1 \leq j \leq k}{f(y_j,i_j)}}$$
        好像..写不出这样的循环啊,当然,如果用dfs强行写也不是不可以,但是时间复杂度必炸。\\\\
        这时候我们要用一种类似\textbf{前缀和}的思维\\
        即,我们每dfs完一棵子树,都进行一次完整的dp更新。假设当前根结点为x,我们刚dfs完它的一棵子树y,那么我们进行如下更新(对所有i):
        $$f(x,i)=\max_{1\leq j\leq i}{\{f(x,j)+f(y,i-j)\}}$$
        \textbf{此时$f(x,i)$表示在以x结点为根的子树中已经被dfs过的部分中选i个点的最大学分,而每次更新就像是把一棵新子树添加到答案中。}\\
        除此以外,还有一点要注意,就是我们要让\textbf{i递减更新},因为大的i要用到之前小的i,而小的i不要用到大的i
    \subsubsection{思路二}
        先将\textbf{森林转二叉树(左孩子右兄弟)}。如何转呢?设D[x]、c[x]和b[x]分别表示x结点的父亲、左孩子和右兄弟,对于每个节点,D[x]是已知的,所以\textbf{b[x]=c[D[x]],c[D[x]]=x}。\\
        再分析,对于某个节点,如果要选其左孩子,则必须选它,而选右孩子则没有限制。\\
        令$f(x,i)$表示在以x结点为根的子树中选择i个点的最大学分。$f(x,1)=s[x]$,s[x]是课程x本身的学分。\\
        还是沿用上面\textbf{前缀和思想},先由左孩子(ls)更新,再由右孩子(rs)\textbf{倒着}更新,两次的方程如下:
        $$f(x,i)=f(x,1)+f(ls,i-1)$$
        $$f(x,i)=\max_{0\leq j\leq i}{f(x,j)+f(rs,i-j)}$$

\subsection{HYSBZ 2427 软件安装}
    \subsubsection{题目描述}
        现在我们的手头有N个软件,对于一个软件i,它要占用Wi的磁盘空间,它的价值为Vi。我们希望从中选择一些软件安装到一台磁盘容量为M的计算机上,使得这些软件的价值尽可能大(即Vi的和最大)。\\
        但是现在有个问题:软件之间存在依赖关系,即软件i只有在安装了软件j(包括软件j的直接或间接依赖)的情况下才能正确工作(软件i依赖软件j)。幸运的是,一个软件最多依赖另外一个软件。如果一个软件不能正常工作,那么他能够发挥的作用为0。\\
        我们现在知道了软件之间的依赖关系:软件i依赖Di。现在请你设计出一种方案,安装价值尽量大的软件。一个软件只能被安装一次,如果一个软件没有依赖则Di=0,这是只要这个软件安装了,它就能正常工作。
    subsubsection{输入格式}
        第1行:N,M (0<=N<=100,0<=M<=500)
        第2行:W1,W2, … Wi, … ,Wn
        第3行:V1,V2, … Vi, … ,Vn
        第4行:D1,D2, … Di, … ,Dn
    \subsubsection{输出格式}
        一个整数,代表最大价值。
    \subsubsection{思路}
        \textbf{注意,此图可能有环}\\
        仔细读题,题中说一个软件只依赖最多一个软件,所以在一个联通分之内最多只有一个环,并且该环可以指向环以外的节点,而环以外的节点不会指向环。\\
        因此,我们可以\textbf{把环当做一个新节点}来处理,这个新节点所占空间和价值都为环内元素加和。\\
        怎么确定环及环内元素呢?我推荐\textbf{着色法(有向图为黑白灰,无向图为黑白)}。\\
        有向图中白色为未探索的点,灰色为正在探索的点,黑色为已经探索过的并且确定不成环的点。若在探索过程中遇到灰点,则说明成环。\\
        无向图中白色为未探索的点,黑色为已经探索过和正在探索的点。若在探索过程中遇到黑点,则说明成环。\\
        处理完成后,DP思路同\textit{"洛谷P2014 选课"}

\subsection{CF486D Valid Sets}
    \subsubsection{题目描述}
        给定一棵树,每个点都有一个权,现在要你选择一个连通块,问有多少种选择方法,使得连通块中的最大点权和最小点权的差值小于等于d。答案对1e9+7取模。
    \subsubsection{输入格式}
        The first line contains two space-separated integers d (0 ≤ d ≤ 2000) and n (1 ≤ n ≤ 2000).\\
        The second line contains n space-separated positive integers a1, a2, ..., an(1 ≤ ai ≤ 2000).\\
        Then the next n - 1 line each contain pair of integers u and v (1 ≤ u, v ≤ n) denoting that there is an edge between u and v. It is guaranteed that these edges form a tree.
    \subsubsection{输出格式}
        Print the number of valid sets modulo 1000000007.
    \subsubsection{思路}
        我起初的错误想法是,设$f(x,i)$和$g(x,i)$分别为以x为根的子树中最大值和最小值为i的方案数,企图通过对g和f的运算得出答案,但这是错误的,主要原因是f和g无法把范围限定住。\\
        受到前面以子树为对象思考的影响,形成了\textbf{思维惯性},这里我们并不是以子树为对象,在一遍dfs里算出所有答案。\\
        而是\textbf{枚举},枚举每一点x,将x作为连通块的最大值,以x为根进行dfs,求方案数。\\
        另外要注意:因为点权可能相同,所以为了避免重复计算,我们需要定序,若权值相同,让编号大的点访问编号小的,而编号小的不能访问大的。

\subsection{CF294E Shaass the Great}
    \subsubsection{题目描述}
        给一颗带边权的树,让你选一条边删除,然后在得到的两个子树中各选一个点,用原来被删除的边连起来,重新拼成一棵树。使得这棵树的所有点对的距离总和最小。\\
    \subsubsection{输入格式}
        The first line of the input contains an integer n denoting the number of cities in the empire, $(2 \leq n \leq 5000)$. The next n - 1 lines each contains three integers ai, bi and wi showing that two cities ai and bi are connected using a road of length wi, $(1 \leq a_i, b_i \leq n, a_i \neq b_i, 1 \leq w_i \leq 106)$.
    \subsubsection{输出格式}
        所有点对距离总和最优解。
    \subsubsection{思路}
        设我们要删除的边为e,以e为分界树被分割成了左右两个部分(我们称为左树和右树),最后我们连接的两个点为vl和vr\\
        则答案为:\\
        左树内点对距离总和 + 右树内点对距离总和\\
        + 左树中所有点到vl的距离总和 * 右树中点的个数 + 右树中所有点到vr的距离总和 * 左树中点的个数\\
        + e的权重 * 左树中点的个数 * 右树中点的个数\\\\
        再观察一下,其实在删除e的情况下 左右树内点对距离总和、左右树中的点的个数、e的权重 都是不变的,变化的只有左右树中所有点到vl和vr的距离总和\\
        由此我们知道了,在删除e的情况下,答案最优的条件即为\textbf{找到 vl和vr,使得左树中所有点到vl的距离总和最小,右树中所有点到vr的距离总和最小}\\
        于是我们知道了,要得到答案,就是要求一棵树的三个值:树内点的个数、树上所有点到某一点的距离总和的最小值、树内所有点对距离总和\\
        树内点的个数最简单,不多说了,而树内所有点对的距离和也可以由树上所有点到某一点的距离和算得(即树上所有点到\textbf{每}一点的和除以2)\\\\
        下面主要考虑如何在$O(n)$的时间内求出树上所有点到每一点的距离:\\\\
        \textbf{考虑dfs,结果发现一遍dfs无论如何都不可能得到,但是可以知道以某一点x为根的子树上所有点到根x的距离总和\\
        在第一遍dfs的基础上,我们再进行一次dfs,这次dfs我们不再是自底向上更新,而是自顶向下更新,更新父节点及其以上所有的点到x的距离总和\\
        可以理解为,第一次dfs我们得到了x以下的所有点到x的距离总和,第二次dfs我们得到了x以上的所有点到x的距离总和}\\\\
        对于第二次dfs,例如我们现在在a节点上,即将去往x结点,我们需要下传给x节点的距离有哪些呢?\\
        1.a结点上面传下来的距离\\
        2.a结点上除了x分支所有旁路上的距离(由第一次dfs可以算出)\\
        3.a与x的边权\\\\
        所以最后的算法是:\textbf{枚举每一条边},运用以上算法,求出最小值
    % \include{tex/DynamicProgramming/LIS}
    %动态规划

    \chapter{字符串}
    % TODO 字符串Hash
    % \section{字符串Hash}
\begin{lstlisting}
#define ull unsigned long long
#define P 131
ull f[N], p[N];
void Init()
{
    p[0] = 1, f[0] = 0;
    for (int i = 1; i <= n; i++)
    {
        p[i] = p[i - 1] * P;
        f[i] = f[i - 1] * P + str[i];
    }
}
ull Hash(int left, int right)
{
    return f[right] - f[left - 1] * p[right - left + 1];
}
\end{lstlisting}


\subsection{应用:后缀数组}

\begin{lstlisting}
#define ull unsigned long long
#define P 131
const int N = 3e5 + 50;
ull f[N], p[N];
char str[N];
int SA[N], n, Height[N];
void Init()
{
    p[0] = 1, f[0] = 0;
    for (int i = 1; i <= n; i++)
    {
        p[i] = p[i - 1] * P;
        f[i] = f[i - 1] * P + str[i];
    }
}
ull Hash(int left, int right)
{
    return f[right] - f[left - 1] * p[right - left + 1];
}
// k:[0,n) 表示后缀S(k,n-1)
// 最长公共前缀
int LCP(int a, int b)
{
    int left = 0;
    int right = N;
    int mid;
    while (left < right)
    {
        mid = (left + right + 1) >> 1;
        if (a + mid - 1 <= n && b + mid - 1 <= n && Hash(a, a + mid - 1) == Hash(b, b + mid - 1))
            left = mid;
        else
            right = mid - 1;
    }
    return left;
}
bool cmp(int a, int b)
{
    int len = LCP(a, b);
    return str[a + len] < str[b + len];
}
void calc_height()
{
    Height[1] = 0;
    for (int i = 2; i <= n; i++)
        Height[i] = LCP(SA[i], SA[i - 1]);
}
int main()
{
    scanf("%s", str + 1);
    n = strlen(str + 1);
    for (int i = 1; i <= n; i++)
        SA[i] = i;
    Init();
    sort(SA + 1, SA + n + 1, cmp);
    calc_height();
}
\end{lstlisting}

\subsection{应用:二维Hash}
给定一个M行N列的01矩阵(只包含数字0或1的矩阵),再执行Q次询问,每次询问给出一个A行B列的01矩阵,求该矩阵是否在原矩阵中出现过。

做法:选取两个不同的P值分别对行列进行Hash处理,应用二维前缀和求取矩阵Hash值。
\begin{lstlisting}
#define P 131
#define Q 13331
#define ull unsigned long long
void Init()
{
    char ch;
    for (int i = 1; i <= m; i++)
        for (int j = 1; j <= n; j++)
            cin >> ch, Hash[i][j] = Hash[i][j - 1] * P + ch;
    for (int i = 1; i <= m; i++)
        for (int j = 1; j <= n; j++)
            Hash[i][j] += Hash[i - 1][j] * Q;
}
ull temp = Hash[i][j] - Hash[i - a][j] * q[a] - Hash[i][j - b] * p[b] + Hash[i - a][j - b] * q[a] * p[b];
\end{lstlisting}

\subsection{应用:一类同构判定的问题}
参考:杨弋《Hash在信息学竞赛中的一类应用》
    % %TODO 后缀自动机初步+Manacher+回文树
    % \section{后缀自动机}
\begin{lstlisting}
#define ll long long
const int MAXLEN = 1e6 + 50;
struct SAM
{
    int len[MAXLEN << 1], link[MAXLEN << 1], next[MAXLEN << 1][26];
    ll sze[MAXLEN << 1]; ////每个结点所代表的字符串的出现次数
    int sz, last, rt;
    int NewNode(int x = 0)
    {
        len[sz] = x;
        link[sz] = -1;
        memset(next[sz], -1, sizeof(next[sz]));
        return sz++;
    }
    void Init()
    {
        //重置
        sz = last = 0, rt = NewNode();
    }
    void Extend(int c)
    {
        int cur = NewNode(len[last] + 1);
        sze[cur] = 1;
        int p = last;
        while (~p && next[p][c] == -1)
            next[p][c] = cur, p = link[p];
        if (p == -1)
            link[cur] = rt;
        else
        {
            int q = next[p][c];
            if (len[q] == len[p] + 1)
                link[cur] = q;
            else
            {
                int clone = NewNode(len[p] + 1);
                memcpy(next[clone], next[q], sizeof(next[q]));
                link[clone] = link[q], link[q] = link[cur] = clone;
                while (~p && next[p][c] == q)
                    next[p][c] = clone, p = link[p];
            }
        }
        last = cur;
    }
    int id[MAXLEN << 1], c[MAXLEN];
    void Topo()
    {
        //计数排序
        memset(c, 0, sizeof(c));
        for (int i = 0; i < sz; i++)
            c[len[i]]++;
        for (int i = 1; i < MAXLEN; i++)
            c[i] += c[i - 1];
        for (int i = 0; i < sz; i++)
            id[--c[len[i]]] = i;
        for (int i = sz - 1; ~i; i--)
        {
            int u = id[i];
            if (~link[u])
                sze[link[u]] += sze[u];
        }
    }
};
\end{lstlisting}
\section{Manacher}
\begin{lstlisting}
const int MAXLEN = 1e5 + 50;
char ori[MAXLEN], str[MAXLEN * 2];
int d1[MAXLEN * 2], n, m;    
void Manacher()
{
    for (int i = 0, l = 0, r = -1; i < m; i++)
    {
        int k = (i > r) ? 1 : min(d1[l + r - i], r - i);
        while (i - k >= 0 && i + k < m && str[i - k] == str[i + k])
            k++;
        d1[i] = k--;
        if (i + k > r)
            l = i - k, r = i + k;
    }
}
int main()
{
    scanf("%s", ori);
    n = strlen(ori);
    str[0] = '#';
    for (int i = 0; i < n; i++)
    {
        str[(i + 1) * 2] = '#';
        str[(i + 1) * 2 - 1] = ori[i];
    }
    m = n * 2 + 1;
    Manacher();
}
\end{lstlisting}

\section{回文树/回文自动机}
\begin{lstlisting}
const int MAXLEN=5e5+50;
struct Palindromic_Tree
{
    int nxt[MAXLEN][26],fail[MAXLEN],len[MAXLEN],s[MAXLEN];
    int cnt[MAXLEN];// 结点表示的本质不同的回文串的个数(调用Count()后) 
    int num[MAXLEN];// 结点表示的最长回文串的最右端点为回文串结尾的回文串个数 
    int last,sz,n;
    int NewNode(int x)
    {
        memset(nxt[sz],0,sizeof(nxt[sz]));
        cnt[sz]=num[sz]=0,len[sz]=x;
        return sz++;
    }
    void Init()
    {
        sz=0;
        NewNode(0),NewNode(-1);
        last=n=0,s[0]=-1,fail[0]=1;
    }
    int GetFail(int u)
    {
        while(s[n-len[u]-1]!=s[n]) u=fail[u];
        return u;
    }
    void Add(int c)
    {
        //c-='a'
        s[++n]=c;
        int u=GetFail(last);
        if(!nxt[u][c])
        {
            int np=NewNode(len[u]+2);
            fail[np]=nxt[GetFail(fail[u])][c];
            num[np]=num[fail[np]]+1;
            nxt[u][c]=np;
        }
        last=nxt[u][c];
        cnt[last]++;
    }
    void Count()
    {
        for(int i=sz-1;~i;i--)
            cnt[fail[i]]+=cnt[i];
    }
}
\end{lstlisting}
    % %TODO KMP
    % \section{KMP}
    \subsection{前缀函数}
        给定一个长度为n的字符串s(假定下标从1开始),其\textbf{前缀函数}被定义为一个长度为n的数组$\pi$,其中$\pi[i]$为
        既是子串$s[1\dots i]$的前缀同时也是该子串的后缀的最长真前缀(proper prefix)长度。一个字符串的真前缀是其前缀但
        不等于该字符串自身。根据定义,$\pi[1]=0$。\\\\
        前缀函数的定义可用数学语言描述如下:
        $$\pi[i]=\max _{k=0 \ldots i-1}\{k : s[1 \ldots k]=s[i-k+1 \ldots i]\}$$
        举例来说,字符串abcabcd的前缀函数为$[0,0,0,1,2,3,0]$,字符串aabaaab的前缀函数为$[0,1,0,1,2,2,3]$。
        \subsubsection{朴素算法}
            直接按定义计算前缀函数:
            \begin{lstlisting}
// 朴素法求前缀函数O(n^3),下标从1开始
void prefix_func0(char t[], int n) {
    int i, k;
    for (i = 1; i <= n; i++)   // 对每一个子串
        for (k = 0; k < i; k++)  // 枚举前缀后缀长度,并判断是否相等
            if (!strncmp(t + 1, t + i - k + 1, k))
                pi[i] = k;
}
            \end{lstlisting}
        \subsubsection{第一个优化}
            第一个重要的事实是相邻的前缀函数值至多增加1。(如不然,会产生矛盾)\\
            所以当移动到下一个位置时,前缀函数要么增加1,要么不变或减少。
            实际上,该事实已经允许我们将复杂度降至$O(n^2)$。
            因为每一步中前缀函数至多增加1,因此在总的运行过程中,前缀函数至多增加n,同时也至多减小n。
            这意味着我们仅需进行$O(n)$次字符串比较,所以总复杂度为$O(n^2)$。
            \begin{lstlisting}
void prefix_func1(char t[], int n) {
    int i, j;
    i = 2, j = 1;
    while (i <= n) {
        if (t[i] == t[j])  // 加1
            pi[i] = pi[i - 1] + 1, ++j;
        else {  // 开始减
            pi[i] = pi[i - 1];
            while (strncmp(t + i - pi[i] + 1, t + 1, pi[i]))
                --pi[i];
            j = pi[i] + 1;
        }
        ++i;
    }
}
            \end{lstlisting}
        \subsubsection{第二个优化}
            考虑计算位置i+1的前缀函数$\pi$的值,如果$s[i+1]=s[\pi[i]+1]$,显然$\pi[i+1]=\pi[i]+1$。
            $$ \underbrace{\overbrace{s_1 ~ s_2 ~ s_3}^{\pi[i]} ~ \overbrace{s_4}^{s_4 = s_{i+1}}}_{\pi[i+1] = \pi[i] + 1} ~ \dots ~ \underbrace{\overbrace{s_{i-2} ~ s_{i-1} ~ s_{i}}^{\pi[i]} ~ \overbrace{s_{i+1}}^{s_4 = s_{i+1}}}_{\pi[i+1] = \pi[i] + 1} $$
            如果不是上述情况,即$s[i+1] \neq s[\pi[i]+1]$,我们需要尝试更短的字符串。为了加速,我们希望直接移动到
            最长的长度$j<\pi[i]$,使得在位置i的前缀性质仍得以保持,也即$s[1 \dots j] = s[i-j+1 \dots i]$:
            $$\overbrace{\underbrace{s_1 ~ s_2}_j ~ s_3 ~ s_4}^{\pi[i]} ~ \dots ~ \overbrace{s_{i-3} ~ s_{i-2} ~ \underbrace{s_{i-1} ~ s_{i}}_j}^{\pi[i]} ~ s_{i+1}$$
            实际上,如果我们找到了这样的j,我们仅需要再次比较$s[i+1]$和$s[j+1]$。如果它们相等,则$\pi[i+1]=j+1$,
            否则,我们就需要找小于j的最大的新的j使得前缀性质仍然保持,如此反复,直到$s[i+1]=s[j+1]$或者确实完全找不到(令$j=-1$)。
            最后$\pi[i+1]=j+1$。\\\\
            所以我们已经有了一个大致框架,现在仅剩的问题是对于满足$s[1 \dots j] = s[i-j+1 \dots i]$的j,
            如何快速找到小于j的最大的新的j,我们令新的j为k,使得$s[1 \dots k] = s[i-k+1 \dots i]$仍然满足。
            $$\overbrace{\underbrace{s_1 ~ s_2}_k ~ s_3 ~ s_4}^j ~ \dots ~ \overbrace{s_{i-3} ~ s_{i-2} ~ \underbrace{s_{i-1} ~ s_{i}}_k}^j ~s_{i+1}$$
            由上图,我们要求的是比j小的最大的k,而两边长度为j的前后缀本身是相等的,那么新的长为k的前后缀则可以只放到最左边
            长为j的前缀中去考虑:
            $$\overbrace{\underbrace{s_1 ~ s_2}_k ~ \underbrace{s_3 ~ s_4}_k}^j$$
            即$k=\pi[j]$,而$\pi[j]$之前已经求过了。
            \begin{lstlisting}
void prefix_func2(char t[], int n) {
    int i, j;
    pi[0] = -1, pi[1] = 0;  // 确实没有找到任何相等的
    for (i = 1; i < n; ++i) {
        j = pi[i];
        while (j >= 0 && t[j + 1] != t[i + 1])  // 若不相等,找更小的新的j
            j = pi[j];
        pi[i + 1] = j + 1;  //最后得出pi[i+1]
    }
}
            \end{lstlisting}
    
    \subsection{统计每个前缀出现次数}
        


    \chapter{数据结构}
    \include{tex/DataStructures/UnionFindSet}
    \include{tex/DataStructures/BinaryIndexedTree}
    \section{线段树}
\begin{lstlisting}
struct SegmentTree
{
    int l, r;
    ll sum, add;
    #define ls(x) (x << 1)
    #define rs(x) (x << 1 | 1)
    #define l(x) tree[x].l
    #define r(x) tree[x].r
    #define sum(x) tree[x].sum
    #define add(x) tree[x].add
} tree[N << 2];
void PushUp(int rt)
{
    sum(rt) = (sum(ls(rt)) + sum(rs(rt))) % mod;
}
void PushDown(int rt)
{
    if (add(rt))
    {
        sum(ls(rt)) = (sum(ls(rt)) + add(rt) * (r(ls(rt)) - l(ls(rt)) + 1)) % mod;
        sum(rs(rt)) = (sum(rs(rt)) + add(rt) * (r(rs(rt)) - l(rs(rt)) + 1)) % mod;
        add(ls(rt)) = (add(ls(rt)) + add(rt)) % mod;
        add(rs(rt)) = (add(rs(rt)) + add(rt)) % mod;
        add(rt) = 0;
    }
}
void SegmentTree_Build(int rt, int l, int r)
{
    l(rt) = l, r(rt) = r;
    if (l == r)
    {
        sum(rt) = weight[l];
        return;
    }
    int mid = (l + r) >> 1;
    SegmentTree_Build(ls(rt), l, mid);
    SegmentTree_Build(rs(rt), mid + 1, r);
    PushUp(rt);
    //向上更新
}
void Update(int rt, int l, int r, int d)
{
    if (l <= l(rt) && r(rt) <= r)
    {
        sum(rt) = (sum(rt) + (r(rt) - l(rt) + 1) * d) % mod;
        add(rt) = (add(rt) + d) % mod;
        return;
    }
    PushDown(rt);
    int mid = (l(rt) + r(rt)) >> 1;
    if (l <= mid)
        Update(ls(rt), l, r, d);
    if (r > mid)
        Update(rs(rt), l, r, d);
    PushUp(rt);
}
ll Query(int rt, int l, int r)
{
    if (l(rt) >= l && r(rt) <= r)
    {
        return sum(rt) % mod;
    }
    PushDown(rt);
    ll ret = 0;
    int mid = (l(rt) + r(rt)) >> 1;
    if (l <= mid)
        ret = (ret + Query(ls(rt), l, r)) % mod;
    if (r > mid)
        ret = (ret + Query(rs(rt), l, r)) % mod;
    return ret;
}    
\end{lstlisting}

    \section{主席树}
    又称“可持久化(权值)线段树”,主要用于查询区间第k小(大)值。效率高于归并树低于划分树。
\begin{lstlisting}
#include <bits/stdc++.h>

using namespace std;
const int N = 1e5 + 5;
struct SegTreeNode {
    int l, r, m;
    int ls, rs;  // 左儿子、右儿子
    int s;       // 结点总数
} tr[N << 5];
// tcnt表示当前空余的节点编号, rt[i]为时间点为i的线段树的根结点
int tcnt, rt[N];

int n, m, a[N], i2x[N], len;

inline int x2i(int x) {
    return lower_bound(i2x + 1, i2x + 1 + len, x) - i2x;
}

// 区间为[l,r), 返回子树的根结点
int build(int l, int r) {
    int x = tcnt++;
    int mid = (l + r) / 2;
    tr[x].l = l, tr[x].r = r, tr[x].m = mid;
    tr[x].s = 0;
    if (r - l == 1)
        return x;
    tr[x].ls = build(l, mid);
    tr[x].rs = build(mid, r);
    return x;
}

// 在pos处插入数, pre为上一个版本的根结点
int insert(int k, int pre) {
    int cur = tcnt++;
    tr[cur] = tr[pre];
    tr[cur].s++;
    if (tr[cur].r - tr[cur].l == 1)
        return cur;
    if (k < tr[cur].m)
        tr[cur].ls = insert(k, tr[cur].ls);
    else
        tr[cur].rs = insert(k, tr[cur].rs);
    return cur;
}

// 查询区间(x,y]中的第k大值
// tr[x] 和 tr[y] 表示时间点不一样的*两棵*的线段树中的节点
int query(int x, int y, int k) {
    // 当前区间的左子树中数的个数 = y时间的左子树中数的个数 - x时间的左子树中数的个数
    int s = tr[tr[y].ls].s - tr[tr[x].ls].s;
    if (tr[x].r - tr[x].l == 1)
        return tr[x].l;  // 此处返回的l不是下标,而是一个权值,是离散化后的权值
    if (k <= s)
        return query(tr[x].ls, tr[y].ls, k);
    else
        return query(tr[x].rs, tr[y].rs, k - s);
}

inline void init() {
    int i;
    // 离散化
    for (i = 1; i <= n; i++)
        i2x[i] = a[i];
    sort(i2x + 1, i2x + 1 + n);
    len = unique(i2x + 1, i2x + 1 + n) - i2x - 1;
    // 将下标当成时间序列,依次“新建”线段树
    rt[0] = build(1, len + 1);
    for (i = 1; i <= n; i++)
        rt[i] = insert(x2i(a[i]), rt[i - 1]);
}

inline void work() {
    int i, l, r, k;
    for (i = 1; i <= m; i++) {
        scanf("%d%d%d", &l, &r, &k);
        // 此处的查询,应该查的是两个时间点的两棵线段树,返回的是离散化后的权值,需要i2x恢复
        printf("%d\n", i2x[query(rt[l - 1], rt[r], k)]);
    }
}

int main() {
    int i;
    scanf("%d%d", &n, &m);
    for (i = 1; i <= n; i++)
        scanf("%d", &a[i]);
    init();
    work();

    return 0;
}
\end{lstlisting}
    \include{tex/DataStructures/SegmentTreeLazy}
    \include{tex/DataStructures/MergingTree}
    \section{划分树}

\subsection{简介}
    划分树是模拟快速排序的。快速排序,自顶向下越来越有序。划分树是在进行快速排序的过程中记录当前子段中每个位置的数是否进入左子树,
    并用前缀和的思想统计它们。之后再查询时便可依据进入左子树的个数计算第k个数是在左子树还是右子树。

\subsection{区间第k小值}
    \subsubsection{问题简述}
        给定n个数,m次查询,每次查询[l,r]内从小到大第k个数,输出这个数。
    \subsubsection{解决方案}
\begin{lstlisting}
#include <bits/stdc++.h>

using namespace std;
typedef long long ll;
const int N = 1e5 + 10;

int a[N], sorted[N], n;
int seg[21][N], toleft[21][N];

// 建树
void build(int l, int r, int dep) {
    if (l == r) {
        seg[dep + 1][l] = seg[dep][l];
        return;
    }
    int i, m = (l + r) / 2, lp = l, rp = m + 1, cnt = 0, t = 0;
    // cnt: 该节点内比sorted[m]小的元素个数
    for (i = l; i <= r; i++)
        cnt += seg[dep][i] < sorted[m];

    for (i = l; i <= r; i++) {
        if (i == l)
            toleft[dep][i] = 0;
        else
            toleft[dep][i] = toleft[dep][i - 1];

        if (seg[dep][i] < sorted[m])
            seg[dep + 1][lp++] = seg[dep][i], toleft[dep][i]++;
        else if (seg[dep][i] > sorted[m])
            seg[dep + 1][rp++] = seg[dep][i];
        else {                        // ==
            if (t < m - l + 1 - cnt)  // m-l+1-cnt: 左边最多可以放多少个sorted[m]
                seg[dep + 1][lp++] = seg[dep][i], toleft[dep][i]++, t++;
            else
                seg[dep + 1][rp++] = seg[dep][i];
        }
    }

    build(l, m, dep + 1);
    build(m + 1, r, dep + 1);
}

// 询问区间[l,r]内从小到大的第k个值,当前区间为[sl,sr](初始时为[1,n]),当前深度为dep(初始时为1)
int query(int l, int r, int k, int sl, int sr, int dep) {
    if (r == l)
        return seg[dep][l];
    int m = (sl + sr) / 2;
    int tlsl = (l - 1 >= sl ? toleft[dep][l - 1] : 0), tlsr = toleft[dep][sr];
    int tlr = toleft[dep][r];
    if (k <= tlr - tlsl)
        return query(sl + tlsl, m - (tlsr - tlr), k, sl, m, dep + 1);
    else
        return query(m + 1 + l - sl - tlsl, sr - (sr - r - (tlsr - tlr)), k - (tlr - tlsl), m + 1, sr, dep + 1);
}

int main() {
    int m, i, l, r, k;
    scanf("%d%d", &n, &m);

    // 以下4行是初始化
    for (i = 1; i <= n; i++)
        scanf("%d", &a[i]), sorted[i] = seg[1][i] = a[i];
    sort(sorted + 1, sorted + 1 + n);
    build(1, n, 1);

    for (i = 1; i <= m; i++) {
        scanf("%d%d%d", &l, &r, &k);
        int ans = query(l, r, k, 1, n, 1);
        printf("%d\n", ans);
    }

    return 0;
}
\end{lstlisting}
    \include{tex/DataStructures/LeftPartialTree}
    \section{线段树练习}

\subsection{区间最大连续子段和}
\begin{lstlisting}
#include <bits/stdc++.h>

using namespace std;
const int N = 5e4 + 10;  // 数组大小,记得改

struct Node {
    int l, r, m;
    int s, f, fl, fr;  // 区间的和,区间最大子段和,包含左端点的最大子段和,包含右端点的最大子段和
} s[N * 4];

// 构建线段树
void build(int l, int r, int i) {
    Node& fa = s[i];
    fa.l = l, fa.r = r, fa.m = (l + r) / 2;
    fa.s = fa.f = fa.fl = fa.fr = 0;
    if (r - l == 1)
        return;
    build(l, fa.m, i * 2);
    build(fa.m, r, i * 2 + 1);
}

// 自底向上更新
void pushup(int i) {
    Node &fa = s[i], &lson = s[i * 2], &rson = s[i * 2 + 1];  // 父亲 左儿子 右儿子
    fa.s = lson.s + rson.s;
    fa.f = max(max(lson.f, rson.f), lson.fr + rson.fl);
    fa.fl = max(lson.fl, lson.s + rson.fl);
    fa.fr = max(rson.fr, rson.s + lson.fr);
}

// 单点更新
void update(int x, int p, int i) {
    Node& fa = s[i];
    if (fa.l == p && fa.r - fa.l == 1) {
        fa.s = fa.f = fa.fl = fa.fr = x;
        return;
    }
    if (p < fa.m)
        update(x, p, i * 2);
    else
        update(x, p, i * 2 + 1);
    // 向上更新
    pushup(i);
}

// 作为查询的返回值
struct Ret {
    int f, s, fl, fr;
};

// 查询
Ret query(int l, int r, int i) {
    Node& fa = s[i];
    if (fa.l == l && fa.r == r)
        return {fa.f, fa.s, fa.fl, fa.fr};
    if (r <= fa.m)
        return query(l, r, i * 2);
    else if (l >= fa.m)
        return query(l, r, i * 2 + 1);
    else {
        Ret lret = query(l, fa.m, i * 2);
        Ret rret = query(fa.m, r, i * 2 + 1);
        return {max(max(lret.f, rret.f), lret.fr + rret.fl), lret.s + rret.s, max(lret.fl, lret.s + rret.fl), max(rret.fr, rret.s + lret.fr)};
    }
}

int main() {
    ios::sync_with_stdio(0);
    cin.tie(0);

    int n, m, i, x, l, r;
    cin >> n;
    build(1, n + 1, 1);  // 不要忘了初始化
    for (i = 1; i <= n; i++) {
        cin >> x;
        update(x, i, 1);  // 修改元素
    }
    cin >> m;  // 询问
    while (m--) {
        cin >> l >> r;
        cout << query(l, r + 1, 1).f << endl;  // 询问[l,r],实际查询[l,r+1),统一用左闭右开区间
    }

    return 0;
}
\end{lstlisting}

\subsection{最长单峰序列的下标的最小字典序和最大字典序}
    简化问题,只考虑最小字典序。那么与LIS类似,能选就选就可以得到最小字典序。\\\\
    选哪些?\textbf{被包含在最长单峰序列中的元素}\\\\
    如何知道是否被包含?\\
    类比LIS,如果a[i]被包含,那么\textbf{最长长度-a[i]之前选的个数=以a[i]为开头的最长长度}。\\
    然后考虑两种情况:1.a[i]在峰的左侧,已选个数+以a[i]为开头的单峰序列长度;
    2.a[i]为峰或在峰的右侧,已选个数+以a[i]为开头的递减序列长度\\\\
    于是我们就需要维护两个量:\\
    1.\textbf{以a[i]为开头的单峰序列长度}\\
    2.\textbf{以a[i]为开头的递减序列长度}\\\\
    递减的就直接LIS很简单。对于单峰:\\
    倒过来维护,设以a[i]为开头的单峰序列最长长度为peak[i],\\
    \textbf{peak[i]=max(以比a[i]大的数为开头的单峰最大长度+1, 以a[i]为开头的递减序列最大长度)}\\
    而维护以比a[i]大的数为开头的单峰最大长度可以用\textbf{权值线段树}
\begin{lstlisting}
/*
求 最长 单峰 序列 的 下标 的 最小字典序 和 最大字典序

首先用LIS求出以a[i]开头的严格递减序列的最长长度,
再用线段树求以a[i]开头的单峰序列的最长长度(倒着求)
求最小字典序即要求:
1. a[i] > pre && L - cnt == peak[i]
2. a[i] < pre && L - cnt == d[i]
*/

#include <bits/stdc++.h>

using namespace std;
const int N = 3e5 + 10;
int a[N], n;
int i2x[N];   // 离散化
int m;        // 离散化后的长度
int c[N];     // LIS的dp数组
int d[N];     // d[i]:以a[i]开头递减序列最长长度
int peak[N];  // peak[i]:以a[i]开头的单峰序列最长长度
int ans[N];

inline int x2i(int x) {
    return lower_bound(i2x + 1, i2x + 1 + m, x) - i2x;
}

// 权值线段树,维护
struct Node {
    int l, r, m;
    int mx;  // [l,r)中的最大值
} t[4 * N];  // 数组大小不要忘记 * 4

inline void pushup(int x) {
    Node& cur = t[x];
    if (cur.r - cur.l == 1)
        return;
    Node &lch = t[x * 2], &rch = t[x * 2 + 1];
    cur.mx = max(lch.mx, rch.mx);
}

void build(int l, int r, int x) {
    Node& cur = t[x];
    cur.l = l, cur.r = r, cur.m = (l + r) / 2;
    cur.mx = 0;
    if (r - l == 1)
        return;
    build(l, cur.m, x * 2);
    build(cur.m, r, x * 2 + 1);
    pushup(x);
}

void update(int l, int r, int x, int v) {
    Node& cur = t[x];
    if (cur.l == l && cur.r == r) {
        cur.mx = v;
        return;
    }
    if (r <= cur.m)
        update(l, r, x * 2, v);
    else if (l >= cur.m)
        update(l, r, x * 2 + 1, v);
    else
        update(l, cur.m, x * 2, v), update(cur.m, r, x * 2 + 1, v);
    pushup(x);
}

int query(int l, int r, int x) {
    Node& cur = t[x];
    if (cur.l == l && cur.r == r)
        return cur.mx;
    int mx = 0;
    if (r <= cur.m)
        mx = query(l, r, x * 2);
    else if (l >= cur.m)
        mx = query(l, r, x * 2 + 1);
    else
        mx = max(query(l, cur.m, x * 2), query(cur.m, r, x * 2 + 1));
    return mx;
}

inline void LIS() {
    int i, j, mx = 0;
    for (i = 1; i <= n; i++) {
        j = lower_bound(c + 1, c + 1 + mx, a[i]) - c;
        d[i] = j, c[j] = a[i], mx = max(mx, j);
    }
}

inline void work() {
    int i;

    // 求以a[i]为开头的最长下降子序列的长度
    reverse(a + 1, a + 1 + n);
    LIS();
    reverse(a + 1, a + 1 + n);
    reverse(d + 1, d + 1 + n);

    // 通过权值线段树维护以a[i]为开头的单峰序列的最大长度
    build(1, m + 1, 1);
    for (i = n; i >= 1; i--) {
        int x = 0;
        if (a[i] + 1 < m + 1)
            x = query(a[i] + 1, m + 1, 1);   // 查询以比a[i]大的数为开头的单峰序列的最大长度
        peak[i] = max(x + 1, d[i]);          // a[i]可以在峰的左侧(原单峰加上a[i]),也可以就是峰或峰的右侧(单调减)
        update(a[i], a[i] + 1, 1, peak[i]);  // 将以a[i]为开头的单峰添加进权值线段树
    }
}

inline int getAns() {
    int L = query(1, m + 1, 1);  // 单峰最大长度
    int cnt = 0, pre = 0, i;     // cnt是已经求得的个数,pre是上一个的值
    // 求单峰的上升部分
    for (i = 1; i <= n; i++) {
        // 如果a[i]被包含在最长里
        if (a[i] > pre && L - cnt == peak[i])
            ans[++cnt] = i, pre = a[i];
        // 如果可以转为下降,就停止上升
        else if (a[i] < pre && L - cnt == d[i])
            break;
    }
    // 单峰的下降部分
    for (; i <= n; i++)
        if (a[i] < pre && L - cnt == d[i])
            ans[++cnt] = i, pre = a[i];
    return cnt;
}

int main() {
    int i;
    while (scanf("%d", &n) != EOF) {
        for (i = 1; i <= n; i++) {
            scanf("%d", &a[i]);
            i2x[i] = a[i];
            ans[i] = c[i] = d[i] = peak[i] = 0;
        }

        sort(i2x + 1, i2x + 1 + n);
        m = unique(i2x + 1, i2x + 1 + n) - i2x - 1;
        for (i = 1; i <= n; i++)
            a[i] = x2i(a[i]);

        work();
        int cnt = getAns();
        for (i = 1; i <= cnt; i++) {
            if (i > 1)
                printf(" ");
            printf("%d", ans[i]);
        }
        printf("\n");

        for (i = 1; i <= n; i++)
            ans[i] = c[i] = d[i] = peak[i] = 0;

        reverse(a + 1, a + 1 + n);
        work();
        cnt = getAns();
        for (i = cnt; i >= 1; i--) {
            if (i < cnt)
                printf(" ");
            printf("%d", n - ans[i] + 1);
        }
        printf("\n");
    }

    return 0;
}
\end{lstlisting}

\subsection{HDU6602 Longest Subarray}
    \subsubsection{题目描述}
        给你一个数组,数的范围是$[1,C]$,给定$K$,让你找一个最长的区间使得区间内任意一个出现的数在该区间内的数量都大于$K$。
    \subsubsection{解决方案}
        主要思路:\textbf{尺取法,枚举右端点,用线段树维护左端点的可行区间}\\
        本题求一个最大的子区间,满足区间内的数字要么出现次数大于等于k次,要么没出现过。给定区间内的数字范围是1到c。\\
        如果r为右边界,对于一种数字x,满足条件的左边界l的范围是r左边第一个x出现的位置+1(即这段区间内没有出现过x,
        如果x在1到r内都没有出现过,那么1到r自然都是l的合法范围),以及1到从右往左数数第k个x出现的位置(即这段区间内
        的x出现次数大于等于k)。所以我们只要找到同时是c种数字的合法左边界的位置中最小的,然后枚举所有的i作为右边界即可得出答案。\\
        由此,我们可以令线段树的叶子节点中的值表示可以作为多少种数字的左端点。然后对每一次查询,我们都尽量向左查最大值为c的节点,
        返回下标。
\begin{lstlisting}
/*
尺取法
移动右端点,用线段树维护左端点的可行区间

思维;线段树区间修改,区间最大值,lazy标记
*/

#include <bits/stdc++.h>

using namespace std;
const int N = 2e5 + 10;

int n;  // 长度
int c, k, a[N];

struct Node {
    int l, r, m;
    int mx;   // [l,r)中的最大值
    int tag;  // lazy标记
} t[4 * N];   // 数组大小不要忘记 * 4

inline void pushdown(int x) {
    Node& cur = t[x];
    if (cur.r - cur.l == 1)
        return;
    Node &lch = t[x * 2], &rch = t[x * 2 + 1];
    lch.mx += cur.tag, rch.mx += cur.tag;
    lch.tag += cur.tag, rch.tag += cur.tag;
    cur.tag = 0;
}

inline void pushup(int x) {
    Node& cur = t[x];
    if (cur.r - cur.l == 1)
        return;
    Node &lch = t[x * 2], &rch = t[x * 2 + 1];
    cur.mx = max(lch.mx, rch.mx);
}

void build(int l, int r, int x) {
    Node& cur = t[x];
    cur.l = l, cur.r = r, cur.m = (l + r) / 2;
    cur.mx = 0, cur.tag = 0;  // 初始化最大值、lazy标记
    if (r - l == 1)
        return;
    build(l, cur.m, x * 2);
    build(cur.m, r, x * 2 + 1);
    pushup(x);  // 若初始值非0,则这句一定要加
}

void update(int l, int r, int x, int v) {
    Node& cur = t[x];
    if (cur.l == l && cur.r == r) {
        cur.tag += v, cur.mx += v;  // 打标记的时候一定是同时更新max的
        return;
    }
    pushdown(x);
    if (r <= cur.m)
        update(l, r, x * 2, v);
    else if (l >= cur.m)
        update(l, r, x * 2 + 1, v);
    else
        update(l, cur.m, x * 2, v), update(cur.m, r, x * 2 + 1, v);
    pushup(x);
}

int query(int l, int r, int x) {
    Node& cur = t[x];
    pushdown(x);
    if (cur.r - cur.l == 1)
        return cur.l;
    Node &lch = t[x * 2], &rch = t[x * 2 + 1];
    int p = 0;
    if (r <= cur.m)
        p = query(l, r, x * 2);
    else if (l >= cur.m)
        p = query(l, r, x * 2 + 1);
    else {
        if (lch.mx == c)
            p = query(l, cur.m, x * 2);
        else
            p = query(cur.m, r, x * 2 + 1);
    }

    return p;
}

vector<int> pos[N];  // pos[x]是x出现的所有位置

int main() {
    int i, l, r, sz, ans = 0;

    while (scanf("%d%d%d", &n, &c, &k) != EOF) {
        for (i = 1; i <= n; i++)
            scanf("%d", &a[i]);
        if (n == 0) {
            printf("0\n");
            continue;
        }
        if (k == 0 || k == 1) {
            printf("%d\n", n);
            continue;
        }
        ans = 0;
        for (i = 1; i <= c; i++)
            pos[i].clear();

        build(1, n + 1, 1);

        // 枚举右端点
        for (i = 1; i <= n; i++) {
            // 对于[i,i],它可以作为除去它本身以外的c-1个数的左端点。所以位置i要加上c-1
            update(i, i + 1, 1, c - 1);
            // 关于a[i],在它上一次出现的位置和当前位置i之间的位置,在转移来i之前是可以当做a[i]的左端点的(即a[i]出现0次)
            // 但是现在右端点为a[i],那么这些位置就不可再作为a[i]的左端点了(除非k<=1,这个作为特殊情况考虑),所以这一段要减1
            sz = pos[a[i]].size();
            if (sz) {
                l = *pos[a[i]].rbegin();
                if (l + 1 < i)
                    update(l + 1, i - 1 + 1, 1, -1);
            } else {
                if (1 < i)
                    update(1, i, 1, -1);
            }
            // 当前a[i]还可以使之前已经出现k-1次的一段由不可以作为a[i]的左端点变为可以作为a[i]的左端点。(如果有的话)
            if (sz >= k - 1) {
                r = pos[a[i]][sz - k + 1], l = 1;
                if (sz >= k)
                    l = pos[a[i]][sz - k] + 1;
                if (l <= r)
                    update(l, r + 1, 1, 1);
            }
            pos[a[i]].push_back(i);
            // 最后查询能被c个数作为左端点的最左位置
            if (t[1].mx == c)
                ans = max(ans, i - query(1, i + 1, 1) + 1);
        }
        cout << ans << endl;
    }

    return 0;
}
\end{lstlisting}
    % %数据结构

    \chapter{图论}
    % \include{tex/GraphTheory/ShortPath}
    % %最短路
    % \include{tex/GraphTheory/MST}
    % %最小生成树
    % \section{树的直径}
\subsection{树形DP求树的直径}
仅能求出直径长度,无法得知路径信息,可处理负权边。
\begin{lstlisting}
int dp[N];
//dp[rt] 以rt为根的子树 从rt出发最远可达距离
/*
    对于每个结点x f[x]:经过节点x的最长链长度
*/
void DP(int rt)
{
    dp[rt]=0;//单点
    vis[rt]=1;
    for(int i=head[rt];i;i=nxt[i])
    {
        int s=ver[i];
        if(!vis[s])
        {
            DP(s);
            diameter=max(diameter,dp[rt]+dp[s]+edge[i]);
            dp[rt]=max(dp[rt],dp[s]+edge[i]);
        }
    }
}
\end{lstlisting}
\subsection{两次BFS/DFS求树的直径}
无法处理负权边,容易记录路径
\begin{lstlisting}
void DFS(int start,bool record_path)
{
    vis[start]=1;
    for(int i=head[start];i;i=nxt[i])
    {
        int s=ver[i];
        if(!vis[s])
        {
            dis[s]=dis[start]+edge[i];
            if(record_path) path[s]=i;
            DFS(s,record_path);
        }
    }
    vis[start]=0;//清理
}
\end{lstlisting}
例题分析

P3629 [APIO2010]巡逻(两种求树直径方法的综合应用)

P1099 树网的核(枚举)

\section{最近公共祖先(LCA)}
\subsection{树上倍增}
\begin{lstlisting}
void BFS()
{
    queue<int> q;
    q.push(1);
    d[1] = 1;
    while (!q.empty())
    {
        int x = q.front();
        q.pop();
        for (int i = head[x]; i; i = nxt[i])
        {
            int y = ver[i];
            if (!d[y])
            {
                d[y] = d[x] + 1;
                fa[y][0] = x;
                for (int j = 1; j <= k; j++)
                {
                    fa[y][j] = fa[fa[y][j - 1]][j - 1];
                }
                q.push(y);
            }
        }
    }
}
int LCA(int x, int y)
{
    if (d[x] < d[y])
        swap(x, y);
    for (int i = k; i >= 0; i--)
        if (d[fa[x][i]] >= d[y])
            x = fa[x][i];
    if (x == y)
        return y;
    for (int i = k; i >= 0; i--)
        if (fa[x][i] != fa[y][i])
            x = fa[x][i], y = fa[y][i];
    return fa[x][0];
}
\end{lstlisting}
\subsection{Tarjan}
\begin{lstlisting}
int Find(int x)
{
    if (x == fa[x])
        return x;
    return fa[x] = Find(fa[x]);
}
void Tarjan(int x)
{
    vis[x] = 1;
    for (int i = head[x]; i; i = nxt[i])
    {
        int y = ver[i];
        if (!vis[y])
        {
            Tarjan(y);
            fa[y] = x;
        }
    }
    for (int i = 0; i < q[x].size(); i++)
    {
        int y = q[x][i].first, id = q[x][i].second;
        if (vis[y] == 2)
            lca[id] = Find(y);
    }
    vis[x] = 2;
}
\end{lstlisting}

\section{树上差分}
例题分析

POJ3417 Network(LCA,树上差分,边覆盖)

6302 雨天的尾巴(LCA,树上差分,点覆盖,权值线段树,线段树合并)

P1600 天天爱跑步(LCA,树上差分)

\section{LCA的综合应用}
例题分析

CH56C 异象石(dfn时间戳,LCA)

P4180 【模板】严格次小生成树[BJWC2010](树上倍增)
\begin{lstlisting}
#include <algorithm>
#include <iostream>
#include <math.h>
#include <queue>
#include <stdio.h>
using namespace std;
const int N = 1e5 + 50;
const int M = 6e5 + 50;
#define ll long long
#define pii pair<int, int>
#define inf 0x3f3f3f3f
int n, m, k, F[N][20], d[N], fa[N];
int head[N], ver[M], nxt[M], edge[M], tot;
ll G[N][20][2];
void add(int x, int y, int z)
{
    ver[++tot] = y, nxt[tot] = head[x], head[x] = tot, edge[tot] = z;
}
struct edge
{
    int x, y, z;
    bool used;
    bool operator<(const edge &e) const
    {
        return z < e.z;
    }
} e[M];
int Find(int x)
{
    if (fa[x] == x)
        return x;
    return fa[x] = Find(fa[x]);
}
ll Kruskal()
{
    for (int i = 1; i <= n; i++)
        fa[i] = i;
    sort(e + 1, e + 1 + m);
    ll ans = 0;
    int cnt = 0;
    for (int i = 1; i <= m; i++)
    {
        int fx = Find(e[i].x), fy = Find(e[i].y);
        if (fx != fy)
        {
            fa[fx] = fy;
            ans += e[i].z;
            e[i].used = true;
            cnt++;
            if (cnt >= n - 1)
                break;
        }
    }
    return ans;
}
void BFS()
{
    k = log2(n) + 1;
    queue<int> q;
    q.push(1), d[1] = 1;
    for (int i = 0; i <= k; i++)
        G[1][i][0] = G[1][i][1] = -inf;
    while (q.size())
    {
        int x = q.front();
        q.pop();
        for (int i = head[x]; i; i = nxt[i])
        {
            int y = ver[i];
            if (!d[y])
            {
                d[y] = d[x] + 1;
                F[y][0] = x;
                G[y][0][0] = edge[i];
                G[y][0][1] = -inf;
                for (int j = 1; j <= k; j++)
                {
                    F[y][j] = F[F[y][j - 1]][j - 1];
                    G[y][j][0] = max(G[y][j - 1][0], G[F[y][j - 1]][j - 1][0]);
                    if (G[y][j - 1][0] == G[F[y][j - 1]][j - 1][0])
                        G[y][j][1] = max(G[y][j - 1][1], G[F[y][j - 1]][j - 1][1]);
                    else if (G[y][j - 1][0] > G[F[y][j - 1]][j - 1][0])
                        G[y][j][1] = max(G[F[y][j - 1]][j - 1][0], G[y][j - 1][1]);
                    else
                        G[y][j][1] = max(G[y][j - 1][0], G[F[y][j - 1]][j - 1][1]);
                }
                q.push(y);
            }
        }
    }
}
pii LCA(int x, int y)
{
    ll val1 = -inf, val2 = -inf;
    if (d[y] > d[x])
        swap(x, y);
    for (int i = k; i >= 0; i--)
    {
        if (d[F[x][i]] >= d[y])
        {
            if (G[x][i][0] > val1)
                val1 = G[x][i][0], val2 = max(val2, G[x][i][1]);
            else if (G[x][i][0] < val1)
                val2 = max(val2, G[x][i][0]);
            x = F[x][i];
        }
    }
    if (x == y)
        return make_pair(val1, val2);
    for (int i = k; i >= 0; i--)
    {
        if (F[x][i] != F[y][i])
        {
            val1 = max(val1, max(G[x][i][0], G[y][i][0]));
            val2 = max(val2, (val1 == G[x][i][0]) ? G[x][i][1] : G[x][i][0]);
            val2 = max(val2, (val1 == G[y][i][0]) ? G[y][i][1] : G[y][i][0]);
            x = F[x][i], y = F[y][i];
        }
    }
    val1 = max(val1, max(G[x][0][0], G[y][0][0]));
    val2 = max(val2, (val1 == G[x][0][0]) ? G[x][0][1] : G[x][0][0]);
    val2 = max(val2, (val1 == G[y][0][0]) ? G[y][0][1] : G[y][0][0]);
    return make_pair(val1, val2);
}
int main()
{
    scanf("%d%d", &n, &m);
    for (int i = 1; i <= m; i++)
        scanf("%d%d%d", &e[i].x, &e[i].y, &e[i].z), e[i].used = false;
    ll sum = Kruskal(), ans = 0x3f3f3f3f3f3f3f3f;
    for (int i = 1; i <= m; i++)
        if (e[i].used)
            add(e[i].x, e[i].y, e[i].z), add(e[i].y, e[i].x, e[i].z);
    BFS();
    for (int i = 1; i <= m; i++)
    {
        if (!e[i].used)
        {
            pii temp = LCA(e[i].x, e[i].y);
            if (e[i].z > temp.first)
                ans = min(ans, sum - temp.first + e[i].z);
            else if (e[i].z == temp.first)
                ans = min(ans, sum - temp.second + e[i].z);
        }
    }
    cout << ans << endl;
    //system("pause");
    return 0;
}
\end{lstlisting}

P1084 疫情控制(二分,树上倍增,贪心)
    % %树的直径与最近公共祖先
    % \section{基环树}
    % %TODO 基环树 
    % \include{tex/GraphTheory/SystemOfDifferenceConstraints}
    % %负环与差分约束
    % \include{tex/GraphTheory/TarjanDCC}
    % %Tarjan算法与无向图连通性
    % \section{Tarjan算法与有向图连通性}
\subsection{强连通分量(SCC)判定法则}
\begin{lstlisting}
void Tarjan(int x)
{
    dfn[x]=low[x]=++num;
    stack[++top]=x,in_stack[x]=true;
    for(int i=head[x];i;i=nxt[i])
    {
        int y=ver[i];
        if(!dfn[y])
        {
            Tarjan(y);
            low[x]=min(low[x],low[y]);
        }
        else if(in_stack[y])
            low[x]=min(low[x],dfn[y]);
    }
    if(dfn[x]==low[x])
    {
        cnt++;
        int y;
        do
        {
            y=stack[top--],in_stack[y]=false;
            color[y]=cnt, scc[cnt].push_back(y);
        } while (x!=y);
    }
}
\end{lstlisting}
\subsection{SCC -> DAG}
\begin{lstlisting}
void SCC()
{
    for (int i = 0; i <= n; i++)
        if (!dfn[i])
            Tarjan(i);
    //缩点
    for (int x = 1; x <= n; x++)
    {
        for (int i = head[x]; i; i = nxt[i])
        {
            int y = ver1[i];
            if (color[x] != color[y])
                add_c(color[x], color[y]);
        }
    }
}
\end{lstlisting}
例题分析

POJ1236 Network of Schools(SCC->DAG,入度出度)

P3275 [SCOI2011]糖果(SPFA TLE,SCC->DAG,Topo,DP)

\subsection{有向图的必经点与必经边}
对于有向无环图(DAG):

    在原图中按照拓扑序进行动态规划,求出起点S到图中每个点 x 的路径条数 fs[x]。

    在反图上再次按照拓扑序进行动态规划,求出每个点 x 到终点T的路径条数 ft[x]。

    显然,fs[T] 表示从S到T的路径总条数。根据乘法原理:

    1:对于一条有向边 (x,y),若 fs[x]*ft[y]=fs[T],则 (x,y) 是有向无环图从S到T的必经边。
    
    2:对于一个点 x,若 fs[x]*ft[x]=fs[T],则 x 是有向无环图从S到T的必经点。

    路径条数规模较大,可对大质数取模后保存,但有概率误判。


例题分析

6703 PKU ACM Team's Excursion (DAG必经边,枚举,DP)

\subsection{2-SAT问题}

原命题与逆否命题互为等价命题,在建立有向边关系时要注意对称性。

例题分析

POJ3678 Katu Puzzle(2-SAT,Tarjan,,SCC)

POJ3683 Priest John's Busiest Day(2-SAT方案构造)

\begin{lstlisting}
for (int i = 1; i <= 2 * n; i++)
{
    val[i] = color[i] > color[opp[i]];
    //若c[i] 大于 c[opp[i]]   opp[i]赋0 
}
\end{lstlisting}
    
    % %Tarjan算法与有向图连通性
    % \section{二分图的匹配}
\subsection{二分图判定}
    一张无向图是二分图,当且仅当图中不存在奇环(长度为奇数的环)。
\begin{lstlisting}
//染色法判定奇环
bool DFS(int x,int color)
{
    vis[x]=color;
    for(int i=head[x];i;i=nxt[i])
    {
        int y=ver[i];
        if(!vis[y])
        {
            if(!DFS(y,3-color)) return false;
        }
        else if(vis[y]==color) return false;
    }
    return true;
}
\end{lstlisting}

例题分析

P1525 关押罪犯(判定二分图,二分)

\subsection{二分图最大匹配}
二分图匹配的模型要素

1:节点能分成独立的两个集合,每个集合内部有 0 条边。 “0要素”

2:每个节点只能与 1 条匹配边相连。 “1要素”
\begin{lstlisting}
\\匈牙利算法
\\在主函数中对每个左部节点调用寻找增广路时,需要对 vis 重置。
bool DFS(int x)
{
    for (int i = head[x]; i; i = nxt[i])
    {
        int y = ver[i];
        if (!vis[y])
        {
            vis[y] = 1;
            if (!match[y] || DFS(match[y]))
            {
                match[y] = x;
                return true;
            }
        }
    }
    return false;
}
\end{lstlisting}

例题分析

6801 棋盘覆盖(奇偶染色)

6802 車的放置(行列)

6803 导弹防御塔(二分,拆点多重匹配)

\subsection{二分图带权匹配}
二分图带权最大匹配的前提是匹配数最大,然后再最大化匹配边的权值总和。
\begin{lstlisting}
/*KM 稠密图上效率高于费用流,但是有较大局限性,只能在满足“带权最大匹配一定是完备匹配”的图中正确求解。
w[][]:边权
la[],lb[]: 左,右部点顶标
visa[],visb[]: 访问标记,是否在交错树中
ans: Σw[match[i]][i]
*/
bool DFS(int x)
{
    visa[x] = true;
    for (int y = 1; y <= n; y++)
    {
        if (!visb[y])
        {
            double temp = fabs(la[x] + lb[y] - w[x][y]);//对于浮点数,相等子图的判定
            if (temp < eps)
            {
                visb[y] = true;
                if (!match[y] || DFS(match[y]))
                {
                    match[y] = x;
                    return true;
                }
            }
            else
                upd[y] = min(upd[y], la[x] + lb[y] - w[x][y]);
        }
    }
    return false;
}
void KM()
{
    for (int i = 1; i <= n; i++)
    {
        la[i] = -inf;
        lb[i] = 0;
        for (int j = 1; j <= n; j++)
            la[i] = max(la[i], w[i][j]);
    }
    for (int i = 1; i <= n; i++)
    {
        while (true)
        {
            memset(visa, 0, sizeof(visa));
            memset(visb, 0, sizeof(visb));
            for (int j = 1; j <= n; j++)
                upd[j] = inf;
            if (DFS(i))
                break;
            else
            {
                delta = inf;
                for (int j = 1; j <= n; j++)
                    if (!visb[j])
                        delta = min(delta, upd[j]);
                for (int j = 1; j <= n; j++)
                {
                    if (visa[j])
                        la[j] -= delta;
                    if (visb[j])
                        lb[j] += delta;
                }
            }
        }
    }
}
\end{lstlisting}

例题分析

POJ3565 Ants(三角形不等式,二分图带权最小匹配)
    % %二分图的匹配
    % \section{二分图的覆盖与独立集}
\subsection{二分图最小点覆盖}
二分图最小覆盖模型特点:

每条边有 2 个端点,二者至少选择一个。  “2要素”
\subsubsection{König's theorem}
二分图最小点覆盖包含的点数等于二分图最大匹配包含的边数。

例题分析

POJ1325 Machine Schedule(二分图最小覆盖)

POJ2226 Muddy Fields(行列连续块,二分图最小覆盖)

\subsection{二分图最大独立集}

无向图 G 的最大团等于其补图 G' 的最大独立集。(补图转化)

设G是有 n 个节点的二分图,G 的最大独立集的大小等于 n 减去最大匹配数。

例题分析

6901 骑士放置 (奇偶染色)

\subsection{有向无环图的最小路径点覆盖}
给定一张有向无环图,要求用尽量少的不相交的简单路径,覆盖有向无环图的所有顶点(也就是每个顶点恰好被覆盖一次)。
这个问题被称为有向无环图的最小路径点覆盖,简称“最小路径覆盖”。

有向无环图 G 的最小路径点覆盖包含的路径条数,等于n(有向无环图的点数)减去拆点二分图 G2 的最大匹配数。

若简单路径可相交,即一个节点可被覆盖多次,这个问题称为有向无环图的最小路径可重复点覆盖。

对于这个问题,可先对G求传递闭包,得到有向无环图 G',再在 G' 上求一般的(路径不可相交的)最小路径点覆盖。

例题分析

6902 Vani和Cl2捉迷藏(最小路径可重复点覆盖,构造方案)
\begin{lstlisting}
// 构造方案,先把所有路径终点(左部非匹配点)作为藏身点
for (int i = 1; i <= n; i++) succ[match[i]] = true;
for (int i = 1, k = 0; i <= n; i++)
    if (!succ[i]) hide[++k] = i;
memset(vis, 0, sizeof(vis));
bool modify = true;
while (modify) {
    modify = false;
    // 求出 next(hide)
    for (int i = 1; i <= ans; i++) 
        for (int j = 1; j <= n; j++)
            if (cl[hide[i]][j]) vis[j] = true;
    for (int i = 1; i <= ans; i++)
        if (vis[hide[i]]) {
            modify = true;
            // 不断向上移动
            while (vis[hide[i]]) hide[i] = match[hide[i]];
        }
}
for (int i = 1; i <= ans; i++) printf("%d ", hide[i]);
cout << endl;
\end{lstlisting}
    % %二分图的覆盖与独立集
    % \section{网络流初步}
\subsection{最大流}
\subsubsection{Edmonds Karp增广路}
\begin{lstlisting}
bool BFS() {
    memset(vis, 0, sizeof(vis));
    queue<int> q;
    q.push(S); vis[S] = 1;
    incf[S] = inf; // 增广路上各边的最小剩余容量
    while (q.size()) {
        int x = q.front(); q.pop();
        for (int i = head[x]; i; i = Next[i])
            if (edge[i]) {
                int y = ver[i];
                if (vis[y]) continue;
                incf[y] = min(incf[x], edge[i]);
                pre[y] = i; // 记录前驱,便于找到最长路的实际方案
                q.push(y), vis[y] = 1;
                if (y == t) return 1;
            }
    }
    return 0;
}
void Update() { // 更新增广路及其反向边的剩余容量
    int x = t;
    while (x != s) {
        int i = pre[x];
        edge[i] -= incf[t];
        edge[i ^ 1] += incf[t]; // 利用“成对存储”的xor 1技巧
        x = ver[i ^ 1];
    }
    maxflow += incf[t];
}
\end{lstlisting}

\subsubsection{Dinic}
\begin{lstlisting}
//可加入当前弧优化(&):在增广时复制head[]到cur[],在增广时同步修改cur[],目的是递归时跳过已增广的边。
bool BFS()
{
    memset(d, 0, sizeof(d));
    queue<int> q;
    q.push(S);
    d[S] = 1; //不为1  陷入死循环
    while (q.size())
    {
        int x = q.front();
        q.pop();
        for (int i = head[x]; i; i = nxt[i])
        {
            int y = ver[i];
            if (edge[i] && !d[y])
            {
                d[y] = d[x] + 1;
                q.push(y);
                if (y == T)
                    return true;
            }
        }
    }
    return false;
}
int Dinic(int x, int flow)
{
    if (x == T)
        return flow;
    int rest = flow, k;
    for (int i = head[x]; i && rest; i = nxt[i])
    {
        int y = ver[i];
        if (edge[i] && d[y] == d[x] + 1)
        {
            k = Dinic(y, min(edge[i], rest));
            if (!k)
                d[y] = 0;
            edge[i] -= k;
            edge[i ^ 1] += k;
            rest -= k;
        }
    }
    return flow - rest;
}
\end{lstlisting}

\subsubsection{二分图最大匹配的必须边与可行边}
在一般的二分图中,可以用最大流计算任一组最大匹配。

此时:
    必须边的判定条件为:(x,y) 流量为 1 ,并且在残量网络上属于不同的SCC。

    可行边的判定条件为:(x,y) 流量为 1 ,或者在残量网络上属于同一个SCC。

例题分析

CH17C 舞动的夜晚(Dinic,Tarjan,二分图可行边)

\subsection{最小割}
\subsubsection{最大流最小割定理}
任何一个网络的最大流量等于最小割中边的容量之和。

例题分析

POJ1966 Cable TV Network (枚举,点边转化)

\subsection{费用流}
\subsubsection{Edmonds Karp 增广路}
BFS寻找增广路 -> SPFA寻找单位费用之和最小的增广路(将费用作为边权,在残量网络上求最短路)。

注意:反向边的费用为相反数。
\begin{lstlisting}
bool SPFA()
{
    memset(dis, 0xcf, sizeof(dis));//-inf
    memset(vis, 0, sizeof(vis));
    queue<int> q;
    dis[S] = 0, vis[S] = 1, incf[S] = 1 << 30;
    q.push(S);
    while (q.size())
    {
        int x = q.front();
        q.pop();
        vis[x] = 0;
        for (int i = head[x]; i; i = nxt[i])
        {
            if (edge[i])
            {
                int y = ver[i];
                if (dis[y] < dis[x] + cost[i])
                {
                    dis[y] = dis[x] + cost[i];
                    incf[y] = min(incf[x], edge[i]);
                    pre[y] = i;
                    if (!vis[y])
                        q.push(y), vis[y] = 1;
                }
            }
        }
    }
    if (dis[T] == 0xcfcfcfcf)
        return false;
    return true;
}
int max_flow, ans;
void Update()
{
    int x = T;
    while (x != S)
    {
        int i = pre[x];
        edge[i] -= incf[T];
        edge[i ^ 1] += incf[T];
        x = ver[i ^ 1];
    }
    max_flow += incf[T];
    ans += incf[T] * dis[T];
}
\end{lstlisting}
例题分析

POJ3422 Kaka's Matrix Travels(点边转化,费用流)
    % %网络流初步
\end{document}