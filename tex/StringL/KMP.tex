\section{KMP}
    \subsection{前缀函数}
        给定一个长度为n的字符串s(假定下标从1开始),其\textbf{前缀函数}被定义为一个长度为n的数组$\pi$,其中$\pi[i]$为
        既是子串$s[1\dots i]$的前缀同时也是该子串的后缀的最长真前缀(proper prefix)长度。一个字符串的真前缀是其前缀但
        不等于该字符串自身。根据定义,$\pi[1]=0$。\\\\
        前缀函数的定义可用数学语言描述如下:
        $$\pi[i]=\max _{k=0 \ldots i-1}\{k : s[1 \ldots k]=s[i-k+1 \ldots i]\}$$
        举例来说,字符串abcabcd的前缀函数为$[0,0,0,1,2,3,0]$,字符串aabaaab的前缀函数为$[0,1,0,1,2,2,3]$。
        \subsubsection{朴素算法}
            直接按定义计算前缀函数:
            \begin{lstlisting}
// 朴素法求前缀函数O(n^3),下标从1开始
void prefix_func0(char t[], int n, int pi[]) {
    int i, k;
    for (i = 1; i <= n; i++)     // 对每一个子串
        for (k = 0; k < i; k++)  // 枚举前缀后缀长度,并判断是否相等
            if (!strncmp(t + 1, t + i - k + 1, k))
                pi[i] = k;
}
            \end{lstlisting}
        \subsubsection{第一个优化}
            第一个重要的事实是相邻的前缀函数值至多增加1。(如不然,会产生矛盾)\\
            所以当移动到下一个位置时,前缀函数要么增加1,要么不变或减少。
            实际上,该事实已经允许我们将复杂度降至$O(n^2)$。
            因为每一步中前缀函数至多增加1,因此在总的运行过程中,前缀函数至多增加n,同时也至多减小n。
            这意味着我们仅需进行$O(n)$次字符串比较,所以总复杂度为$O(n^2)$。
            \begin{lstlisting}
void prefix_func1(char t[], int n, int pi[]) {
    int i, j;
    i = 2, j = 1;
    while (i <= n) {
        if (t[i] == t[j])  // 加1
            pi[i] = pi[i - 1] + 1, ++j;
        else {  // 开始减
            pi[i] = pi[i - 1];
            while (strncmp(t + i - pi[i] + 1, t + 1, pi[i]))
                --pi[i];
            j = pi[i] + 1;
        }
        ++i;
    }
}
            \end{lstlisting}
        \subsubsection{第二个优化}
            考虑计算位置i+1的前缀函数$\pi$的值,如果$s[i+1]=s[\pi[i]+1]$,显然$\pi[i+1]=\pi[i]+1$。
            $$ \underbrace{\overbrace{s_1 ~ s_2 ~ s_3}^{\pi[i]} ~ \overbrace{s_4}^{s_4 = s_{i+1}}}_{\pi[i+1] = \pi[i] + 1} ~ \dots ~ 
            \underbrace{\overbrace{s_{i-2} ~ s_{i-1} ~ s_{i}}^{\pi[i]} ~ \overbrace{s_{i+1}}^{s_4 = s_{i+1}}}_{\pi[i+1] = \pi[i] + 1} $$
            如果不是上述情况,即$s[i+1] \neq s[\pi[i]+1]$,我们需要尝试更短的字符串。为了加速,我们希望直接移动到
            最长的长度$j<\pi[i]$,使得在位置i的前缀性质仍得以保持,也即$s[1 \dots j] = s[i-j+1 \dots i]$:
            $$\overbrace{\underbrace{s_1 ~ s_2}_j ~ s_3 ~ s_4}^{\pi[i]} ~ \dots ~ \overbrace{s_{i-3} ~ s_{i-2} ~ \underbrace{s_{i-1} ~ s_{i}}_j}^{\pi[i]} ~ s_{i+1}$$
            实际上,如果我们找到了这样的j,我们仅需要再次比较$s[i+1]$和$s[j+1]$。如果它们相等,则$\pi[i+1]=j+1$,
            否则,我们就需要找小于j的最大的新的j使得前缀性质仍然保持,如此反复,直到$s[i+1]=s[j+1]$或者确实完全找不到(令$j=-1$)。
            最后$\pi[i+1]=j+1$。\\\\
            所以我们已经有了一个大致框架,现在仅剩的问题是对于满足$s[1 \dots j] = s[i-j+1 \dots i]$的j,
            如何快速找到小于j的最大的新的j,我们令新的j为k,使得$s[1 \dots k] = s[i-k+1 \dots i]$仍然满足。
            $$\overbrace{\underbrace{s_1 ~ s_2}_k ~ s_3 ~ s_4}^j ~ \dots ~ \overbrace{s_{i-3} ~ s_{i-2} ~ \underbrace{s_{i-1} ~ s_{i}}_k}^j ~s_{i+1}$$
            由上图,我们要求的是比j小的最大的k,而两边长度为j的前后缀本身是相等的,那么新的长为k的前后缀则可以只放到最左边
            长为j的前缀中去考虑:
            $$\overbrace{\underbrace{s_1 ~ s_2}_k ~ \underbrace{s_3 ~ s_4}_k}^j$$
            即$k=\pi[j]$,而$\pi[j]$之前已经求过了。
            \begin{lstlisting}
void prefix_func2(char t[], int n, int pi[]) {
    int i, j;
    pi[0] = -1, pi[1] = 0;  // 确实没有找到任何相等的
    for (i = 1; i < n; ++i) {
        j = pi[i];
        while (j >= 0 && t[j + 1] != t[i + 1])  // 若不相等,找更小的新的j
            j = pi[j];
        pi[i + 1] = j + 1;  //最后得出pi[i+1]
    }
}

            \end{lstlisting}
    \subsection{在线KMP}
        最基础的字符串匹配。下面的算法不是在线的,但只要稍作修改就可以变成在线的了。
        \begin{lstlisting}
#include <bits/stdc++.h>

using namespace std;
const int N = 1e3 + 10;

char s[N], t[N];  // 与OIWIKI上不一样,这里s是待匹配串,t是模式串
int n, m;         // s和t的长度
int pi[N];        // 前缀函数
vector<int> ans;  // 所有出现位置

inline void prefix_func(char t[], int n, int pi[]) {
    int i, j;
    pi[0] = -1, pi[1] = 0;
    for (i = 1; i < n; i++) {
        j = pi[i];
        while (j >= 0 && t[i + 1] != t[j + 1])
            j = pi[j];
        pi[i + 1] = j + 1;
    }
}

int main() {
    ios::sync_with_stdio(0);
    cin.tie(0);

    int i, j, last;

    cin >> (s + 1) >> (t + 1);
    n = strlen(s + 1), m = strlen(t + 1);

    prefix_func(t, m, pi);  // 求模式串的前缀函数

    // 用类似求前缀函数的方法匹配s和t
    last = 0;
    for (i = 0; i < n; i++) {
        j = last;
        while (j >= 0 && s[i + 1] != t[j + 1])
            j = pi[j];
        last = j + 1;
        if (last == m)
            cout << i + 2 - m << " ";
    }
    cout << endl;

    return 0;
}
        \end{lstlisting}
    \subsection{统计每个前缀出现次数}
        \subsubsection{单串统计}
            统计s的每个前缀在s中出现的次数。首先我们明确,一个长度为i的前缀中会出现长度为$\pi[i]$的前缀,然后长度为$\pi[i]$
            的前缀又会出现长度为$\pi[\pi[i]]$的前缀,等等。所以我们考虑后缀和的思想,首先统计每一个位置的$\pi[i]$,
            然后将$ans[i]$累加给长为$ans[\pi[i]]$的前缀个数,按照长度递减的顺序依次累加下去就可以了。
            最后再统计原始前缀,即对每个$ans[i]$加一。
            \begin{lstlisting}
// 统计s[]的每个前缀在s[]中出现的次数
inline void count_prefix(char s[], int n, int pi[]) {
    prefix_func(s, n, pi);  // 计算前缀函数
    int i;
    for (i = 1; i <= n; i++)
        ++ans[pi[i]];
    // 令真前后缀长度为i,其个数为ans[i],则长为i的真前后缀的真前后缀长为pi[i],其原本个数为ans[pi[i]]
    // 现在需要累加上它在更长的真前后缀中出现的次数。有点类似倍增
    for (i = n; i >= 1; i--)
        ans[pi[i]] += ans[i];
    // 这里不能放到开头,否则,一开始ans[n]=1,也就认为s有一个长为n的真前后缀
    for (i = 1; i <= n; i++)
        ++ans[i];
}
            \end{lstlisting}
        \subsubsection{双串统计}
            给出串s和t,问t的每个前缀在s中出现的次数。首先运用类似KMP的思想,通过'\#'连接t和s,即"t\#s",设为link,
            对link求前缀函数。接下来我们只关心与s有关的前缀函数值,即$i \ge m+2$的$\pi[i]$。
            \begin{lstlisting}
inline void count_prefix(char s[], int n, char t[], int m, int pi[]) {
    // 连接字符串 t..#s..
    strcpy(link + 1, t + 1);
    link[m + 1] = '#';
    strcpy(link + m + 2, s + 1);
    // 计算前缀函数
    prefix_func(link, n + m + 1, pi);

    int i;
    // 只关心#后面的pi值
    for (i = m + 2; i <= n + m + 1; i++)
        ++ans[pi[i]];
    // pi[i]的值不会超过t的长度,即i<=m
    for (i = m; i >= 1; i--)
        ans[pi[i]] += ans[i];
}
            \end{lstlisting}
    \subsection{统计一个字符串本质不同的子串的数目}
        给定一个长度为n的字符串s,我们希望计算其本质不同子串的数目。\\
        假设现在知道了当前s本质不同的子串的数目,那么接下来可以考虑在原来的s末尾加上一个字符c,然后统计产生了多少新的子串。\\
        我们枚举$i$,对每一个$i$,反转$s[1 \dots i]$令其为t,然后对t求前缀函数,$\pi_{max}$即为以$s[i]$结尾的重复子串的数目,
        那么$|s|-\pi_{max}$即为以$s[i]$结尾的新的子串的数目。
        \begin{lstlisting}
// 统计串s中本质不同的子串数目
inline int diff(char s[], int n) {
    int i, ret = 0;
    for (i = 1; i <= n; i++) {
        reverse_copy(s + 1, s + 1 + i, t + 1);  // 翻转s并存入t
        int mx = prefix_func2(t, i, pi);        // 略微修改一下prefix_func,使其可以返回最大的pi值
        ret += i - mx;                          // 统计新的子串数目
    }
    return ret;
}
        \end{lstlisting}
    \subsection{字符串压缩}
        给定一个长度为n的字符串s,我们希望找到其最短的“压缩”表示,也即我们希望寻找一个最短的字符串t,使得s可以被t的一份或多份拷贝的拼接表示。\\
        显然,我们只需要找到t的长度即可。知道了长度,该问题的答案即为长度为该值的s的前缀。\\
        直接说结论,首先求s的前缀函数,令$k=n-\pi[n]$,若$n \bmod k = 0$,则该长度为$k$,否则为$n$。
        \begin{lstlisting}
// 计算s的最短压缩表示,返回最短压缩的长度
int compress(char s[], int n, int pi[]) {
    prefix_func(s, n, pi);
    int k = n - pi[n];
    if (n % k)
        return n;
    return k;
}
        \end{lstlisting}
    \subsection{根据前缀函数构建一个自动机}
        让我们重新回到通过一个分割符将两个字符串拼接的新字符串。设模式串为t,
        待匹配串为s,则拼接成$t+\#+s$。前面我们就知道,我们只需要管$t+\#$的前缀函数值就可以,
        所以在这里我们的自动机也是如此。\\
        自动机的状态为当前前缀函数的值,而下一个读入自动机的字符则决定状态如何转移。
        下面我们就是要求状态转移表$aut$,$aut[i][c]$表示当前前缀函数值为$i$(也
        即当前处于$t[i]$处)下一个字符为$c$的转移的目标状态。
        \begin{lstlisting}
// 计算长度为n的字符串t[]的自动机的转移表,t[]下标从1开始
void compute_automaton0(char t[], int n, int pi[], int aut[][26]) {
    prefix_func(t, n, pi);  // 先求t[]的前缀函数
    int i, j, c;
    // 0是初始状态,匹配长度为0;n是终止状态,完全匹配
    for (i = 0; i <= n; i++) {
        // 转移的因素--下一个字符
        for (c = 0; c < 26; c++) {
            j = i;
            // 通过不断跳转前缀来匹配下一个字符
            while (j >= 0 && c + 'a' != t[j + 1])
                j = pi[j];
            aut[i][c] = j + 1;
        }
    }
}
        \end{lstlisting}
        在上面的代码中,由于有while循环的存在,所以时间复杂度为$O(|\Sigma|n^2)$。\\
        事实上,我们可以通过动态规划来优化。我们注意到当下一个字符$c \neq s[j+1]$时,
        $j$会跳转至$\pi[j]$,而在之前我们已经计算过所有$aut[\pi[j]][c],\forall c \in \Sigma$,
        所以我们可以直接利用$aut[\pi[i]][c]$。时间复杂度$O(|\Sigma|n)$。
        \begin{lstlisting}
// 计算长度为n的字符串t[]的自动机的转移表,t[]下标从1开始
void compute_automaton1(char t[], int n, int pi[], int aut[][26]) {
    prefix_func(t, n, pi);  // 先求t[]的前缀函数
    int i, c;
    // 0是初始状态,匹配长度为0;n是终止状态,完全匹配
    aut[0][t[1] - 'a'] = 1;
    for (i = 1; i <= n; i++) {
        // 转移的因素--下一个字符
        for (c = 0; c < 26; c++) {
            // 利用动态规划的思想来优化,在上面j跳转至pi[j]时,aut[pi[j]][c]对于所有的c都已经被计算过了
            if (c + 'a' == t[i + 1])
                aut[i][c] = i + 1;
            else
                aut[i][c] = aut[pi[i]][c];
        }
    }
}
        \end{lstlisting}
    \subsection{Gray字符串}
        首先定义Gray字符串,令$g[0]=""$(空串),$g[1]="1"$,之后$g[i]=g[i-1]+i+g[i-1]$,加号为字符串拼接。\\
        接下来我们考虑这样一个问题(与OIWIKI不同的是数据范围作了修改):给定长为$n(n \le 1000)$的字符串$s(1 \le s[i] \le n)$,
        一个整数$k(k \le 1000)$,要求s在g[k]中出现的次数。\\\\
        这题是kmp自动机+dp。\\
        显然,g[k]的长度非常大,我们不可能去构造。但我们可以好好利用Gray字符串递归的性质,即$g[i]=g[i-1]+i+g[i-1]$。\\
        对s构建一个\textbf{自动机aut[i][j]},表示从当前状态i通过输入的j到达的状态。\\
        假设当前自动机处于状态i,接下来要处理g[j],我们可以分为3步:\\
        1.从状态i开始处理g[j-1],自动机到达状态t1\\
        2.从状态t1开始处理j,自动机到达状态t2\\
        3.从状态t2开始处理g[j-1],自动机到达状态t3。\\
        其中t3即为目标状态。\\\\
        显然,如果每一步都老老实实做的话,工作量并未减少。仔细观察发现我们可以建立一个dp状态,\textbf{G[i][j]}表示自动机从状态i开始处理g[j],处理完成后自动机所处的状态。
        那么上面的3步就可以变为如下形式:
        $$t1 = G[i][j-1]$$
        $$t2 = aut[t1][j]$$
        $$t3 = G[t2][j-1]$$
        由于$i \le n$且$j \le k$复杂度$O(nk)$。\\
        如何计算答案呢?我们只要在自动机转移的过程中看当前状态是否为$|s|$状态(和s完全匹配)即可。但是我们上面都是一跳一大步,可能有的$|s|$状态就给跳过去了。
        所以我们再用一个\textbf{K[i][j]}记录自动机从状态i开始处理g[j]直到处理完成,这个过程中几次达到状态$|s|$。显然有:
        $$K[i][j] = K[i][j-1] + (t2 == |s|) + K[t2][j-1]$$
        初始时,aut[i][j]可全部求出,G[i][0]=i,K[i][0]=0,因为g[0]为空串,自动机不会转移,g[0]也不会包含s。\\
        显然,最后答案为K[0][k]。
        \begin{lstlisting}
#include <bits/stdc++.h>

using namespace std;
const int N = 1e3 + 10;

int s[N];
int n, k;
int pi[N];
int aut[N][N];  // 自动机,aut[i][j]表示当前状态通过输入j得到的下一个状态
int G[N][N];    // G(i,j)表示自动机从状态i开始处理g[j],处理完成后自动机的状态
int K[N][N];    // K(i,j)表示自动机从状态i开始处理g[j],处理完成后s在g[j]中出现的次数

void prefix_func(int t[], int n, int pi[]) {
    int i, j;
    pi[0] = -1, pi[1] = 0;  // 确实没有找到任何相等的
    for (i = 1; i < n; ++i) {
        j = pi[i];
        while (j >= 0 && t[j + 1] != t[i + 1])  // 若不相等,找更小的新的j
            j = pi[j];
        pi[i + 1] = j + 1;  //最后得出pi[i+1]
    }
}

// 计算长度为n的字符串t[]的自动机的转移表,t[]下标从1开始
void compute_automaton1(int t[], int n, int pi[], int aut[][N]) {
    t[n + 1] = -1;          // 结束符,结束符应是字母表中没有的符号
    prefix_func(t, n, pi);  // 先求t[]的前缀函数
    int i, c;
    // 0是初始状态,匹配长度为0;n是终止状态,完全匹配
    aut[0][t[1]] = 1;
    for (i = 1; i <= n; i++) {
        // 转移的因素--下一个字符
        for (c = 1; c <= n; c++) {
            // 利用动态规划的思想来优化,在上面j跳转至pi[j]时,aut[pi[j]][c]对于所有的c都已经被计算过了
            if (c == t[i + 1])
                aut[i][c] = i + 1;
            else
                aut[i][c] = aut[pi[i]][c];
        }
    }
}

int main() {
    ios::sync_with_stdio(0);
    cin.tie(0);

    int i, j;
    cin >> n >> k;
    for (i = 1; i <= n; i++)
        cin >> s[i];

    compute_automaton1(s, n, pi, aut);

    // 初始化,从状态i开始处理g[0],由于g[0]是空串,所以自动机不会转移,s也不会出现在g[0]中
    for (i = 0; i <= n; i++)
        G[i][0] = i, K[i][0] = 0;
    // 类似于动态规划的递推
    for (j = 1; j <= k; j++) {
        for (i = 0; i <= n; i++) {
            int mid = aut[G[i][j - 1]][j];
            G[i][j] = G[mid][j - 1];
            K[i][j] = K[i][j - 1] + (n == mid) + K[mid][j - 1];
        }
    }

    cout << K[0][k] << endl;

    return 0;
}
        \end{lstlisting}
    \subsection{UVA11022 String Factoring}
        给定一个字符串$s(|s| \le 80)$,问最小压缩长度。例如,$AAA=(A)^3$最小长度为1,$CABAB=C(AB)^2$最小长度为3,
        再如$POPPOP$既可以是$PO(P)^2OP$也可以是$(POP)^2$,但后者长度为3,前者为5,所以后者更优。\\\\
        这题是kmp字符串压缩+区间dp。\\
        f[i][j]表示s[i..j]被压缩后的最短长度,pi2[i][j]表示以i为左端点,j处的前缀函数值。先考虑最简单的区间dp:
        $$f(l,r)=\min_{l \le k < r}{\{f(l,m)+f(m+1,r)\}}$$
        可以发现,这是\textbf{不完整}的,例如ATTATT,假设已知$f(1,3)=2$和$f(4,6)=2$,接下来算$f(1,6)$就会等于4,而正确答案是2。
        也就是说,我们没有考虑s[l..r]可能本身就可以被压缩。若s[l..r]本身可以被压缩,则一定存在一个最小循环节,这我们可以通过上面的pi2[l][r]来求。\\
        若确实存在循环节,设其长度为k,则$f(l,r)=f(l,l+k-1)$,之所以这样,是因为循环节本身也是可以被压缩的(类似子问题)。之后再用上面的区间dp,就是对的了。
        \begin{lstlisting}
#include <bits/stdc++.h>

using namespace std;
const int N = 85;

char s[N];
int n;
int pi[N], pi2[N][N];  // pi2[i][j]表示以i为左端点,j处的前缀函数值
int f[N][N];           // f[i][j]表示s[i..j]被压缩后的最短长度

void prefix_func(char t[], int n, int pi[]) {
    int i, j;
    pi[0] = -1, pi[1] = 0;  // 确实没有找到任何相等的
    for (i = 1; i < n; ++i) {
        j = pi[i];
        while (j >= 0 && t[j + 1] != t[i + 1])  // 若不相等,找更小的新的j
            j = pi[j];
        pi[i + 1] = j + 1;  //最后得出pi[i+1]
    }
}

int main() {
    ios::sync_with_stdio(0);
    cin.tie(0);

    int i, j, k, l, r, m;

    while (cin >> (s + 1)) {
        if (s[1] == '*')
            break;
        n = strlen(s + 1);

        // 求pi2[i][j]
        for (i = 1; i <= n; i++)
            prefix_func(s + i - 1, n - i + 1, pi2[i] + i - 1);

        // dp求f(l,r),s[l..r]被压缩后的最短长度
        for (i = 1; i <= n; i++) {              // 枚举长度
            for (l = 1; l + i - 1 <= n; l++) {  // 枚举左端点
                r = l + i - 1;                  // 右端点
                f[l][r] = i;                    // 初始化为最坏情况
                int len = r - l + 1;            // 当前区间的长度
                // 如果本身完全是循环节
                k = len - pi2[l][r];
                if (len % k == 0)
                    f[l][r] = min(f[l][r], f[l][l + k - 1]);  // 循环节的性质,转化为已求的子问题
                // 区间DP
                for (m = l; m < r; m++)
                    f[l][r] = min(f[l][r], f[l][m] + f[m + 1][r]);
            }
        }

        cout << f[1][n] << endl;
    }

    return 0;
}
        \end{lstlisting}