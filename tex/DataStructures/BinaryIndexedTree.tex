\section{树状数组}
\emph{推荐阅读:https://www.cnblogs.com/RabbitHu/p/BIT.html}
\subsection{单点修改,区间查询}
\begin{lstlisting}
#define N 1000100
long long c[N];
int n,q;
int lowbit(int x)
{
    return x&(-x);
}
void change(int x,int v)
{
    while(x<=n)
    {
        c[x]+=v;
        x+=lowbit(x);
    }
}
long long getsum(int x)
{
    long long ans=0;
    while(x>=1)
    {
        ans+=c[x];
        x-=lowbit(x);
    }
    return ans;
}
\end{lstlisting}
\emph{例题:https://loj.ac/problem/130}

\subsection{区间修改,单点查询}
引入差分数组来解决树状数组的区间更新
\begin{lstlisting}
//初始化
change(i,cur-pre);
//区间修改
change(l,x);
change(r+1,-x);
//单点查询
getsum(x)
\end{lstlisting}
\emph{例题:https://loj.ac/problem/131}

\subsection{区间修改,区间查询}
\begin{lstlisting}
//初始化
change(c1,i,cur-pre);
change(c2,i,i*(cur-pre));
//为什么这么写? 你需要写一下前缀和的表达式
//区间修改
change(c1,l,x);
change(c2,l,l*x);
change(c1,r+1,-x);
change(c2,r+1,-(r+1)*x);
//区间查询
temp1=l*getsum(c1,l-1)-getsum(c2,l-1);
temp2=(r+1)*getsum(c1,r)-getsum(c2,r);
ans=temp2-temp1
\end{lstlisting}
\emph{例题:https://loj.ac/problem/132}

\section{二维树状数组}
\subsection{单点修改,区间查询}
\begin{lstlisting}
#define N 5050
long long tree[N][N];
long long n,m;
long long lowbit(long long x)
{
    return x&(-x);
}
void change(long long x,long long y,long long val)
{
    long long init_y=y;
    //这里注意n,m的限制
    while(x<=n)
    {
        y=init_y;
        while(y<=m)
        {
            tree[x][y]+=val;
            y+=lowbit(y);
        }
        x+=lowbit(x);
    }
}
long long getsum(long long x,long long y)
{
    long long ans=0;
    long long init_y=y;
    while(x>=1)
    {
        y=init_y;
        while(y>=1)
        {
            ans+=tree[x][y];
            y-=lowbit(y);
        }
        x-=lowbit(x);
    }
    //这里画图理解
    return ans;
}
//初始化
change(x,y,k);
//二维前缀和
ans = getsum(c,d)+getsum(a-1,b-1)-getsum(a-1,d)-getsum(c,b-1);
\end{lstlisting}
\emph{例题:https://loj.ac/problem/133}

\subsection{区间修改,区间查询}
\begin{lstlisting}
#define N 2050
long long t1[N][N];
long long t2[N][N];
long long t3[N][N];
long long t4[N][N];
long long n,m;
long long lowbit(long long x)
{
    return x&(-x);
}
long long getsum(long long x,long long y)
{
    long long ans=0;
    long long init_y=y;
    long long init_x=x;
    while(x>=1)
    {
        y=init_y;
        while(y>=1)
        {
            ans+=(init_x+1)*(init_y+1)*t1[x][y];
            ans-=(init_y+1)*t2[x][y];
            ans-=(init_x+1)*t3[x][y];
            ans+=t4[x][y];
            y-=lowbit(y);
        }
        x-=lowbit(x);
    }
    return ans;
}
void change(long long x,long long y,long long val)
{
    long long init_x=x;
    long long init_y=y;
    while(x<=n)
    {
        y=init_y;
        while(y<=m)
        {
            t1[x][y]+=val;
            t2[x][y]+=init_x*val;
            t3[x][y]+=init_y*val;
            t4[x][y]+=init_x*init_y*val;
            y+=lowbit(y);
        }
        x+=lowbit(x);
    }
}
//区间修改
change(c+1,d+1,x);
change(a,b,x);
change(a,d+1,-x);
change(c+1,b,-x);
//区间查询
ans=getsum(c,d)+getsum(a-1,b-1)-getsum(c,b-1)-getsum(a-1,d);
\end{lstlisting}
\emph{例题:https://loj.ac/problem/135}
