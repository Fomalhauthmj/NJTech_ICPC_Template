\section{带Lazy标记的线段树}
    以下是区间修改+区间最大值查询。\\
    若是区间修改+\textbf{区间和查询},则在pushdown时需要将lazy标记\textbf{乘上区间长度}加到结点上。
\begin{lstlisting}
#include <bits/stdc++.h>

using namespace std;
const int N = 1e5 + 10;

int n;  // 长度

struct Node {
    int l, r, m;
    int mx;   // [l,r)中的最大值
    int tag;  // lazy标记
} t[4 * N];   // 数组大小不要忘记 * 4

// 将lazy标记下推
inline void pushdown(int x) {
    Node& cur = t[x];
    if (cur.r - cur.l == 1)
        return;
    Node &lch = t[x * 2], &rch = t[x * 2 + 1];
    lch.mx += cur.tag, rch.mx += cur.tag;
    lch.tag += cur.tag, rch.tag += cur.tag;
    cur.tag = 0;
}

// 由x的儿子更新x结点,此时应确保x的儿子为最新
inline void pushup(int x) {
    Node& cur = t[x];
    if (cur.r - cur.l == 1)
        return;
    Node &lch = t[x * 2], &rch = t[x * 2 + 1];
    cur.mx = max(lch.mx, rch.mx);
}

// 建树。注意初始时叶子是否为0,若不是,需要pushup
void build(int l, int r, int x) {
    Node& cur = t[x];
    cur.l = l, cur.r = r, cur.m = (l + r) / 2;
    cur.mx = 0, cur.tag = 0;  // 初始化最大值、lazy标记
    if (r - l == 1)
        return;
    build(l, cur.m, x * 2);
    build(cur.m, r, x * 2 + 1);
    pushup(x);  // 若初始值非0,则这句一定要加
}

// [l,r)每个元素加v。注意打标记时应更新被打标记结点的mx,并且应立即pushdown,pushup,过程中也需要pushdown,pushup
void update(int l, int r, int x, int v) {
    Node& cur = t[x];
    if (cur.l == l && cur.r == r) {
        cur.tag += v, cur.mx += v;  // 打标记的时候一定是同时更新max的
        pushdown(x), pushup(x);
        return;
    }
    pushdown(x);
    if (r <= cur.m)
        update(l, r, x * 2, v);
    else if (l >= cur.m)
        update(l, r, x * 2 + 1, v);
    else
        update(l, cur.m, x * 2, v), update(cur.m, r, x * 2 + 1, v);
    pushup(x);
}

// 查询[l,r)的最大值。过程中注意pushdown
int query(int l, int r, int x) {
    Node& cur = t[x];
    pushdown(x);
    if (cur.l == l && cur.r == r)
        return cur.mx;
    int mx = 0;
    if (r <= cur.m)
        mx = query(l, r, x * 2);
    else if (l >= cur.m)
        mx = query(l, r, x * 2 + 1);
    else
        mx = max(query(l, cur.m, x * 2), query(cur.m, r, x * 2 + 1));
    return mx;
}

int main() {
    ios::sync_with_stdio(0);
    cin.tie(0);

    int m, i, l, r, v;
    cin >> n >> m;
    build(1, n + 1, 1);  // 不要忘记build初始化,[l,r+1)
    for (i = 1; i <= m; i++) {
        cin >> v >> l >> r;
        if (!v)
            cout << query(l, r + 1, 1) << endl;
        else
            update(l, r + 1, 1, 1);
    }

    return 0;
}
\end{lstlisting}