\section{线段树}
\begin{lstlisting}
struct SegmentTree
{
    int l, r;
    ll sum, add;
    #define ls(x) (x << 1)
    #define rs(x) (x << 1 | 1)
    #define l(x) tree[x].l
    #define r(x) tree[x].r
    #define sum(x) tree[x].sum
    #define add(x) tree[x].add
} tree[N << 2];
void PushUp(int rt)
{
    sum(rt) = (sum(ls(rt)) + sum(rs(rt))) % mod;
}
void PushDown(int rt)
{
    if (add(rt))
    {
        sum(ls(rt)) = (sum(ls(rt)) + add(rt) * (r(ls(rt)) - l(ls(rt)) + 1)) % mod;
        sum(rs(rt)) = (sum(rs(rt)) + add(rt) * (r(rs(rt)) - l(rs(rt)) + 1)) % mod;
        add(ls(rt)) = (add(ls(rt)) + add(rt)) % mod;
        add(rs(rt)) = (add(rs(rt)) + add(rt)) % mod;
        add(rt) = 0;
    }
}
void SegmentTree_Build(int rt, int l, int r)
{
    l(rt) = l, r(rt) = r;
    if (l == r)
    {
        sum(rt) = weight[l];
        return;
    }
    int mid = (l + r) >> 1;
    SegmentTree_Build(ls(rt), l, mid);
    SegmentTree_Build(rs(rt), mid + 1, r);
    PushUp(rt);
    //向上更新
}
void Update(int rt, int l, int r, int d)
{
    if (l <= l(rt) && r(rt) <= r)
    {
        sum(rt) = (sum(rt) + (r(rt) - l(rt) + 1) * d) % mod;
        add(rt) = (add(rt) + d) % mod;
        return;
    }
    PushDown(rt);
    int mid = (l(rt) + r(rt)) >> 1;
    if (l <= mid)
        Update(ls(rt), l, r, d);
    if (r > mid)
        Update(rs(rt), l, r, d);
    PushUp(rt);
}
ll Query(int rt, int l, int r)
{
    if (l(rt) >= l && r(rt) <= r)
    {
        return sum(rt) % mod;
    }
    PushDown(rt);
    ll ret = 0;
    int mid = (l(rt) + r(rt)) >> 1;
    if (l <= mid)
        ret = (ret + Query(ls(rt), l, r)) % mod;
    if (r > mid)
        ret = (ret + Query(rs(rt), l, r)) % mod;
    return ret;
}    
\end{lstlisting}
