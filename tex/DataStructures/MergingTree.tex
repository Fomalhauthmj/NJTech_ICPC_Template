\section{归并树}
\subsection{简介}
    归并排序,自底向上越来越有序。而归并树则是将归并排序的中间结果保留了下来。\\
    为什么要这么做呢?因为这样当询问区间时,总是可以将询问区间二分成某一次或几次归并排序的中间结果。
\subsection{区间第k小值}
    \subsubsection{问题简述}
        给定n个数,m次查询,每次查询[l,r]内从小到大第k个数,输出这个数。
    \subsubsection{解决方案}
        对这n个数先排序,然后二分尝试,将二分得到的中间值代入归并树中,看是否排名为k,\textbf{对所有排名小于等于k的取最大值}。\\
        建树时间复杂度O(NlogN),查询时间复杂度O(logNlogN)。
\begin{lstlisting}
#include <bits/stdc++.h>

using namespace std;
typedef long long ll;
const int N = 1e5 + 10, INF = 1e9 + 8;

int a[N], n, m;

struct Node {
    int l, r, m;
    vector<int> a;
} t[N * 3];

// 由a[]数组初始化归并树
void build(int l, int r, int x) {
    Node& cur = t[x];
    cur.l = l, cur.r = r, cur.m = (l + r) / 2;
    cur.a.clear();
    if (r - l == 1) {
        cur.a.push_back(a[cur.l]);
        return;
    }
    build(l, cur.m, x * 2);
    build(cur.m, r, x * 2 + 1);
    Node &lch = t[x * 2], &rch = t[x * 2 + 1];
    vector<int>::iterator it1, it2;
    it1 = lch.a.begin(), it2 = rch.a.begin();
    // 合并左右子树
    while (it1 != lch.a.end() && it2 != rch.a.end()) {
        if (*it1 <= *it2)
            cur.a.push_back(*it1), it1++;
        else
            cur.a.push_back(*it2), it2++;
    }
    while (it1 != lch.a.end())
        cur.a.push_back(*it1), it1++;
    while (it2 != rch.a.end())
        cur.a.push_back(*it2), it2++;
}

// 查询在区间[l,r)内比v小的数的个数,稍加修改便可查询大于等于k的最小值
int query(int l, int r, int v, int x) {
    Node& cur = t[x];
    if (cur.l == l && cur.r == r)
        return lower_bound(cur.a.begin(), cur.a.end(), v) - cur.a.begin();
    if (r <= cur.m)
        return query(l, r, v, x * 2);
    else if (l >= cur.m)
        return query(l, r, v, x * 2 + 1);
    else
        return query(l, cur.m, v, x * 2) + query(cur.m, r, v, x * 2 + 1);
}

// 查询区间[l,r)区间内的从小到大第k个值
int QR(int ql, int qr, int k) {
    int l = 1, r = n + 1, m, ans = -INF;
    while (l < r) {
        m = (l + r) / 2;
        int rank = query(ql, qr, a[m], 1) + 1;
        if (rank <= k)  // 可能有相等的值,所以需要<=
            ans = max(ans, a[m]);
        if (rank <= k)
            l = m + 1;
        else
            r = m;
    }
    return ans;
}

int main() {
    int i, l, r, k;
    while (scanf("%d%d", &n, &m) != EOF) {
        for (i = 1; i <= n; i++)
            scanf("%d", &a[i]);
        build(1, n + 1, 1);
        sort(a + 1, a + 1 + n);
        for (i = 1; i <= m; i++) {
            scanf("%d%d%d", &l, &r, &k);
            printf("%d\n", QR(l, r + 1, k));
        }
    }

    return 0;
}
\end{lstlisting}
    \subsubsection{思考}
        如何取区间内不重复的第k大?\\
        结点中用两个vector,其中一个是原来的不变,另一个用于存放不重复的数,查询的时候在后者中查。

\subsection{区间内大于等于v的最小值}
    \subsubsection{问题简述}
        给定n个数,m次查询,每次查询[l,r]内比v大的最小值,输出这个最小值。
    \subsubsection{解决方案}
        归并树天生适合的问题。对于对应区间,直接返回lower\_bound找到的第一个数注意,如果没找到返回一个无限大,
        对于其余祖先节点,返回左右子树中找到的最小值。
\begin{lstlisting}
int query(int l, int r, int v, int x) {
    Node& cur = t[x];
    if (cur.l == l && cur.r == r) {
        vector<int>::iterator it;
        it = lower_bound(cur.a.begin(), cur.a.end(), v);
        if (it == cur.a.end())
            return n + 1;
        return *it;
    }
    if (r <= cur.m)
        return query(l, r, v, x * 2);
    else if (l >= cur.m)
        return query(l, r, v, x * 2 + 1);
    else
        return min(query(l, cur.m, v, x * 2), query(cur.m, r, v, x * 2 + 1));
}
\end{lstlisting}