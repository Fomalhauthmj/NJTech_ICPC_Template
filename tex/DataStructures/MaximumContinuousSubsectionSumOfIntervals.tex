\section{区间最大连续子段和}
\begin{lstlisting}
#include <bits/stdc++.h>

using namespace std;
const int N = 5e4 + 10;  // 数组大小,记得改

struct Node {
    int l, r, m;
    int s, f, fl, fr;  // 区间的和,区间最大子段和,包含左端点的最大子段和,包含右端点的最大子段和
} s[N * 4];

// 构建线段树
void build(int l, int r, int i) {
    Node& fa = s[i];
    fa.l = l, fa.r = r, fa.m = (l + r) / 2;
    fa.s = fa.f = fa.fl = fa.fr = 0;
    if (r - l == 1)
        return;
    build(l, fa.m, i * 2);
    build(fa.m, r, i * 2 + 1);
}

// 自底向上更新
void pushup(int i) {
    Node &fa = s[i], &lson = s[i * 2], &rson = s[i * 2 + 1];  // 父亲 左儿子 右儿子
    fa.s = lson.s + rson.s;
    fa.f = max(max(lson.f, rson.f), lson.fr + rson.fl);
    fa.fl = max(lson.fl, lson.s + rson.fl);
    fa.fr = max(rson.fr, rson.s + lson.fr);
}

// 单点更新
void update(int x, int p, int i) {
    Node& fa = s[i];
    if (fa.l == p && fa.r - fa.l == 1) {
        fa.s = fa.f = fa.fl = fa.fr = x;
        return;
    }
    if (p < fa.m)
        update(x, p, i * 2);
    else
        update(x, p, i * 2 + 1);
    // 向上更新
    pushup(i);
}

// 作为查询的返回值
struct Ret {
    int f, s, fl, fr;
};

// 查询
Ret query(int l, int r, int i) {
    Node& fa = s[i];
    if (fa.l == l && fa.r == r)
        return {fa.f, fa.s, fa.fl, fa.fr};
    if (r <= fa.m)
        return query(l, r, i * 2);
    else if (l >= fa.m)
        return query(l, r, i * 2 + 1);
    else {
        Ret lret = query(l, fa.m, i * 2);
        Ret rret = query(fa.m, r, i * 2 + 1);
        return {max(max(lret.f, rret.f), lret.fr + rret.fl), lret.s + rret.s, max(lret.fl, lret.s + rret.fl), max(rret.fr, rret.s + lret.fr)};
    }
}

int main() {
    ios::sync_with_stdio(0);
    cin.tie(0);

    int n, m, i, x, l, r;
    cin >> n;
    build(1, n + 1, 1);  // 不要忘了初始化
    for (i = 1; i <= n; i++) {
        cin >> x;
        update(x, i, 1);  // 修改元素
    }
    cin >> m;  // 询问
    while (m--) {
        cin >> l >> r;
        cout << query(l, r + 1, 1).f << endl;  // 询问[l,r],实际查询[l,r+1),统一用左闭右开区间
    }

    return 0;
}
\end{lstlisting}