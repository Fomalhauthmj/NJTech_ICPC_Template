\section{最长上升子序列}

    \subsection{基本实现}
\begin{lstlisting}
#include <bits/stdc++.h>

using namespace std;
const int N = 1e3 + 10;

int n, a[N];
int b[N], c[N];  // b[i]是以a[i]为右端点的最长上升子序列的长度,c[i]是长度为i的最长上升子序列的右端点的最小值

inline void solve() {
    int i, p, len = 0;
    for (i = 1; i <= n; i++) {
        p = lower_bound(c + 1, c + 1 + len, a[i]) - c;
        b[i] = p, c[p] = a[i], len = max(len, p);
    }
}

int main() {
    ios::sync_with_stdio(0);
    cin.tie(0);

    int i;
    cin >> n;
    for (i = 1; i <= n; i++)
        cin >> a[i];
    solve();
    int ans = 0;
    for (i = 1; i <= n; i++)
        ans = max(ans, b[i]);
    cout << ans << endl;

    return 0;
}
\end{lstlisting}

    \subsection{另一种解法:权值线段树}
        对于a[i],我们可以通过查找以比a[i]小的数为结尾的最长长度,然后+1即可得到以a[i]为结尾的最长长度。\\
        权值线段树维护以某权值为结尾的最大长度。注意需要离散化。\\
        由于是查前缀,所以权值线段树可以换用树状数组
\begin{lstlisting}
#include <bits/stdc++.h>

using namespace std;
const int N = 1e5 + 10;

int n, a[N];
int b[N];    // b[i]是以a[i]为右端点的最长上升子序列的长度
int i2x[N];  // 离散化
int m;       // 离散化的长度
int t[N];    // 树状数组

inline int x2i(int x) {
    return lower_bound(i2x + 1, i2x + 1 + m, x) - i2x;
}

inline int lowbit(int x) {
    return -x & x;
}

inline void upd(int p, int v) {
    for (; p <= m; p += lowbit(p))
        t[p] = max(t[p], v);
}

inline int qry(int p) {
    int ret = 0;
    for (; p > 0; p -= lowbit(p))
        ret = max(ret, t[p]);
    return ret;
}

inline void solve() {
    int i;
    for (i = 1; i <= n; i++) {
        b[i] = qry(x2i(a[i]) - 1) + 1;
        upd(x2i(a[i]), b[i]);
    }
}

int main() {
    ios::sync_with_stdio(0);
    cin.tie(0);

    int i;
    cin >> n;
    for (i = 1; i <= n; i++)
        cin >> a[i], i2x[i] = a[i];
    sort(i2x + 1, i2x + 1 + n);
    m = unique(i2x + 1, i2x + 1 + n) - i2x - 1;

    solve();

    int ans = 0;
    for (i = 1; i <= n; i++)
        ans = max(ans, b[i]);
    cout << ans << endl;

    return 0;
}
\end{lstlisting}

    \subsection{输出最小字典序的下标}
        原问题是记录了以a[i]为结尾的最长上升子序列的长度,
        而此问题是记录以a[i]为\textbf{开头}的最长上升子序列的长度\\
        PS:若要求输出最大字典序的下标,只要还改成记录以a[i]为\textbf{结尾}的最长上升子序列的长度,然后倒过来筛就可以了
\begin{lstlisting}
#include <bits/stdc++.h>

using namespace std;
const int N = 1e5 + 10;

int n, a[N];
int d[N], c[N];  // d[i]是以a[i]为左端点的最长上升子序列的长度,c[i]用于记录从右向左长度为i的最小左端点

// 此处定义的是小于号,注意,不可以有等于。lower_bound配上<=就是upper_bound。
// 所以无论是lower还是upper都不带等号
bool cmp(const int& lhs, const int& rhs) {
    return lhs > rhs;
}

inline void solve() {
    int i, len, mx = 0;
    for (i = n; i >= 1; i--) {
        len = lower_bound(c + 1, c + 1 + mx, a[i], cmp) - c;
        d[i] = len, c[len] = a[i], mx = max(mx, len);
    }
}

int main() {
    ios::sync_with_stdio(0);
    cin.tie(0);

    int i;
    cin >> n;
    for (i = 1; i <= n; i++)
        cin >> a[i];

    solve();

    int mx = 0;
    for (i = 1; i <= n; i++)
        mx = max(mx, d[i]);

    cout << mx << endl;

    int cur = mx, last = 0;
    for (i = 1; i <= n; i++) {
        if (d[i] == cur && a[i] > last) {
            if (cur != mx)
                cout << " ";
            cout << i;  // 输出下标
            --cur, last = a[i];
        }
    }
    cout << endl;

    return 0;
}
\end{lstlisting}

    \subsection{输出值的最小字典序}
        与上面问题不同的是,这里要求的是值的最小字典序。\\
        例如:6 7 8 9 1 2 3 4,我们要输出的是1 2 3 4。\\
        对于这个问题,我们只要记录每个数的前驱即可。
\begin{lstlisting}
#include <bits/stdc++.h>

using namespace std;
const int N = 1e5 + 10;

int n, a[N];
int b[N], c[N];      // b[i]是以a[i]为右端点的最长上升子序列的长度,c[i]用于记录长度为i的最小端点
int pos[N], pre[N];  // pos[i]是与c[i]对应的下标,pre[i]是a[i]的前驱的下标
int ans[N];
// a[] b[] pre[] 下标一致;c[] pos[] 下标一致

inline void solve() {
    int i, len, mx = 0;
    for (i = 1; i <= n; i++) {
        len = lower_bound(c + 1, c + 1 + mx, a[i]) - c;
        b[i] = len, pre[i] = pos[len - 1];
        c[len] = a[i], pos[len] = i;
        mx = max(mx, len);
    }
}

int main() {
    ios::sync_with_stdio(0);
    cin.tie(0);

    int i, T;
    cin >> T;
    while (T--) {
        cin >> n;
        for (i = 1; i <= n; i++)
            cin >> a[i];

        solve();

        int mx = 0;
        for (i = 1; i <= n; i++)
            mx = max(mx, b[i]);

        cout << mx << endl;

        int p = pos[mx], len = mx;
        while (p) {
            ans[len--] = a[p];
            p = pre[p];
        }

        for (i = 1; i <= mx; i++) {
            if (i > 1)
                cout << " ";
            cout << ans[i];
        }
        cout << endl;
    }

    return 0;
}
\end{lstlisting}


    \subsection{最长不降子序列}
\begin{lstlisting}
p = upper_bound(c + 1, c + 1 + len, a[i]) - c;
\end{lstlisting}