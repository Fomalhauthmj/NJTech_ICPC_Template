\section{最长公共子序列}
    给定S1和S2串,要求一个序列,同时是S1和S2的子序列。
\subsection{CCPC 2019 秦皇岛 C.Sakura Reset}
    \subsubsection{题目描述}
        给定A、B串及其长度n、m$(1\le n,m\le 5000)$,要求A的子序列a和B的子序列b,
        使得a的长度大于b的长度或a、b长度相等且字符串a大于字符串b。
    \subsubsection{解决方案}
        整体上,要分类讨论:1.a的长度大于b的长度;2.长度相等且字符串a大于字符串b\\\\
        对于分类1:\\
        即要分别考虑A和B的各个长度的本质不同的子序列个数。答案就是枚举A的每个长度,其个数乘上B的小于该长度的自序列个数。\\
        如何统计呢?首先要理解贪心匹配,即如何判断t是否是s的子序列?
        根据贪心匹配,对于元素$t_1$我们就是要找它在s中出现的\textbf{第一个}
        位置$p_1$($t_1=s_{p_1}$),接下来在s的$p_1$后面找$t_2$并令其为$p_2$,以此类推。\\
        根据上面的规则,对于一个串X,我们令$f(i)$表示以$X[i]$为结尾的\textbf{与之前的不重复}的且\textbf{本质不同}的子序列个数。
        则转移方程
        $$f(i)=\sum_{j=last[i]}^{i-1}{f(j)}$$
        其中$last[i]$表示$X[i]$上一次出现的位置,若第一次出现$last[i]=0$。\\
        为什么是从$last[i]$开始累加呢?
        因为以下标小于$last[i]$的元素为结尾的子序列后面再加上当前的$X[i]$得出的新的子序列,和之前
        加上$last[i]$处的$X[last[i]]$组成的子序列重复了。通过\textbf{一维前缀和}可以优化至$O(n)$。
        回到现在的问题,我们要求每个长度上的本质不同的子序列个数,只要加上一维长度维度即可($f(i,j)$)。时间$O(n^2)$。\\\\
        对于分类2:\\
        要求长度相等,且按字符串的方式比较a大于b。即要求a和b有一段长度大于等于0的相同前缀,在第一个不相等的位置a的元素
        大于b的元素,后面只要有长度相等的后缀即可。\\
        令$g(i,j)$表示以$A[i]$结尾的a和以$B[j]$结尾的b\textbf{与之前的不重复}的且\textbf{本质不同}的公共子序列对数。\\
        若$A[i]\neq b[i]$,显然$g(i,j)=0$;若$A[i]=B[j]$,
        $$g(i,j)=\sum_{u=last[i]}^{i-1}{\sum_{v=last[j]}^{j-1}{g(u,v)}}$$
        再考虑后缀,令$h(i,j)$为从后向前以$A[i]$结尾的a和以$B[j]$结尾的b\textbf{与之前的不重复}的且\textbf{本质不同}的长度相等子序列对数。
        后缀的求法有两种,可以和求$f$向类似,但是在算答案的时候回非常复杂,也可以和求$g$类似,只不过是倒着求,并且不要求$A[i]=B[j]$。
        最后枚举$i$、$j$,若$A[i]>B[j]$,则
        $$ans+=(\sum_{u=1}^{i-1}{\sum_{v=1}^{j-1}{g(u,v)}})*h(i,j)$$
        以上都可以通过\textbf{二维前缀和或二维后缀和}优化至$O(N^2)$。\\\\
        \textbf{注意:上面两个前缀和的意义是不一样的。$f$的前缀和是一维的,它只对下标i求前缀和,不对长度j求;
        而下面$g$的前缀和以及$h$的后缀和是二维的,即i,j都要求。\underline{所以还要注意两种在边界上的初始化是不一样的}}。
    \subsubsection{代码}
\begin{lstlisting}
#include <bits/stdc++.h>

using namespace std;
typedef long long ll;
const ll N = 5e3 + 10, P = 998244353;

ll len[2];         // 两串的长度
ll S[2][N];        // A串和B串
ll f[2][N][N];     // A串和B串的dp,f[0][i][j]:以A[i]为结尾且长度为j的本质不同的子序列个数
ll fpre[2][N][N];  // 对f的**i**求前缀和,并不是二维前缀和
ll last[2][N];     // A串和B串,A[i]或B[i]上一次出现的位置
ll pos[2][110];    // pos[i],值为i的元素上一次出现的位置
ll g[N][N];        // 二维矩阵,g[i][j]表示以A[i]结尾的子序列和以B[i]结尾的子序列为公共子序列的方案数
ll gpre[N][N];     // g的二维前缀和
ll h[N][N];        // 二维矩阵,h[i][j]表示 从后向前 以A[i]结尾的子序列和以B[i]结尾的子序列长度相等但本质不同的后缀对个数
ll hsuf[N][N];     // h的二维后缀和

inline bool isOk(ll r, ll c) {
    return r >= 0 && c >= 0;
}

inline ll presum(ll r1, ll c1, ll r2, ll c2) {
    ll v1 = 0, v2 = 0, v3 = 0, v4 = 0;
    if (isOk(r2, c2))
        v1 = gpre[r2][c2];
    if (isOk(r1 - 1, c2))
        v2 = gpre[r1 - 1][c2];
    if (isOk(r2, c1 - 1))
        v3 = gpre[r2][c1 - 1];
    if (isOk(r1 - 1, c1 - 1))
        v4 = gpre[r1 - 1][c1 - 1];
    return v1 - v2 - v3 + v4;
}

inline ll sufsum(ll r1, ll c1, ll r2, ll c2) {
    return hsuf[r2][c2] - hsuf[r1 + 1][c2] - hsuf[r2][c1 + 1] + hsuf[r1 + 1][c1 + 1];
}

int main() {
    ios::sync_with_stdio(0);
    cin.tie(0);

    ll i, j, k, ans = 0;

    cin >> len[0] >> len[1];

    for (k = 0; k <= 1; k++)
        for (i = 1; i <= len[k]; i++)
            cin >> S[k][i];

    // 预处理last
    for (k = 0; k <= 1; k++)
        for (i = 1; i <= len[k]; i++)
            last[k][i] = pos[k][S[k][i]],
            pos[k][S[k][i]] = i;

    // 对于长度不同的情况-------------------------------------------
    // 首先求f和fpre
    f[0][0][0] = f[1][0][0] = fpre[0][0][0] = fpre[1][0][0] = 1;
    for (k = 0; k <= 1; k++) {
        // 初始化边界
        for (i = 1; i <= len[k]; i++)
            fpre[k][i][0] = 1;
        for (i = 1; i <= len[k]; i++)
            for (j = 1; j <= i; j++)
                f[k][i][j] = (fpre[k][i - 1][j - 1] - fpre[k][last[k][i]][j - 1] + f[k][last[k][i]][j - 1]) % P,
                fpre[k][i][j] = (fpre[k][i - 1][j] + f[k][i][j]) % P;
    }
    // 计算长度不同的答案
    ll t = 0;
    for (i = 1; i <= len[0]; i++)
        ans = (ans + fpre[0][len[0]][i] * t % P) % P,
        t += fpre[1][len[1]][i];

    // 对于长度相同的情况-------------------------------------------
    // 首先求g和gpre
    g[0][0] = gpre[0][0] = 1;
    // 初始化边界
    for (i = 1; i <= len[0]; i++)
        gpre[i][0] = 1;
    for (i = 1; i <= len[1]; i++)
        gpre[0][i] = 1;
    for (i = 1; i <= len[0]; i++)
        for (j = 1; j <= len[1]; j++)
            g[i][j] = (S[0][i] == S[1][j] ? presum(last[0][i], last[1][j], i - 1, j - 1) : 0),
            gpre[i][j] = gpre[i - 1][j] + gpre[i][j - 1] - gpre[i - 1][j - 1] + g[i][j];
    // 再求从后向前以i为结尾的长度为j的本质不同的子序列个数,求法类似之前求的f,只是倒过来了
    // 在此之前,先预处理last
    for (k = 0; k <= 1; k++)
        for (i = 1; i <= 100; i++)
            pos[k][i] = len[k] + 1;
    for (k = 0; k <= 1; k++)
        for (i = len[k]; i >= 1; i--)
            last[k][i] = pos[k][S[k][i]],
            pos[k][S[k][i]] = i;
    // 开始求
    hsuf[len[0] + 1][len[1] + 1] = h[len[0] + 1][len[1] + 1] = 1;
    // 初始化边界
    for (i = 1; i <= len[1]; i++)
        hsuf[len[0] + 1][i] = 1;
    for (i = 1; i <= len[0]; i++)
        hsuf[i][len[1] + 1] = 1;
    for (i = len[0]; i >= 1; i--)
        for (j = len[1]; j >= 1; j--)
            h[i][j] = sufsum(last[0][i], last[1][j], i + 1, j + 1),
            hsuf[i][j] = hsuf[i + 1][j] + hsuf[i][j + 1] - hsuf[i + 1][j + 1] + h[i][j];
    // 计算长度相等时的答案
    for (i = 1; i <= len[0]; i++)
        for (j = 1; j <= len[1]; j++)
            if (S[0][i] > S[1][j])
                ans = (ans + gpre[i - 1][j - 1] * h[i][j]) % P;

    cout << ans << endl;

    return 0;
}
\end{lstlisting}