\section{后缀自动机}
\begin{lstlisting}
#define ll long long
const int MAXLEN = 1e6 + 50;
struct SAM
{
    int len[MAXLEN << 1], link[MAXLEN << 1], next[MAXLEN << 1][26];
    ll sze[MAXLEN << 1]; ////每个结点所代表的字符串的出现次数
    int sz, last, rt;
    int NewNode(int x = 0)
    {
        len[sz] = x;
        link[sz] = -1;
        memset(next[sz], -1, sizeof(next[sz]));
        return sz++;
    }
    void Init()
    {
        //重置
        sz = last = 0, rt = NewNode();
    }
    void Extend(int c)
    {
        int cur = NewNode(len[last] + 1);
        sze[cur] = 1;
        int p = last;
        while (~p && next[p][c] == -1)
            next[p][c] = cur, p = link[p];
        if (p == -1)
            link[cur] = rt;
        else
        {
            int q = next[p][c];
            if (len[q] == len[p] + 1)
                link[cur] = q;
            else
            {
                int clone = NewNode(len[p] + 1);
                memcpy(next[clone], next[q], sizeof(next[q]));
                link[clone] = link[q], link[q] = link[cur] = clone;
                while (~p && next[p][c] == q)
                    next[p][c] = clone, p = link[p];
            }
        }
        last = cur;
    }
    int id[MAXLEN << 1], c[MAXLEN];
    void Topo()
    {
        //计数排序
        memset(c, 0, sizeof(c));
        for (int i = 0; i < sz; i++)
            c[len[i]]++;
        for (int i = 1; i < MAXLEN; i++)
            c[i] += c[i - 1];
        for (int i = 0; i < sz; i++)
            id[--c[len[i]]] = i;
        for (int i = sz - 1; ~i; i--)
        {
            int u = id[i];
            if (~link[u])
                sze[link[u]] += sze[u];
        }
    }
};
\end{lstlisting}
\section{Manacher}
\begin{lstlisting}
const int MAXLEN = 1e5 + 50;
char ori[MAXLEN], str[MAXLEN * 2];
int d1[MAXLEN * 2], n, m;    
void Manacher()
{
    for (int i = 0, l = 0, r = -1; i < m; i++)
    {
        int k = (i > r) ? 1 : min(d1[l + r - i], r - i);
        while (i - k >= 0 && i + k < m && str[i - k] == str[i + k])
            k++;
        d1[i] = k--;
        if (i + k > r)
            l = i - k, r = i + k;
    }
}
int main()
{
    scanf("%s", ori);
    n = strlen(ori);
    str[0] = '#';
    for (int i = 0; i < n; i++)
    {
        str[(i + 1) * 2] = '#';
        str[(i + 1) * 2 - 1] = ori[i];
    }
    m = n * 2 + 1;
    Manacher();
}
\end{lstlisting}

\section{回文树/回文自动机}
\begin{lstlisting}
const int MAXLEN=5e5+50;
struct Palindromic_Tree
{
    int nxt[MAXLEN][26],fail[MAXLEN],len[MAXLEN],s[MAXLEN];
    int cnt[MAXLEN];// 结点表示的本质不同的回文串的个数(调用Count()后) 
    int num[MAXLEN];// 结点表示的最长回文串的最右端点为回文串结尾的回文串个数 
    int last,sz,n;
    int NewNode(int x)
    {
        memset(nxt[sz],0,sizeof(nxt[sz]));
        cnt[sz]=num[sz]=0,len[sz]=x;
        return sz++;
    }
    void Init()
    {
        sz=0;
        NewNode(0),NewNode(-1);
        last=n=0,s[0]=-1,fail[0]=1;
    }
    int GetFail(int u)
    {
        while(s[n-len[u]-1]!=s[n]) u=fail[u];
        return u;
    }
    void Add(int c)
    {
        //c-='a'
        s[++n]=c;
        int u=GetFail(last);
        if(!nxt[u][c])
        {
            int np=NewNode(len[u]+2);
            fail[np]=nxt[GetFail(fail[u])][c];
            num[np]=num[fail[np]]+1;
            nxt[u][c]=np;
        }
        last=nxt[u][c];
        cnt[last]++;
    }
    void Count()
    {
        for(int i=sz-1;~i;i--)
            cnt[fail[i]]+=cnt[i];
    }
}
\end{lstlisting}