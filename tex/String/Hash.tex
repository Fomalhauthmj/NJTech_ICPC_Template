\section{字符串Hash}
\begin{lstlisting}
#define ull unsigned long long
#define P 131
ull f[N], p[N];
void Init()
{
    p[0] = 1, f[0] = 0;
    for (int i = 1; i <= n; i++)
    {
        p[i] = p[i - 1] * P;
        f[i] = f[i - 1] * P + str[i];
    }
}
ull Hash(int left, int right)
{
    return f[right] - f[left - 1] * p[right - left + 1];
}
\end{lstlisting}

\subsection{应用:最长回文子串}
枚举回文子串的中心位置,求解从中心位置出发向左右两侧最长可扩展出多长的回文串,

分奇、偶回文子串,二分长度。
\begin{lstlisting}
#define ull unsigned long long
#define P 131
const int N = 1e6 + 50;
ull pre[N], suf[N], p[N];
string str;
int n;
void Init()
{
    pre[0] = suf[0] = 0;
    for (int i = 1; i <= n; i++)
    {
        pre[i] = pre[i - 1] * P + str[i - 1];
        suf[i] = suf[i - 1] * P + str[n - i];
    }
}
ull Hash(int left, int right)
{
    return pre[right] - pre[left - 1] * p[right - left + 1];
}
ull HashReverse(int left, int right)
{
    //WA 这里注意推导
    return suf[n - left + 1] - suf[n - right] * p[right - left + 1];
}
#define ODD 1
#define EVEN 2
bool Judge(int pos, int len, int flag)
{
    if (flag == ODD)
    {
        if (pos - len < 1 || pos > n || pos < 1 || pos + len > n)
            return false;
        return Hash(pos - len, pos) == HashReverse(pos, pos + len);
    }
    else
    {
        if (pos - len < 1 || pos - 1 > n || pos < 1 || pos + len - 1 > n)
            return false;
        return Hash(pos - len, pos - 1) == HashReverse(pos, pos + len - 1);
    }
}
int Solve(int pos)
{
    int ans = 1;
    //奇数
    int left = 0;
    int right = N;
    int mid;
    while (left < right)
    {
        mid = (left + right + 1) >> 1;
        if (Judge(pos, mid, ODD))
            left = mid;
        else
            right = mid - 1;
    }
    ans = max(ans, 2 * left + 1);
    //偶数
    left = 0;
    right = N;
    while (left < right)
    {
        mid = (left + right + 1) >> 1;
        if (Judge(pos, mid, EVEN))
            left = mid;
        else
            right = mid - 1;
    }
    ans = max(ans, 2 * left);
    return ans;
}
\end{lstlisting}

\subsection{应用:后缀数组}

\begin{lstlisting}
#define ull unsigned long long
#define P 131
const int N = 3e5 + 50;
ull f[N], p[N];
char str[N];
int SA[N], n, Height[N];
void Init()
{
    p[0] = 1, f[0] = 0;
    for (int i = 1; i <= n; i++)
    {
        p[i] = p[i - 1] * P;
        f[i] = f[i - 1] * P + str[i];
    }
}
ull Hash(int left, int right)
{
    return f[right] - f[left - 1] * p[right - left + 1];
}
// k∈[0,n) 表示后缀S(k,n-1)
// 最长公共前缀
int LCP(int a, int b)
{
    int left = 0;
    int right = N;
    int mid;
    while (left < right)
    {
        mid = (left + right + 1) >> 1;
        if (a + mid - 1 <= n && b + mid - 1 <= n && Hash(a, a + mid - 1) == Hash(b, b + mid - 1))
            left = mid;
        else
            right = mid - 1;
    }
    return left;
}
bool cmp(int a, int b)
{
    int len = LCP(a, b);
    return str[a + len] < str[b + len];
}
void calc_height()
{
    Height[1] = 0;
    for (int i = 2; i <= n; i++)
        Height[i] = LCP(SA[i], SA[i - 1]);
}
int main()
{
    scanf("%s", str + 1);
    n = strlen(str + 1);
    for (int i = 1; i <= n; i++)
        SA[i] = i;
    Init();
    sort(SA + 1, SA + n + 1, cmp);
    calc_height();
}
\end{lstlisting}

\subsection{二维Hash}
给定一个M行N列的01矩阵(只包含数字0或1的矩阵),再执行Q次询问,每次询问给出一个A行B列的01矩阵,求该矩阵是否在原矩阵中出现过。

做法:选取两个不同的P值分别对行列进行Hash处理,应用二维前缀和求取矩阵Hash值。
\begin{lstlisting}
#define P 131
#define Q 13331
#define ull unsigned long long
void Init()
{
    char ch;
    for (int i = 1; i <= m; i++)
        for (int j = 1; j <= n; j++)
            cin >> ch, Hash[i][j] = Hash[i][j - 1] * P + ch;
    for (int i = 1; i <= m; i++)
        for (int j = 1; j <= n; j++)
            Hash[i][j] += Hash[i - 1][j] * Q;
}
ull temp = Hash[i][j] - Hash[i - a][j] * q[a] - Hash[i][j - b] * p[b] + Hash[i - a][j - b] * q[a] * p[b];
\end{lstlisting}