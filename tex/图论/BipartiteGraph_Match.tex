\section{二分图的匹配}
\subsection{二分图判定}
    一张无向图是二分图,当且仅当图中不存在奇环(长度为奇数的环)。
\begin{lstlisting}
//染色法判定奇环
bool DFS(int x,int color)
{
    vis[x]=color;
    for(int i=head[x];i;i=nxt[i])
    {
        int y=ver[i];
        if(!vis[y])
        {
            if(!DFS(y,3-color)) return false;
        }
        else if(vis[y]==color) return false;
    }
    return true;
}
\end{lstlisting}

例题分析

P1525 关押罪犯(判定二分图,二分)

\subsection{二分图最大匹配}
二分图匹配的模型要素

1:节点能分成独立的两个集合,每个集合内部有 0 条边。 “0要素”

2:每个节点只能与 1 条匹配边相连。 “1要素”
\begin{lstlisting}
\\匈牙利算法
\\在主函数中对每个左部节点调用寻找增广路时,需要对 vis 重置。
bool DFS(int x)
{
    for (int i = head[x]; i; i = nxt[i])
    {
        int y = ver[i];
        if (!vis[y])
        {
            vis[y] = 1;
            if (!match[y] || DFS(match[y]))
            {
                match[y] = x;
                return true;
            }
        }
    }
    return false;
}
\end{lstlisting}

例题分析

6801 棋盘覆盖(奇偶染色)

6802 車的放置(行列)

6803 导弹防御塔(二分,拆点多重匹配)

\subsection{二分图带权匹配}
二分图带权最大匹配的前提是匹配数最大,然后再最大化匹配边的权值总和。
\begin{lstlisting}
/*KM 稠密图上效率高于费用流,但是有较大局限性,只能在满足“带权最大匹配一定是完备匹配”的图中正确求解。
w[][]:边权
la[],lb[]: 左,右部点顶标
visa[],visb[]: 访问标记,是否在交错树中
ans: Σw[match[i]][i]
*/
bool DFS(int x)
{
    visa[x] = true;
    for (int y = 1; y <= n; y++)
    {
        if (!visb[y])
        {
            double temp = fabs(la[x] + lb[y] - w[x][y]);//对于浮点数,相等子图的判定
            if (temp < eps)
            {
                visb[y] = true;
                if (!match[y] || DFS(match[y]))
                {
                    match[y] = x;
                    return true;
                }
            }
            else
                upd[y] = min(upd[y], la[x] + lb[y] - w[x][y]);
        }
    }
    return false;
}
void KM()
{
    for (int i = 1; i <= n; i++)
    {
        la[i] = -inf;
        lb[i] = 0;
        for (int j = 1; j <= n; j++)
            la[i] = max(la[i], w[i][j]);
    }
    for (int i = 1; i <= n; i++)
    {
        while (true)
        {
            memset(visa, 0, sizeof(visa));
            memset(visb, 0, sizeof(visb));
            for (int j = 1; j <= n; j++)
                upd[j] = inf;
            if (DFS(i))
                break;
            else
            {
                delta = inf;
                for (int j = 1; j <= n; j++)
                    if (!visb[j])
                        delta = min(delta, upd[j]);
                for (int j = 1; j <= n; j++)
                {
                    if (visa[j])
                        la[j] -= delta;
                    if (visb[j])
                        lb[j] += delta;
                }
            }
        }
    }
}
\end{lstlisting}

例题分析

POJ3565 Ants(三角形不等式,二分图带权最小匹配)