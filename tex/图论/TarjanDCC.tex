\section{Tarjan算法与无向图连通性}
\subsection{无向图的割点与桥}
\subsubsection{割边判定法则}
\begin{lstlisting}
void Tarjan(int x, int in_edge)
{
    dfn[x] = low[x] = ++num;
    for (int i = head[x]; i; i = nxt[i])
    {
        int y = ver[i];
        if (!dfn[y])
        {
            Tarjan(y, i);
            low[x] = min(low[x], low[y]);
            if (low[y] > dfn[x])
            {
                bridge[i] = bridge[i ^ 1] = true;
            }
        }
        else if (i != (in_edge ^ 1))
            low[x] = min(low[x], dfn[y]);
    }
}
\end{lstlisting}
\subsubsection{割点判定法则}
\begin{lstlisting}
void Tarjan(int x)
{
    dfn[x] = low[x] = ++num;
    int flag = 0;
    for (int i = head[x]; i; i = nxt[i])
    {
        int y = ver[i];
        if (!dfn[y])
        {
            Tarjan(y);
            low[x] = min(low[x], low[y]);
            if (low[y] >= dfn[x])
            {
                flag++;
                if (x != root || flag >= 2)
                    cut[x] = true;
            }
        }
        else
            low[x] = min(low[x], dfn[y]);
    }
}
\end{lstlisting}

例题分析

P3469 [POI2008]BLO-Blockade(割点,连通块计数)

\subsection{无向图的双连通分量}
\subsubsection{边双连通分量e-DCC与其缩点}
\begin{lstlisting}
void DFS(int x)
{
    color[x] = dcc;
    for (int i = head[x]; i; i = nxt[i])
    {
        int y = ver[i];
        if (!color[y] && !bridge[i])
            DFS(y);
    }
}
void e_DCC()
{
    dcc = 0;
    for (int i = 1; i <= n; i++)
        if (!color[i])
            ++dcc, DFS(i);
    totc = 1;
    for (int i = 2; i <= tot; i++)
    {
        int u = ver[i ^ 1], v = ver[i];
        if (color[u] != color[v])
            add_c(color[u], color[v]);
    }
}
\end{lstlisting}
\subsubsection{点双连通分量v-DCC与其缩点}
\begin{lstlisting}
void Tarjan(int x)
{
    dfn[x] = low[x] = ++num;
    int flag = 0;
    stack[++top] = x;
    if (x == root && !head[x])
    {
        dcc[++cnt].push_back(x);
        return;
    }
    for (int i = head[x]; i; i = nxt[i])
    {
        int y = ver[i];
        if (!dfn[y])
        {
            Tarjan(y);
            low[x] = min(low[x], low[y]);
            if (low[y] >= dfn[x])
            {
                flag++;
                if (x != root || flag >= 2)
                    cut[x] = true;
                cnt++;
                int z;
                do
                {
                    z = stack[top--];
                    dcc[cnt].push_back(z);
                } while (z != y);
                dcc[cnt].push_back(x);
            }
        }
        else
            low[x] = min(low[x], dfn[y]);
    }
}
void v_DCC()
{
    cnt = 0;
    top = 0;
    for (int i = 1; i <= n; i++)
    {
        if (!dfn[i])
            root = i, Tarjan(i);
    }
    // 给每个割点一个新的编号(编号从cnt+1开始)
    num = cnt;
    for (int i = 1; i <= n; i++)
        if (cut[i]) new_id[i] = ++num;
    // 建新图,从每个v-DCC到它包含的所有割点连边
    tc = 1;
    for (int i = 1; i <= cnt; i++)
        for (int j = 0; j < dcc[i].size(); j++) 
        {
            int x = dcc[i][j];
            if (cut[x]) {
                add_c(i, new_id[x]);
                add_c(new_id[x], i);
            }
            else c[x] = i; // 除割点外,其它点仅属于1个v-DCC
        }
}
\end{lstlisting}
例题分析

POJ3694 Network(e-DCC缩点,LCA,并查集)

POJ2942 Knights of the Round Table(补图,v-DCC,染色法奇环判定)

\subsection{欧拉路问题}
欧拉图的判定

无向图连通,所有点度数为偶数。

欧拉路的存在性判定

无向图连通,恰有两个节点度数为奇数,其他节点度数均为偶数

\begin{lstlisting}
// 模拟系统栈,答案栈
void Euler() {
    stack[++top] = 1;
    while (top > 0) {
        int x = stack[top], i = head[x];
        // 找到一条尚未访问的边
        while (i && vis[i]) i = Next[i];
        // 沿着这条边模拟递归过程,标记该边,并更新表头
        if (i) {
            stack[++top] = ver[i];
            head[x] = Next[i];
            vis[i] = vis[i ^ 1] = true;
        }		
        // 与x相连的所有边均已访问,模拟回溯过程,并记录于答案栈中
        else {
            top--;
            ans[++t] = x;
        }
    }
}
\end{lstlisting}
例题分析

POJ2230 Watchcow(欧拉回路)
