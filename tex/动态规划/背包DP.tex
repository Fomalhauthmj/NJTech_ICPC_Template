\section{背包DP}

\subsection{0-1背包 洛谷P1048 采药}
    \subsubsection{题目描述}
        有$n (1\le n\le 100)$株药草,每一株有一个采摘时长$a_i$和价值$b_i$ $(1\le a_i,b_i\le 100)$,给你$m (1\le m\le 1000)$时间,问最大价值是多少?
    \subsubsection{代码}
\begin{lstlisting}
#include <bits/stdc++.h>

using namespace std;
const int N = 1e5 + 10;

int n, m;            // 物品数量,最大容量
int f[N];            // f[i]表示用i容量能得到的最大价值
int c[110], v[110];  // 每一个物品的消耗和价值

int main() {
    ios::sync_with_stdio(0);
    cin.tie(0);

    int i, j;
    cin >> m >> n;
    for (i = 1; i <= n; i++)
        cin >> c[i] >> v[i];

    // 对于每个物品i,恰好使用j容量时,不取或取a[i]
    // 注意j要倒着更新
    for (i = 1; i <= n; i++)
        for (j = m; j >= c[i]; j--)
            f[j] = max(f[j], f[j - c[i]] + v[i]);

    int ans = 0;
    for (i = 1; i <= m; i++)
        ans = max(ans, f[i]);

    cout << ans << endl;

    return 0;
}
\end{lstlisting}

\subsection{完全背包 洛谷P1616 疯狂的采药}
    \subsubsection{题目描述}
        与上一题描述差不多,但此题和原题的不同点:\\
        1.每种草药可以无限制地疯狂采摘。\\
        2.药的种类眼花缭乱,采药时间好长好长啊!\\
        另外,物品数$n (1\le n\le 10000)$,最大容量$m (1\le m\le 100000)$,物品消耗和价值$(1\le a_i,b_i\le 10000)$,
        并且$n*m\le 10^7$
    \subsubsection{代码1,直接枚举每种取的数量}
        大概率TLE,但还是要放在这,因为有时候可能数据范围不大,但却有别的限制
\begin{lstlisting}
#include <bits/stdc++.h>

using namespace std;
const int N = 1e4 + 10, M = 1e5 + 10;

int n, m;        // 物品数量,最大容量
int f[M];        // f[i]表示用i容量能得到的最大价值
int c[N], v[N];  // 每一个物品的消耗和价值

int main() {
    ios::sync_with_stdio(0);
    cin.tie(0);

    int i, j, k;
    cin >> m >> n;
    for (i = 1; i <= n; i++)
        cin >> c[i] >> v[i];

    // 对于每个物品i,恰好使用j容量时,取k个a[i]
    // 注意j要倒着更新
    for (i = 1; i <= n; i++)
        for (j = m; j >= c[i]; j--)
            for (k = 0; k <= j / c[i]; k++)
                f[j] = max(f[j], f[j - k * c[i]] + k * v[i]);

    int ans = 0;
    for (i = 1; i <= m; i++)
        ans = max(ans, f[i]);

    cout << ans << endl;

    return 0;
}
\end{lstlisting}
    \subsubsection{代码2,巧妙地用无限取这个条件}
        由于可以无限取,所以其实不用关心先后更新
\begin{lstlisting}
// 对于每个物品i,恰好使用j容量时,取k个a[i]
// 注意j的更新顺序,非常妙
for (i = 1; i <= n; i++)
    for (j = c[i]; j <= m; j++)
        f[j] = max(f[j], f[j - c[i]] + v[i]);
}
\end{lstlisting}

\subsection{分组背包}
    \subsubsection{问题描述}
        有N件物品和一个容量为V的背包。第i件物品的费用是c[i],价值是w[i]。这些物品被划分为若干组,每组中的物品互相冲突,最多选一件。求解将哪些物品装入背包可使这些物品的费用总和不超过背包容量,且价值总和最大。
    \subsubsection{解决方案}
        状态空间:f[k][v]表示前k组物品花费费用v能取得的最大权值\\
        状态转移方程:\\
        f[k][v]=max{f[k-1][v],f[k-1][v-c[i]]+w[i]|物品i属于第k组}\\
        使用一维数组的伪代码如下:
\begin{lstlisting}
for 所有的组k
    for v=V..0
        for 所有的i属于组k
            f[v]=max{f[v],f[v-c[i]]+w[i]}
\end{lstlisting}

\subsection{多重背包}
    \subsubsection{问题描述}
        有 N 种物品和一个容量是 V 的背包。\\
        第 i 种物品最多有 si 件,每件体积是 vi,价值是 wi。\\
        求解将哪些物品装入背包,可使物品体积总和不超过背包容量,且价值总和最大。\\
        输出最大价值。
    \subsubsection{解法1:直接枚举}
        与完全背包的暴力解法相似,只要加上k<=s[i]这个条件
    \subsubsection{解法2:二进制优化}
\begin{lstlisting}
#include <bits/stdc++.h>

using namespace std;
const int N = 16 * 1e3 + 10, M = 2e3 + 10;

int n, m;              // 物品数量,最大容量
int f[M];              // f[i]表示用i容量能得到的最大价值
int s[N], v[N], w[N];  // 每一个物品的个数、消耗和价值

int main() {
    ios::sync_with_stdio(0);
    cin.tie(0);

    int i, j, cnt;
    cin >> n >> m;
    for (i = 1; i <= n; i++)
        cin >> v[i] >> w[i] >> s[i];

    // 二进制优化
    cnt = n;
    for (i = 1; i <= n; i++) {
        int t = s[i], d = 1;
        while (t >= d)
            v[++cnt] = v[i] * d, w[cnt] = w[i] * d, t -= d, d <<= 1;
        if (t)
            v[++cnt] = v[i] * t, w[cnt] = w[i] * t;
    }

    // 0-1背包DP
    // 注意i要从n+1开始,因为前n个是未拆分的,如果算上那么每个物品的数量就被错误的乘2了
    for (i = n + 1; i <= cnt; i++)
        for (j = m; j >= v[i]; j--)
            f[j] = max(f[j], f[j - v[i]] + w[i]);

    int ans = 0;
    for (i = 1; i <= m; i++)
        ans = max(ans, f[i]);

    cout << ans << endl;

    return 0;
}
\end{lstlisting}
    \subsubsection{解法3:单调队列优化}
\begin{lstlisting}
#include <bits/stdc++.h>

using namespace std;
const int N = 1e3 + 10, M = 2e4 + 10;

int n, m;              // 物品数量,最大容量
int f[M];              // f[i]表示用i容量能得到的最大价值
int s[N], v[N], w[N];  // 每一个物品的个数、消耗和价值

struct CycQue {
    const static int MAXSZ = M;
    typedef pair<int, int> Ele;
    Ele q[MAXSZ];
    int head, tail, sz;
    CycQue()
        : head(1), tail(0), sz(0) {}
    void clear() {
        head = 1, tail = 0, sz = 0;
    }
    bool empty() {
        return !sz;
    }
    void push_back(Ele x) {
        // 队列满了
        if (sz >= MAXSZ - 1)
            sz = sz / (sz - sz);  // 除0,抛出异常
        tail = (tail + 1) % MAXSZ, q[tail] = x, ++sz;
    }
    void push_front(Ele x) {
        if (sz >= MAXSZ - 1)
            sz = sz / (sz - sz);
        head = (head - 1 + MAXSZ) % MAXSZ, q[head] = x, ++sz;
    }
    void pop_front() {
        if (!sz)
            sz = sz / (sz - sz);
        head = (head + 1) % MAXSZ, --sz;
    }
    void pop_back() {
        if (!sz)
            sz = sz / (sz - sz);
        tail = (tail - 1 + MAXSZ) % MAXSZ, --sz;
    }
    Ele front() {
        if (!sz)
            sz = sz / (sz - sz);
        return q[head];
    }
    Ele back() {
        if (!sz)
            sz = sz / (sz - sz);
        return q[tail];
    }
} q;

int main() {
    ios::sync_with_stdio(0);
    cin.tie(0);

    int i, j, b;
    scanf("%d%d", &n, &m);
    for (i = 1; i <= n; i++)
        scanf("%d%d%d", &v[i], &w[i], &s[i]);

    // 枚举物品
    for (i = 1; i <= n; i++) {
        // 枚举余数
        for (b = 0; b < v[i]; b++) {
            q.clear();
            // 枚举当前取的个数
            for (j = 0; j <= (m - b) / v[i]; j++) {
                int t = f[b + j * v[i]] - j * w[i];
                // 入队
                while (!q.empty() && q.back().second <= t)
                    q.pop_back();
                q.push_back({j, t});
                while (j - q.front().first > s[i])
                    q.pop_front();
                f[b + j * v[i]] = q.front().second + j * w[i];
            }
        }
    }

    int ans = 0;
    for (i = 1; i <= m; i++)
        ans = max(ans, f[i]);

    printf("%d\n", ans);

    return 0;
}
\end{lstlisting}

\subsection{杭电多校2019第一场C HDU6580 Milk}
    \subsubsection{题目描述}
        有一个n*m的矩阵,左右可以随便走,但只能在每一行的中点往下走,每走一格花费时间1.
        现在这个矩阵里放了k瓶牛奶,第i个牛奶喝下去需要ti时间
        起点是(1,1)
        对于每个i∈[1,k],问喝掉k瓶牛奶花费的最小时间
    \subsubsection{解决方案}
        首先对瓶子的横坐标离散化处理,一行一行地更新答案。
        每到新的一行,先预处理出向左(右)走喝了\(i\)瓶水后,回到原点以及不回到原点所需要的最短时间,分别记为\(l,r,l\_back,r\_back\)。
        然后将左右两边合并,求出喝了\(i\)瓶水后,回到或不回到原点所需时间,记为\(g,g\_back\)。
        设\(f[i]\)为从\((1,1)\)出发,喝了\(i\)瓶水后回到当前行中点所需的最短时间,
        则这时就可以根据\(g\)来和之前的\(f[i]\)来更新答案,并用\(g\_back\)来更新\(f\)的值。时间复杂度为\(O(k^2)\)

\subsection{ICPC上海2019网络预选赛 J-Stone game}
    \subsubsection{题目描述}
        有n个石头,每个石头都有一个重量为$a_i$。现要取出一部分,我们设它们为集合A,则剩余部分为集合B,要求重量$A\ge B$,
        并且当移去A中的任意一个石头$t$时,要求$(A-t)\le B$。问取法有多少种,最后输出模$1e9+7$\\
        其中$n\le 300, a_il\le 500$,样例数$T\le 100$
    \subsubsection{解决方案}
        容易发现,当划分出A之后,只要移除A中\textbf{最轻的石头t}是满足题意的这种方案就是可行的。\\
        于是我们可以先将\textbf{数组排序,然后枚举t},对每一个t分别考虑。最暴力的想法是考虑t后面的每一个$a_i$取或不取,
        但是这样复杂度上天。\\
        再仔细想想,就会发现所有$a_i$加起来也就$1.5e5$,也就是可以用\textbf{背包dp},$f(i,j)$表示$t=i$时总重量为j的方案数。
        复杂度是?不考虑T,枚举t为$O(N)$,对于每个t,考虑其后的每个数取或不取为$O(N*1.5e5)$,总复杂度为$O(N^2*1.5*10^5)$,又上天了。\\
        但是当我们\textbf{倒过来想},\textbf{将数组从大到小排序},然后从前向后枚举t,就会发现复杂度降到了$O(N*1.5*10^5)$。
        每次$++t$,都相当于整体加上$a_t$($a_t$必取),然后再考虑不取$a_{t-1}$的情况($a_{t-1}$在上一个t中是必选的)。