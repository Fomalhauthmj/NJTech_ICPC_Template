\section{二分图的覆盖与独立集}
\subsection{二分图最小点覆盖}
二分图最小覆盖模型特点:

每条边有 2 个端点,二者至少选择一个。  “2要素”
\subsubsection{König's theorem}
二分图最小点覆盖包含的点数等于二分图最大匹配包含的边数。

例题分析

POJ1325 Machine Schedule(二分图最小覆盖)

POJ2226 Muddy Fields(行列连续块,二分图最小覆盖)

\subsection{二分图最大独立集}

无向图 G 的最大团等于其补图 G' 的最大独立集。(补图转化)

设G是有 n 个节点的二分图,G 的最大独立集的大小等于 n 减去最大匹配数。

例题分析

6901 骑士放置 (奇偶染色)

\subsection{有向无环图的最小路径点覆盖}
给定一张有向无环图,要求用尽量少的不相交的简单路径,覆盖有向无环图的所有顶点(也就是每个顶点恰好被覆盖一次)。
这个问题被称为有向无环图的最小路径点覆盖,简称“最小路径覆盖”。

有向无环图 G 的最小路径点覆盖包含的路径条数,等于n(有向无环图的点数)减去拆点二分图 G2 的最大匹配数。

若简单路径可相交,即一个节点可被覆盖多次,这个问题称为有向无环图的最小路径可重复点覆盖。

对于这个问题,可先对G求传递闭包,得到有向无环图 G',再在 G' 上求一般的(路径不可相交的)最小路径点覆盖。

例题分析

6902 Vani和Cl2捉迷藏(最小路径可重复点覆盖,构造方案)