\section{网络流初步}
\subsection{最大流}
\subsubsection{Edmonds Karp增广路}
\begin{lstlisting}
bool BFS() {
    memset(vis, 0, sizeof(vis));
    queue<int> q;
    q.push(S); vis[S] = 1;
    incf[S] = inf; // 增广路上各边的最小剩余容量
    while (q.size()) {
        int x = q.front(); q.pop();
        for (int i = head[x]; i; i = Next[i])
            if (edge[i]) {
                int y = ver[i];
                if (vis[y]) continue;
                incf[y] = min(incf[x], edge[i]);
                pre[y] = i; // 记录前驱,便于找到最长路的实际方案
                q.push(y), vis[y] = 1;
                if (y == t) return 1;
            }
    }
    return 0;
}
void Update() { // 更新增广路及其反向边的剩余容量
    int x = t;
    while (x != s) {
        int i = pre[x];
        edge[i] -= incf[t];
        edge[i ^ 1] += incf[t]; // 利用“成对存储”的xor 1技巧
        x = ver[i ^ 1];
    }
    maxflow += incf[t];
}
\end{lstlisting}

\subsubsection{Dinic}
\begin{lstlisting}
//可加入当前弧优化(&):在增广时复制head[]到cur[],在增广时同步修改cur[],目的是递归时跳过已增广的边。
bool BFS()
{
    memset(d, 0, sizeof(d));
    queue<int> q;
    q.push(S);
    d[S] = 1; //不为1  陷入死循环
    while (q.size())
    {
        int x = q.front();
        q.pop();
        for (int i = head[x]; i; i = nxt[i])
        {
            int y = ver[i];
            if (edge[i] && !d[y])
            {
                d[y] = d[x] + 1;
                q.push(y);
                if (y == T)
                    return true;
            }
        }
    }
    return false;
}
int Dinic(int x, int flow)
{
    if (x == T)
        return flow;
    int rest = flow, k;
    for (int i = head[x]; i && rest; i = nxt[i])
    {
        int y = ver[i];
        if (edge[i] && d[y] == d[x] + 1)
        {
            k = Dinic(y, min(edge[i], rest));
            if (!k)
                d[y] = 0;
            edge[i] -= k;
            edge[i ^ 1] += k;
            rest -= k;
        }
    }
    return flow - rest;
}
\end{lstlisting}

\subsubsection{二分图最大匹配的必须边与可行边}
在一般的二分图中,可以用最大流计算任一组最大匹配。

此时:
    必须边的判定条件为:(x,y) 流量为 1 ,并且在残量网络上属于不同的SCC。

    可行边的判定条件为:(x,y) 流量为 1 ,或者在残量网络上属于同一个SCC。

例题分析

CH17C 舞动的夜晚(Dinic,Tarjan,二分图可行边)

\subsection{最小割}
\subsubsection{最大流最小割定理}
任何一个网络的最大流量等于最小割中边的容量之和。

例题分析

POJ1966 Cable TV Network (枚举,点边转化)

\subsection{费用流}
\subsubsection{Edmonds Karp 增广路}
BFS寻找增广路 -> SPFA寻找单位费用之和最小的增广路(将费用作为边权,在残量网络上求最短路)。

注意:反向边的费用为相反数。
\begin{lstlisting}
void add(int x, int y, int z, int c)
{
    ver[++tot] = y, nxt[tot] = head[x], head[x] = tot, edge[tot] = z, cost[tot] = c;
    ver[++tot] = x, nxt[tot] = head[y], head[y] = tot, edge[tot] = 0, cost[tot] = -c;
}
bool SPFA()
{
    memset(dis, 0x3f, sizeof(dis)); //inf
    memset(vis, 0, sizeof(vis));
    queue<int> q;
    dis[S] = 0, vis[S] = 1, incf[S] = 1 << 30;
    q.push(S);
    while (q.size())
    {
        int x = q.front();
        q.pop();
        vis[x] = 0;
        for (int i = head[x]; i; i = nxt[i])
        {
            if (edge[i])
            {
                int y = ver[i];
                if (dis[y] > dis[x] + cost[i])//注意修改不等号及初始化值
                {
                    dis[y] = dis[x] + cost[i];
                    incf[y] = min(incf[x], edge[i]);
                    pre[y] = i;
                    if (!vis[y])
                        q.push(y), vis[y] = 1;
                }
            }
        }
    }
    if (dis[T] >= inf)
        return false;
    return true;
}
ll max_flow, ans;
void Update()
{
    int x = T;
    while (x != S)
    {
        int i = pre[x];
        edge[i] -= incf[T];
        edge[i ^ 1] += incf[T];
        x = ver[i ^ 1];
    }
    max_flow += incf[T];
    ans += incf[T] * dis[T];
}
\end{lstlisting}
例题分析

POJ3422 Kaka's Matrix Travels(点边转化,费用流)