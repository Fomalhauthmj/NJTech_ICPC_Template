\section{Hash}
    Hash 的核心思想在于,暴力算法中,单次比较的时间太长了,应当如何才能缩短一些呢?\\
    如果要求每次只能比较$O(1)$个字符,应该怎样操作呢?\\
    我们定义一个\textbf{把string映射成int}的函数$f$,这个$f$称为是 Hash 函数。\\
    我们需要关注的是时间复杂度和 Hash 的准确率。\\
    通常我们采用的是多项式 Hash 的方法,即
    $$f(s)=\sum{(s[i]*b^i)}(\bmod M)$$
    其中$b$与$M$互质,且M越大错误率越小。(单次匹配错误率$\frac{1}{M}$,n次匹配的错误率为$\frac{n}{M}$)

\subsection{基础Hash匹配}
\begin{lstlisting}
#include <bits/stdc++.h>

using namespace std;
typedef long long ll;
// b和M互质;M可以尽量取大、随机化。对于本问题,无需建立关于M的数组,所以M最大可以达到1<<31大小。
const ll N = 1e3 + 10, b = 131, M = 1 << 20;

char s[2][N];  // s[0]是待匹配串,s[1]是模式串。下标从1开始
ll len[2];     // 两串的长度
ll Exp[N];     // Exp[i]=b^i

// 返回s[l..r]的哈希值 s[l]*Exp[1]+s[l+1]*Exp[2]+..
inline ll Hash(char s[], ll l, ll r) {
    ll i, ret = 0;
    for (i = 1; l + i - 1 <= r; i++)
        ret = (ret + Exp[i] * s[l + i - 1]) % M;
    return ret;
}

vector<ll> ans;  // 匹配子串的开始下标
inline void match() {
    ans.clear();
    ll i;
    ll h0 = Hash(s[0], 1, len[1]);  // 初始时待匹配串对应模式串那部分子串的哈希值
    ll h1 = Hash(s[1], 1, len[1]);  // 初始时模式串的哈希值
    for (i = 1; i <= len[0] - len[1] + 1; i++) {
        // 若两哈希值一致,则认为匹配
        if ((h0 - h1 * Exp[i - 1]) % M == 0)
            ans.push_back(i);
        h0 = (h0 - Exp[i] * s[0][i] + Exp[i + len[1]] * s[0][i + len[1]]) % M;  // 模式串向后移动,对应h0也要改变
    }
}

int main() {
    ios::sync_with_stdio(0);
    cin.tie(0);

    ll i, j;

    // 初始化
    Exp[0] = 1;
    for (i = 1; i < N; i++)
        Exp[i] = Exp[i - 1] * b % M;

    cin >> (s[0] + 1) >> (s[1] + 1);
    len[0] = strlen(s[0] + 1), len[1] = strlen(s[1] + 1);

    match();

    return 0;
}
\end{lstlisting}